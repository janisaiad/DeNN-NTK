
Consider a training set $\cD = \{(\x_i,y_i)\}_{i=1}^n, \x_i \in \cX \subseteq \R^d, y_i \in \cY \subseteq \R$. We will denote by $X\in\R^{n\times d}$ the matrix whose $i$th row is $\x_i^\top$. 
%For a suitable loss function $\ell$, the goal is to minimize
%the empirical loss:   $\cL(\theta)  = \frac{1}{n} \sum_{i=1}^n \ell(  y_i, \hat{y}_i) = \frac{1}{n} \sum_{i=1}^n \ell (y_i,f(\theta;\x_i))$, where the prediction $\hat{y}_i:= f(\theta;\x_i)$ is from a deep model, and the parameter vector $\theta\in\R^p$. 
In our setting $f$ is a feed-forward multi-layer (fully-connected) neural network with depth $L$~\footnote{\pcedit{The network has $L$ hidden layers, and so has depth $L+1$ if considering the output layer. However, since the term $L$ appears more frequently in our results than $L+1$, $L$ will be referred as the \emph{depth} for convenience.}}\lzcomment{the network depth is $L+1$  actually}
 and widths $m_l, l \in [L] := \{1,\ldots,L\}$ given by
\begin{equation}
\begin{split}
    &\alpha^{(0)}(\x)  = \x~, \\
     &\alpha^{(l)}(\x) = \phi \left( \frac{1}{\sqrt{m_{l-1}}} W^{(l)} \alpha^{(l-1)}(\x) \right)~,~~~l=1,\ldots,L~,\\
    &f(\theta;\x)  = \alpha^{(L+1)}(\x) = \frac{1}{\sqrt{m_L}} \v^\top \alpha^{(L)}(\x)~,
    \end{split}
\label{eq:DNN}    
\end{equation}
where $W^{(l)} \in \R^{m_l \times m_{l-1}}, l \in [L]$ are layer-wise weight matrices, $\v\in\R^{m_{L}}$ is the last layer vector, $\phi(\cdot )$ is the smooth (pointwise) activation function, and the total set of parameters is represented by the weight vector 
\begin{align}
\theta &:= (\vec(W^{(1)})^\top,\ldots,\vec(W^{(L)})^\top, \v^\top )^\top \nonumber \\
&~~~~~~\in \R^{\sum_{k=1}^L m_k m_{k-1}+m_{L}}~,
\label{eq:theta_def}
\end{align}
with $m_0=d$. \pcdelete{For simplicity, we will assume that the width of all the layers is the same, i.e., $m_l = m$, $l\in [L]$.} For simplicity, we consider deep models with only one output, i.e., $f(\theta;\x) \in \R$ as in~\citep{SD-JL-HL-LW-XZ:19}, but our results can be extended to multi-dimension outputs as in~\citep{DZ-QG:19}, using $\V \in \R^{k \times m_L}$ for $k$ outputs at the last layer. \pcedit{We use the notation $\alpha^{(l)}(\x)=\phi(\tilde{\alpha}^{(l)}(\x))$, with $\alpha^{(l)}$ being the output and  $\tilde{\alpha}^{(l)}$ the pre-activation at later $l$.}
\pcedit{We also let $A^{(l)} \in \R^{n \times m_{l}}$ be such that the $i$th row is defined as $A^{(l)}_{i,:} := \alpha^{(l)}(x_i)$, i.e., $A^{(l)}$ is the output (matrix) of layer $l\in[L]$ for input dataset $x_i, i\in[n]$ -- the weight vector $\theta$ under which this is evaluated will be understood by the context.} \pcedit{Likewise, we let $A^{(L+1)}\in \R^n$ be the vector of outputs for the input dataset.} Let $\bm{0}_p$ be the zero vector of dimension $p$ and $\I_p$ the $p\times p$ identity matrix.

\pcdelete{Define the pointwise loss $\ell_i:=\ell(y_i,\cdot) : \R \to \R_+$ and denote its first- and second-derivative as $\ell'_i := \frac{d \ell(y_i,\hat{y}_i)}{d \hat{y}_i}$ and $\ell''_i := \frac{d^2 \ell(y_i,\hat{y}_i)}{d \hat{y}_i^2}$.
%
%
The particular case of square loss is $\ell(y_i,\hat{y}_i)=(y_i-\hat{y}_i)^2$. We denote the gradient and Hessian of $f(\cdot;\x_i):\R^p \to \R$ as
$\nabla_i f := \frac{\partial f(\theta;\x_i)}{\partial \theta}$, and $\nabla^2_i f := \frac{\partial^2 f(\theta;\x_i)}{\partial \theta^2}$. The {\em neural tangent kernel} (NTK) $K_{\ntk}( \cdot ; \theta) \in \R^{n \times n}$ corresponding to parameter $\theta$ is defined as 
%\begin{align}
$K_{\ntk}(\x_i,\x_j; \theta) = \langle \nabla_i f, \nabla_j f \rangle$.
%\label{eq:ntk}
%\end{align}
By chain rule, the gradient and Hessian of the empirical loss w.r.t.~$\theta$ are given by
%\begin{align}
$\frac{\partial \cL(\theta)}{\partial \theta} = \frac{1}{n} \sum_{i=1}^n \ell_i' \nabla_i f$ and % \\
$\frac{\partial^2 \cL(\theta)}{\partial \theta^2} = \frac{1}{n} \sum_{i=1}^n \left[ \ell''_i \nabla_i f  \nabla_i f^\top + \ell'_i \nabla_i^2 f  \right]$.
%\end{align}
}

We denote the gradient and Hessian of $f(\cdot;\x_i):\R^p \to \R$ as
$\nabla_i f := \frac{\partial f(\theta;\x_i)}{\partial \theta}$, and $\nabla^2_i f := \frac{\partial^2 f(\theta;\x_i)}{\partial \theta^2}$. The {\em neural tangent kernel} (NTK) $K_{\ntk}( \cdot ; \theta) \in \R^{n \times n}$ corresponding to parameter $\theta$ is defined as 
\begin{align}
K_{\ntk}(\x_i,\x_j; \theta) = \langle \nabla_i f, \nabla_j f \rangle.
\label{eq:ntk}
\end{align}

We make the following assumption regarding the activation function $\phi$:
\begin{asmp}[\textbf{Activation function}]
The activation $\phi$ is 1-Lipschitz, i.e., $|\phi'| \leq 1$, and $\beta_\phi$-smooth, i.e., $|\phi_l''| \leq \beta_\phi$.
\label{asmp:actinit}
\end{asmp}
\begin{remark}
Our analysis holds for any $\varsigma_{\phi}$-Lipchitz smooth activations, with a dependence on $\varsigma_{\phi}$ on most key results. The main (qualitative) conclusions stay true if $\varsigma_{\phi} \leq 1 + o(1)$ or $\varsigma_{\phi} = \text{poly}(L)$, which is typically satisfied for commonly used smooth activations and moderate values of $L$. \qed
\end{remark}
%
%
\pcdelete{We define two types of balls over parameters that will be used throughout our analysis.}
%
\pcdelete{\begin{defn}[\textbf{Norm balls}]
Given $\overline{\theta}\in\R^p$ of the form~\eqref{eq:theta_def} with parameters $\overline{W}^{(l)}, l \in [L], \overline{\v}$ and
with $\| \cdot \|_2$ denoting spectral norm for matrices and $L_2$-norm for vectors, we define
\begin{align}
B_{\rho, \rho_1}^{\spec}(\bar{\theta}) & := \left\{ \theta \in \R^p ~\text{as in \eqref{eq:theta_def}} ~\mid \| W^{(\ell)} - \overline{W}^{(\ell)} \|_2 \leq \rho,\right.\nonumber \\
&\left. ~~~~~~~~~~~~~~~~~~~~~~~~~~~~~~~~~~~~~~ \ell \in [L], \| \v - \bar{\v} \|_2 \leq \rho_1 \right\}~,\label{eq:specball} \\
B_{\rho}^{\euc}(\bar{\theta}) & := \left\{ \theta \in \R^p ~\text{as in \eqref{eq:theta_def}} ~\mid \| \theta - \bar{\theta} \|_2 \leq \rho \right\}~.\label{eq:frobball}
\end{align}
\end{defn}
}

\begin{asmp}[Input data scaling]
\label{asmp:scaling}
Every input data $\x_i \in \R^d$, $i\in[n]$, has norm $\norm{\x_i}_2^2 = d$.    
\end{asmp}
\pcedit{The previous assumption is done for convenience. Scaling assumptions are common in the literature~\citep{ZAZ-YL-ZS:19,oymak2020hermite,ng2021hermite2}.} \abcomment{state this as a formal assumption} \pccomment{Done. For a moment I thought this scaling was for the optimization part and not the NTK analysis!}

%\subsection*{Spectral Norm of the Hessian}
%\label{sec:arXiv_hessian}
%
In the above setup, for a suitable initialization of the layerwise weights, one can bound the spectral norm of the Hessian in the spectral norm ball around the initialization. Such results have appeared in the recent literature~\cite{CL-LZ-MB:20,CL-LZ-MB:21,AB-PCV-LZ-MB:22}, and we suitably adapt the result in~\citep[Theorem~4.1]{AB-PCV-LZ-MB:22}.

\begin{restatable}[\textbf{Hessian Spectral Norm Bound}]{theo}{boundhess}
\label{theo:bound-Hess}
Consider Assumption~\ref{asmp:actinit} and that the elements of $W_0^{(l)}$, $l\in[L]$, are drawn i.i.d from $\cN(0,\nu_0^2)$, where $\nu_0^2 = \frac{\sigma_0^2}{c_{\phi,\sigma_0}}$ with $c_{\phi,\sigma_0} := \E_{z \sim \cN(0,\sigma_0^2)}[\phi^2(z)]$ \abdelete{$\sigma_0 = \frac{\sigma_1}{2\left(1 + \frac{2\sqrt{\log m}}{\sqrt{m}}\right)}, \sigma_1 > 0$}, and $\v_0$ is a random unit vector with $\norm{\v_0}_2=1$. Then, for $\theta \in B_{\rho,\rho_1}^{\spec}(\theta_0)$, 
%$\rho_1=O(1)$ or 
$\rho_1=O(\poly(L))$,  with probability at least $(1-\frac{2(L+1)}{m})$, we have 
\begin{equation}
\label{eq:bound_Hessian}
   \max_{i \in [n]} ~\norm{ \nabla^2_\theta f(\theta;\x_i)}_2 \leq \frac{c_H}{\sqrt{m}}~,
\end{equation}
with $c_H = O(\poly(L)(1+\gamma^{2L}))$ where $\gamma := \frac{\rho}{\sqrt{m}} + \frac{2\sigma_0}{\sqrt{c_{\phi,\sigma_0}}} \left(1 + \frac{2 \sqrt{\log m}}{\sqrt{m}} \right) $. \end{restatable}

\proof The proof follows by a direct extension of \citep[Theorem~4.1]{AB-PCV-LZ-MB:22}. \qed 

\begin{remark}
Note that the $c_{\phi,\sigma_0}$ term is a scaling factor to suitably normalize the layerwise inputs and shows up in prior work with smooth activations~\cite{SD-JL-HL-LW-XZ:19}. While such prior work have used the scaling explicitly in the model, i.e., a factor of $\sqrt{\frac{c_{\sigma}}{m c_{\phi,\sigma_0}}}$ where $c_{\sigma}$ in \cite{SD-JL-HL-LW-XZ:19} is $\frac{1}{c_{\phi,\sigma_0}}$ for us with $\sigma_0=1$, Theorem~\ref{theo:bound-Hess} has the equivalent scaling in the initialization variance. Note that we develop the results for general $\sigma_0$ so the effect of the choice of the variance is clear. \qed 
\end{remark}


\begin{remark}
Note that for $L = \tilde{O}(1)$, $c_H = \poly(L)$. More generally ... \abcomment{maybe say $m$ needs to scale as $(\frac{2\sigma_0}{\sqrt{c_{\phi,\sigma_0}}} + \frac{\cdots}{\sqrt{m}})^L$ etc., or drop this}
\end{remark}
%
%\begin{remark}[\textbf{Desirable operating regimes}]

\abdelete{We also remark that choosing $\rho_1=O(1)$ yields the same result in Theorem~\ref{theo:bound-Hess}.}
%
\pcdelete{The work~\cite[Remark~4.1]{AB-PCV-LZ-MB:22} remarks that for any choice of the spectral norm radius $\rho < \sqrt{m}$, we can choose $\sigma_1 \leq 1 - \frac{\rho}{\sqrt{m}}$ ensuring $\gamma \leq 1$ and hence $c_H = O(\text{poly}(L))$. If $\rho = O(1)$, we can keep $\sigma_1 = 1$ so that $\gamma = 1 + \frac{O(1)}{\sqrt{m}}$, and $c_H = O(\text{poly(L)})$ as long as $L < \sqrt{m}$, which is common. Both of these give good choices for $\sigma_1$ and desirable operating regime for the result. If we choose $\sigma_1 > 1$, an undesirable operating regime, then $c_H = O(c^{\Theta(L)})$, $c >1$, and we will need $m = \Omega(c^{\Theta(L)})$ for the result to be of interest.}
%
\pcdelete{\pcedit{When the elements of $\v_0$ are drawn i.i.d from $\cN(0,\sigma_0^2)$ --- our setting ---,~\cite{AB-PCV-LZ-MB:22} show that, under the choices aforementioned for the parameters of the spectral ball, we can obtain $c_H=O(\polylog(m)\poly(L))$.}}
%\qed 
%\label{rem:gamma}
%\end{remark}
%
\pcedit{
\begin{remark}[\textbf{Difference in balls around initialization}]
%
Unlike the work~\citep{AB-PCV-LZ-MB:22} which considers the spectral ball $B_{\rho,\rho_1}^{\spec}(\theta_0)$ around the initialization point $\theta_0\in\R^d$, we consider the Euclidean ball $B_{\rho}^{\euc}(\theta_0)$, which has been a more common assumption in the literature~\citep{CL-LZ-MB:20}. Since $B_{\rho}^{\euc}(\theta_0)\subseteq B_{\rho,\rho}^{\spec}(\theta_0)$, the result in Theorem~\ref{theo:bound-Hess} also holds for our setting.
\end{remark}}\abcomment{not sure we need the last remark, we can state in terms of spectral norm ball}


\abcomment{We have not described how the initialization weights $\theta_0$ (or $W_0$) are sampled, copy from the old draft, we can show how $\sigma_0^2$ gets used.}
\pccomment{The thing is that the way the initialization weights are initialized are on the statement of Theorem~4.1. We would have to change it.} \abcomment{Got it}