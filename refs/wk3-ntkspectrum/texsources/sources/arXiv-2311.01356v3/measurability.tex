
\section{A note on measurability} \label{app:measurable}
In this appendix we elaborate on the measurability of the function
\begin{equation} \label{eq:map}
(W^{(0)}, ..., W^{(L)}, b^{(0)}, ..., b^{(L)}) \mapsto \underset{x \in \RR^d}{\sup} \Vert W^{(L)}D^{(L-1)}(x)W^{(L-1)} \cdots D^{(0)}(x) W^{(0)}\Vert_2.
\end{equation}
Note that we computed (upper bounds for) the expectation of this function with respect to the weights $W^{(0)}, ..., W^{(L)}$ and the biases $b^{(0)}, ..., b^{(L)}$ in \Cref{sec:upper} in order to establish upper bounds for the Lipschitz constant of ReLU networks. In order for the expectation to make sense, the function needs to be measurable, which we verify (to a sufficient extent) in this appendix. 


 
As a preparation, we recall from \cite[Section~1.5]{cohn2013measure} the notion of the \emph{completion} of a measure. 
Given a measure $\mu$ on a measurable space $(\Omega, \mathscr{A})$, we define
\begin{equation*}
\mathscr{A}_\mu \defeq \left\{ A \subseteq \Omega : \ \exists \ E,F \in \mathscr{A}: \ E \subseteq A \subseteq F \text{ and } \mu(F \setminus E) = 0\right\}.
\end{equation*}
Then, $\mathscr{A}_\mu$ is a $\sigma$-algebra with $\mathscr{A} \subseteq \mathscr{A}_\mu$, and we can extend $\mu$ to a well-defined measure $\overline{\mu}$ on $\mathscr{A}_\mu$ by setting
\begin{equation*}
\overline{\mu}(A) \defeq \mu(E) \quad \text{for} \quad E,F \in \mathscr{A} \quad \text{with} \quad E \subseteq A \subseteq F \quad \text{and} \quad \mu(F \setminus E) = 0.
\end{equation*} 
 Moreover, the measure space $(X, \mathscr{A}_\mu, \overline{\mu})$ is then \emph{complete}, meaning that $A \in \mathscr{A}_\mu$, $\overline{\mu}(A)= 0$ and $B \subseteq A$ together imply $B \in \mathscr{A}_\mu$ for all subsets $A,B \subseteq \Omega$.
 
 Next, given a measurable space $(\Omega, \mathscr{A})$, we say that $A \subseteq \Omega$ is \emph{universally measurable} if $A \in \mathscr{A}_\mu$ for \emph{every} finite measure $\mu: \mathscr{A} \to [0, \infty]$. It is easy to see that the universally measurable sets over $(\Omega, \mathscr{A})$ form a $\sigma$-algebra, and that every $A \in \mathscr{A}$ is universally measurable. 
 
 We show that the map considered in \eqref{eq:map} is universally measurable, i.e., that the preimage of every Borel set is universally measurable. 
\begin{proposition}
Let $\arrow{W} \defeq (W^{(0)}, ..., W^{(L)}) \in \RR^{N \times d} \times \RR^{N \times N} \times \cdots \times \RR^{N \times N} \times \RR^{1 \times N} \cong \RR^{D_1}$ and $\arrow{b} \defeq (b^{(0)}, ..., b^{(L)}) \in \RR^N \times \cdots \times \RR^N \times \RR \cong \RR^{D_2}$. Then the map
\begin{equation*}
\xi: \quad \RR^{D_1} \times \RR^{D_2} \to \RR , \quad (\arrow{W}, \arrow{b}) \mapsto \underset{x \in \RR^d}{\sup} \Vert W^{(L)}D^{(L-1)}(x)W^{(L-1)} \cdots D^{(0)}(x) W^{(0)}\Vert_2
\end{equation*}
is universally measurable. That is, $\xi^{-1}(U)$ is universally measurable over $(\RR^{D_1} \times \RR^{D_2}, \mathscr{B})$ (with $\mathscr{B}$ denoting the Borel $\sigma$-algebra over $\RR^{D_1} \times \RR^{D_2}$) for any Borel measurable set $U \subseteq \RR$. Here, the matrices $D^{(L-1)}(x), ..., D^{(0)}(x)$ are as defined in \Cref{subsec:gradient}.
\end{proposition}
\begin{proof}
According to \cite[Corollary 4.25]{aliprantis_infinite_2006} it suffices to show that $\xi^{-1}((a, \infty))$ is universally measurable for every $a \in \RR$ in order to show that $\xi$ is universally measurable. 

Note that
\begin{align*}
\xi^{-1}((a,\infty)) &= \left\{(\arrow{W}, \arrow{b}) \in \RR^{D_1 + D_2}: \ \exists x \in \RR^d \text{ with } \Vert W^{(L)}D^{(L-1)}(x) \cdots D^{(0)}(x) W^{(0)}\Vert_2 > a\right\} \\
&= \pi \left(\left\{ (\arrow{W}, \arrow{b}, x) \in \RR^{D_1} \times \RR^{D_2} \times \RR^d: \ \Vert W^{(L)}D^{(L-1)}(x) \cdots D^{(0)}(x) W^{(0)}\Vert_2 > a\right\}\right).
\end{align*}
Here, 
\begin{equation*}
\pi: \quad \RR^{D_1 + D_2 + d} \to \RR^{D_1 + D_2}, \quad (\arrow{W}, \arrow{b}, x) \mapsto (\arrow{W}, \arrow{b})
\end{equation*}
denotes the projection from $\RR^{D_1 + D_2 + d}$ onto $\RR^{D_1 + D_2}$. It is easy to see that the map
\begin{equation*}
\RR^{D_1 + D_2 + d} \to \RR, \quad (\arrow{W}, \arrow{b}, x) \mapsto \Vert W^{(L)}D^{(L-1)}(x)W^{(L-1)} \cdots D^{(0)}(x) W^{(0)}\Vert_2
\end{equation*}
is Borel measurable, whence the set 
\begin{equation*}
\left\{ (\arrow{W}, \arrow{b}, x) \in \RR^{D_1} \times \RR^{D_2} \times \RR^d: \ \Vert W^{(L)}D^{(L-1)}(x) \cdots D^{(0)}(x) W^{(0)}\Vert_2 > a\right\}
\end{equation*}
is Borel measurable. Since $\pi$ is trivially Borel measurable, it follows from \cite[Theorem 12.24]{aliprantis_infinite_2006} that $\xi^{-1}((a, \infty))$ is an analytic set and hence, $\xi^{-1}((a, \infty))$ is universally measurable by \cite[Theorem 12.41]{aliprantis_infinite_2006}.
\end{proof}
It remains unclear, whether $\xi$ is not merely universally measurable, but in fact Borel measurable. 

In order for the expectation in \Cref{sec:upper} to be well-defined we have to require that the map
\begin{equation*}
\mathscr{W}: \quad \Omega \to \RR^{D_1 + D_2}, \quad \omega \mapsto (\arrow{W}(\omega), \arrow{b}(\omega)),
\end{equation*}
is measurable with respect to $\mathscr{A}$ and the $\sigma$-algebra of universally measurable sets on $\RR^{D_1 + D_2}$. Here, $(\Omega, \mathscr{A}, \PP)$ denotes the underlying ``master probability space''. This can be achieved by defining the probability space to be $\Omega \defeq \RR^{D_1 + D_2}$ together with the $\sigma$-algebra of universally measurable subsets, $\mathscr{W} = \id$ and $\PP$ as the (completion of the) probability measure representing the distribution of $(\arrow{W}, \arrow{b})$ according to \Cref{assum:1}.
