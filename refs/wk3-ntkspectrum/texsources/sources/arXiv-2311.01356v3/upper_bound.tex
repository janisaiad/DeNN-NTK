
\section{Proof of the upper bound} \label{sec:upper}

Referring to \eqref{eq:lowbound}, the goal of this section is to establish upper bounds for
\begin{equation*}
\underset{x \in \RR^d}{\sup}\Vert W^{(L)} \cdot D^{(L-1)}(x) \cdot W^{(L-1)} \cdots D^{(0)}(x)\cdot W^{(0)}\Vert_2
\end{equation*}
that hold with high probability and in expectation, where the randomness is over the random matrices $W^{(0)} , ..., W^{(L)}$ and the random bias vectors $b^{(0)}, ..., b^{(L-1)}$. For this to make sense one first needs to know that 
\begin{equation} \label{eq:suppp}
\underset{x \in \RR^d}{\sup}\Vert W^{(L)} \cdot D^{(L-1)}(x) \cdot W^{(L-1)}\cdots D^{(0)}(x)\cdot W^{(0)}\Vert_2
\end{equation} 
is indeed measurable; we refer to \Cref{app:measurable} for a proof of this fact. 

Throughout the entire section, we assume that \Cref{assum:1} is satisfied. 
Since the random matrices $W^{(\ell)}$ and the random biases $b^{(\ell)}$ are jointly independent it is possible to calculate the expectation iteratively by first assuming that the matrices
 $W^{(0)}, ... ,W^{(L-1)}$ and the biases $b^{(0)}, ..., b^{(L-1)}$ are fixed and deriving an upper bound when only $W^{(L)}$ is assumed to be random. Then as the final step we are also going to allow randomness in $W^{(0)}, ..., W^{(L-1)}$ and $b^{(0)}, ..., b^{(L-1)}$ to get the desired result. In other words, we are conditioning on $W^{(0)}, ..., W^{(L-1)}, b^{(0)}, ..., b^{(L-1)}$.

The central tool for deriving these bounds is \emph{Dudley's inequality} which can be found for example in \cite[Theorems~5.25~and~5.29]{van2014probability}. We refer to \Cref{app:dudley} for details on Dudley's inequality. The key idea of this section is contained in the following proposition. 
\begin{proposition}\label{prop:key}
Let the matrices $W^{(0)}, ..., W^{(L-1)}$ and the biases $b^{(0)},..., b^{(L-1)}$ be fixed and set $\Lambda \defeq \Vert W^{(L-1)} \Vert_2 \cdots \Vert W^{(0)} \Vert_2$. For $x \in \RR^d$ and $z \in \overline{B}_d(0,1)$ we define
\begin{equation*}
Y_{z,x} \defeq D^{(L-1)}(x) W^{(L-1)}\cdots D^{(0)}(x) \cdot W^{(0)}z \in \RR^N
\end{equation*}
and further
\begin{equation*}
\mathcal{L} = \mathcal{L}(d,N,L,W^{(0)}, ..., W^{(L-1)}, b^{(0)},..., b^{(L-1)})\defeq \left\{ Y_{z,x}: \ x \in \RR^d, \ z \in \overline{B}_d(0,1)\right\} \subseteq \RR^N. 
\end{equation*}
Then there exists an absolute constant $C>0$ such that the following holds: Given any $u \geq 0$, we have
\begin{equation*}
\underset{x \in \RR^d}{\sup} \Vert W^{(L)} \cdot D^{(L-1)}(x)\cdot W^{(L-1)} \cdots D^{(0)}(x) \cdot W^{(0)}\Vert_2 \leq  C \cdot \left( \int_0^{\Lambda} \sqrt{\ln \left(\mathcal{N}(\mathcal{L}, \Vert \cdot \Vert_2, \eps)\right)} \ \dd \eps + u \Lambda \right)
\end{equation*}
with probability at least $(1 - 2\exp(-u^2))$ (with respect to the choice of $W^{(L)}$). Moreover, 
\begin{equation*}
\underset{W^{(L)}}{\EE} \left[ \underset{x \in \RR^d}{\sup} \Vert W^{(L)} \cdot D^{(L-1)}(x) \cdot W^{(L-1)}\cdots D^{(0)}(x)\cdot W^{(0)}\Vert_2\right] \leq C \cdot \int_0^{\Lambda}  \sqrt{\ln \left(\mathcal{N}(\mathcal{L}, \Vert \cdot \Vert_2, \eps)\right)} \ \dd \eps.
\end{equation*}
\end{proposition}
\begin{proof}
For $y \in \mathcal{L}$ it holds that there are $x \in \RR^d$ and $z \in \overline{B}_d(0,1)$ satisfying 
\begin{equation*}
y=Y_{z,x} = D^{(L-1)}(x) W^{(L-1)}\cdots D^{(0)}(x) W^{(0)}z. 
\end{equation*}
We compute
\begin{align*}
\Vert y \Vert_2 &\leq \underbrace{\Vert D^{(L-1)}(x)\Vert_2}_{\leq 1} \cdot \Vert W^{(L-1)}\Vert_2 \cdots \underbrace{\Vert D^{(0)}(x) \Vert_2}_{\leq 1} \cdot \Vert W^{(0)} \Vert_2 \cdot \underbrace{\Vert z \Vert_2}_{\leq 1} \\
&\leq \Vert W^{(L-1)} \Vert_2 \cdots \Vert W^{(0)} \Vert_2 = \Lambda.
\end{align*}
Hence, since $0 = D^{(L-1)}(0) W^{(L-1)}\cdots D^{(0)}(0)W^{(0)}0 \in \mathcal{L}$ it follows 
\begin{equation}\label{eq:cov=1}
\mathcal{N}(\mathcal{L}, \Vert \cdot \Vert_2, \eps) = 1 \quad \text{for } \eps \geq  \Lambda.
\end{equation}
To get the final result we rewrite
\begin{align*}
\underset{x \in \RR^d}{\sup} \Vert W^{(L)}D^{(L-1)}(x)\cdot W^{(L-1)}\cdots D^{(0)}(x) W^{(0)} \Vert_2  &=  \underset{x \in \RR^d, z \in \overline{B}_d(0,1)}{\sup} \left\langle \left(W^{(L)}\right)^T, Y_{z,x}\right\rangle  \\
&=   \underset{Y \in \mathcal{L}}{\sup} \left\langle \left(W^{(L)}\right)^T, Y \right\rangle.
\end{align*}
From the observation $\mathcal{L} \subseteq \overline{B}_N(0, \Lambda)$ we infer $\diam(\mathcal{L}) \leq 2\Lambda$. \Cref{prop:dudley} and \eqref{eq:cov=1} then yield the claim by noticing once again that $0 \in \mathcal{L}$.
\end{proof}
Given the above proposition, the problem of bounding the Lipschitz constant of random ReLU networks has been transferred to bounding the covering numbers of the set $\mathcal{L}$. Finding upper bounds for these covering numbers is the essential task in the following two subsections. In fact, in the following we show that
\begin{equation*}
\mathcal{N}(\mathcal{L}, \Vert \cdot \Vert_2, \eps) \leq \left(\frac{9 \Vert W^{(L-1)} \Vert_2 \cdots \Vert W^{(0)} \Vert_2}{\eps} \right)^{C(d+1)} \cdot \left(\frac{\ee N}{d+1}\right)^{L(d+1)},
\end{equation*}
at least if $N > d+2$. Here, $C>0$ is an absolute constant. However, in the case of shallow networks, i.e., $L=1$, it is possible to show that the above inequality holds without the additional factor $\left(\frac{\ee N}{d+1}\right)^{L(d+1)}$, without the assumption $N > d+2$ and $d$ can be replaced by $\min\{N,d\}$, which in the end leads to a sharper bound on the Lipschitz constant. Therefore, the cases of shallow and deep networks are treated separately in \Cref{sec:shallow,sec:deep}, respectively. The final bounds on the covering numbers can be found in \Cref{lem:cov_num_bound,lem:cov_bound}.


\subsection{The shallow case}


\label{sec:shallow}
Firstly, we consider shallow neural networks, i.e., networks that only have a single hidden layer and can hence be written as
\begin{equation*}
\left(x \mapsto (W^{(1)} \cdot x + b^{(1)})\right) \circ \relu \circ \left( x \mapsto (W^{(0)}\cdot x + b^{(0)}) \right).
\end{equation*}
As already explained above, we are from now on going to assume that the matrix $W^{(0)}$ and the vector $b^{(0)}$ are fixed and only assume randomness in $W^{(1)}$ and $b^{(1)}$.
\begin{lemma} \label{lem:scalarproduct_alternative_form}
Let $W^{(0)} \in \RR^{N \times d}$ and $b^{(0)} \in \RR^N$ be fixed. We recall that for $\alpha \in \RR^{d+1}$ we define
\begin{equation*}
f_\alpha: \quad \RR^{d+1} \to \{0,1\}, \quad x \mapsto \mathbbm{1}_{\alpha^Tx > 0}.
\end{equation*}
Furthermore, for any vector $z \in \RR^{d}$, let
\begin{equation*}
\tau_{\alpha, z} \defeq (W^{(0)}z) \odot \left( f_\alpha\left((W^{(0)}_{1,-}, b^{(0)}_1)^T\right) , ..., f_\alpha \left((W^{(0)}_{N,-}, b^{(0)}_N)^T\right)\right) \in \RR^N.
\end{equation*}
Then the following two statements hold:
\begin{enumerate}
\item{$Y_{z,x}=  \tau_{(x,1)^T, z}$ for every $x \in \RR^d$ and $z \in \B_d(0,1)$,}
\item{\label{item:lem_2}$\Vert \tau_{\alpha,z} \Vert_2 \leq \Vert W^{(0)} \Vert_2 \cdot \Vert z \Vert_2$ for all $\alpha \in \RR^{d+1}, z \in \RR^d$.}
\end{enumerate}
Here, $Y_{z,x}$ is as introduced in \Cref{prop:key}.
\end{lemma}

\begin{proof}
\leavevmode
\begin{enumerate}
\item{ Let $x \in \RR^d$ and $z \in \B_d(0,1)$. For every $i \in \{1,...,N\}$ we calculate
\begin{align*}
(D^{(0)}(x) \cdot W^{(0)} \cdot z)_i &= \mathbbm{1}_{W^{(0)}_{i,-} x + b^{(0)}_i > 0} \cdot \left(W^{(0)}z\right)_i \\
&= \left(W^{(0)}z\right)_i \cdot f_{(x,1)^T}\left(\left(W^{(0)}_{i,-}, b^{(0)}_i\right)^T\right) = \left(\tau_{(x,1)^T,z}\right)_i,
\end{align*}
which yields the claim.}
\item{It is immediate that
\begin{equation*}
\Vert \tau_{\alpha, z} \Vert_2 \leq \Vert W^{(0)}z \Vert_2 \cdot \underset{i=1,...,N}{\max} \underbrace{\abs{f_\alpha\left(\left(W^{(0)}_{i,-}, b^{(0)}_i\right)^T\right)}}_{\leq 1} \leq \Vert W^{(0)} \Vert_2 \cdot \Vert z \Vert_2. \qedhere
\end{equation*}
}
\end{enumerate}
\end{proof}

The desired bound for the covering number of $\mathcal{L}$ in the case of shallow networks is contained in the following lemma. 
\begin{lemma} \label{lem:cov_num_bound}
Assume that $W^{(0)} \in \RR^{N \times d}$ and $b^{(0)} \in \RR^N$ are fixed. There exists an absolute constant $C>0$ such that, writing $k \defeq \rang(W^{(0)})$, for every $\varepsilon \in (0, \Vert W^{(0)} \Vert_2)$ it holds
\begin{equation*}
\mathcal{N}(\mathcal{L}, \Vert \cdot \Vert_2, \eps) \leq \left(\frac{9 \Vert W^{(0)} \Vert_2}{\varepsilon}\right)^{C\cdot k}.
\end{equation*}
Here, $\mathcal{L}$ is as introduced in \Cref{prop:key}.
\end{lemma}
\begin{proof}
Without loss of generality, we assume that $W^{(0)} \neq 0$, since otherwise $(0, \Vert W^{(0)} \Vert_2) = \emptyset$. Note that according to \Cref{lem:scalarproduct_alternative_form} we can write
\begin{equation*}
\mathcal{L} = \left\{\tau_{(x,1)^T, v}: \ x \in \RR^d, v \in \B_d(0,1) \right\}.
\end{equation*}
We can even weaken this identity and infer
\begin{equation*}
\mathcal{L} = \left\{\tau_{(x,1)^T, v}: \ x \in \ker(W^{(0)})^{\perp}, v \in \B_d(0,1) \cap \ker(W^{(0)})^{\perp} \right\}.
\end{equation*}
Here, $\ker(W^{(0)})^{\perp}$ denotes the orthogonal complement of $\ker(W^{(0)})$. We further note 
\begin{equation*}
\dim(\ker(W^{(0)})^\perp) = k.
\end{equation*}

Let $\eps \in (0, \Vert W^{(0)} \Vert_2)$. From \Cref{prop:covering_ball} we infer the existence of a natural number $M \in \NN$ with $M \leq \left( \frac{8 \Vert W^{(0)} \Vert_2}{\varepsilon} + 1\right)^k$ and $v_1, ..., v_M \in \B_d(0,1) \cap \ker(W^{(0)})^\perp$ such that
\begin{equation*}
\B_d(0,1) \cap \ker(W^{(0)})^\perp \subseteq \bigcup_{i=1}^M \overline{B}_d\left(v_i,\frac{\varepsilon}{ 4 \Vert W^{(0)} \Vert_2}\right) \cap \ker(W^{(0)})^\perp.
\end{equation*}
For $i \in \{1,...,M\}$ let $w_i \defeq W^{(0)}v_i$. Fix $i \in \{1,...,M\}$ and first assume $w_i \neq 0$. Define a probability measure $\mu_i$ on $\ker(W^{(0)})^{\perp} \times \RR$ by
\begin{equation*}
\mu_i \defeq \frac{1}{\Vert w_i \Vert_2^2} \cdot \sum_{\ell = 1}^N (w_i)_\ell^2 \cdot \delta_{\left(W^{(0)}_{\ell,-}, b^{(0)}_\ell\right)^T}.
\end{equation*}
Here, we note that by definition it holds $\left(W^{(0)}_{\ell,-}\right)^T \in \ker(W^{(0)})^\perp$ for every $\ell \in \{1,...,N\}$.

Let $\mathcal{F} \defeq \left\{ \fres{f_\alpha}{\ker(W^{(0)})^\perp \times \RR} : \ \alpha \in \ker(W^{(0)})^\perp \times \RR\right\}$, where $f_\alpha$ is as introduced in the previous \Cref{lem:scalarproduct_alternative_form}. From \Cref{prop:vc_half_spaces_2} we infer that 
\begin{equation*}
\vc(\mathcal{F})  = k+1. 
\end{equation*}
Further, \Cref{prop:covering_vc} shows for every $\delta \in (0,1)$ that
\begin{equation*}
\mathcal{N}(\mathcal{F}, L^2(\mu_i), \delta) \leq \left( \frac{2}{\delta}\right)^{C'(k+1)}
\end{equation*}
with an absolute constant $C'>0$. Thus, there exists $K_i \in \NN$ with $K_i \leq \left(\frac{8 \Vert W^{(0)} \Vert_2}{\varepsilon}\right)^{C'(k+1)}$ and vectors $\alpha_1^{(i)},..., \alpha_{K_i}^{(i)} \in \ker(W^{(0)})^\perp \times \RR$ such that
\begin{equation*}
\mathcal{F} \subseteq \bigcup_{j=1}^{K_i} \overline{B}_\mathcal{F}^{L^2(\mu_i)} \left(\fres{f_{\alpha_j^{(i)}}}{\ker(W^{(0)})^\perp \times \RR},\frac{\epsilon}{4\Vert W^{(0)} \Vert_2}\right).
\end{equation*}
If $w_i=0$ let $\alpha_j^{(i)} \defeq 0 \in \ker(W^{(0)})^\perp \times \RR$ for $1\leq j \leq K_i \defeq 1$. 

Now, let $v \in \B_d(0,1) \cap \ker(W^{(0)})^\perp$ and $x \in \ker(W^{(0)})^\perp$ be arbitrary. Then there exists $i \in \{1,...,M\}$ such that 
\begin{equation} \label{eq:v_bound}
\Vert v - v_i \Vert_2 \leq \frac{\varepsilon}{4\Vert W^{(0)} \Vert_2}.
\end{equation} 
Let us first consider the case $w_i = W^{(0)}v_i \neq 0$. Then there exists $j \in \{1,..., K_i\}$ such that
\begin{equation*}
\left\Vert \fres{f_{(x,1)^T}}{\ker(W^{(0)})^\perp \times \RR} - \fres{f_{\alpha_j^{(i)}}}{\ker(W^{(0)})^\perp \times \RR}\right\Vert_{L^2(\mu_i)} \leq \frac{\epsilon}{4\Vert W^{(0)} \Vert_2}.
\end{equation*} 
We compute
\begin{align}
\left\Vert \tau_{(x,1)^T,v}- \tau_{\alpha_j^{(i)}, v_i}\right\Vert_2 &\leq \left\Vert \tau_{(x,1)^T,v} - \tau_{(x,1)^T, v_i}\right\Vert_2 + \left\Vert \tau_{(x,1)^T, v_i} - \tau_{\alpha_j^{(i)},v_i}\right\Vert_2 \nonumber\\
&= \left\Vert \tau_{(x,1)^T, v- v_i}\right\Vert_2 + \left\Vert \tau_{(x,1)^T, v_i} - \tau_{\alpha_j^{(i)},v_i}\right\Vert_2 \nonumber\\
\overset{\text{Lemma}~\ref{lem:scalarproduct_alternative_form}~\eqref{item:lem_2}}&{\leq} \Vert W^{(0)} \Vert_2 \cdot \Vert v - v_i \Vert_2 + \left\Vert \tau_{(x,1)^T, v_i} - \tau_{\alpha_j^{(i)},v_i}\right\Vert_2 \nonumber\\
\label{eq:first_bound}
\overset{\eqref{eq:v_bound}}&{\leq} \frac{\varepsilon}{4} + \left\Vert \tau_{(x,1)^T, v_i} - \tau_{\alpha_j^{(i)},v_i}\right\Vert_2.
\end{align}
Finally, we note because of $w_i = W^{(0)}v_i$ and by definition of $\mu_i$ that
\begin{align}
\left\Vert \tau_{(x,1)^T, v_i} - \tau_{\alpha_j^{(i)}, v_i}\right\Vert_2^2 &= \sum_{\ell = 1}^N (W^{(0)}v_i)_\ell^2 \cdot \left( f_{(x,1)^T}\left((W^{(0)}_{\ell, -}, b^{(0)}_\ell)^T\right) - f_{\alpha_j^{(i)}}\left((W^{(0)}_{\ell, -}, b^{(0)}_\ell)^T\right)\right)^2 \nonumber\\
&= \Vert w_i \Vert_2^2 \cdot \left\Vert \fres{f_{(x,1)^T}}{\ker(W^{(0)})^\perp \times \RR} - \fres{f_{\alpha_j^{(i)}}}{\ker(W^{(0)})^\perp \times \RR}\right\Vert^2_{L^2(\mu_i)} \nonumber \\
&\leq \Vert w_i \Vert_2^2 \cdot \left(\frac{\eps}{4\Vert W^{(0)} \Vert_2}\right)^2 
\label{eq:second_bound}\leq \Vert W^{(0)} \Vert_2^2 \cdot \Vert v_i \Vert_2^2 \cdot \left(\frac{\eps}{4\Vert W^{(0)} \Vert_2}\right)^2 \leq \left(\frac{\eps}{4}\right)^2,
\end{align}
and this trivially remains true in the case $w_i = 0$ if we choose $j=1$.

Overall, \eqref{eq:first_bound} and \eqref{eq:second_bound} together imply in any case that
\begin{equation*}
\left\Vert \tau_{(x,1)^T, v} - \tau_{\alpha_j^{(i)},v_i} \right\Vert_2 \leq \frac{\eps}{2}.
\end{equation*}
Hence, the set 
\begin{equation*}
\left\{ \tau_{\alpha_j^{(i)}, v_i}: \ 1 \leq i \leq M, \ 1 \leq j \leq K_i\right\}
\end{equation*}
is an $\frac{\eps}{2}$-net of $\mathcal{L}$ with respect to $\Vert \cdot \Vert_2$. However, this set does not necessarily have to be a subset of $\mathcal{L}$. Yet, using \cite[Exercise 4.2.9]{vershynin_high-dimensional_2018} for the first inequality, we get
\begin{align*}
\mathcal{N}(\mathcal{L}, \Vert \cdot \Vert_2, \eps) &\leq \sum_{i=1}^M K_i \leq \left(\frac{8 \Vert W^{(0)} \Vert_2}{\eps} + 1\right)^k \cdot \left(\frac{8\Vert W^{(0)} \Vert_2}{\eps}\right)^{C'(k+1)} \\
\overset{\eps < \Vert W^{(0)} \Vert_2}&{\leq} \left(\frac{9\Vert W^{(0)} \Vert_2}{\eps}\right)^{C'(k+1) + k} 
\leq \left( \frac{9\Vert W^{(0)} \Vert_2}{\eps}\right)^{(2C'+1)k},
\end{align*}
so the claim follows choosing $C= 2C' + 1$.
\end{proof}

The derived bound for the covering number of $\mathcal{L}$ leads to the following bound when we only assume randomness in $W^{(1)}$. 
\begin{proposition}\label{thm:lower_bound_1}
There exists an absolute constant $C>0$ such that for fixed $W^{(0)} \in \RR^{N \times d}$ and $b^{(0)} \in \RR^N$, writing $k = \rang(W^{(0)})$, the following hold:
\begin{enumerate}
\item{
For any $u \geq 0$, we have
\begin{equation*}
\underset{x \in \RR^d}{\sup} \left\Vert W^{(1)} \cdot D^{(0)}(x) \cdot W^{(0)}\right\Vert_2 \leq C \cdot \Vert W^{(0)} \Vert_2 \cdot (\sqrt{k} + u)
\end{equation*}
with probability at least $1 - 2\exp(-u^2)$ (with respect to the choice of $W^{(1)}$).}
\item{$\displaystyle
\underset{W^{(1)}}{\EE} \left[ \underset{x \in \RR^d}{\sup} \left\Vert W^{(1)} \cdot D^{(0)}(x) \cdot W^{(0)}\right\Vert_2\right] \leq C \cdot \sqrt{k} \cdot \Vert W^{(0)} \Vert_2.$
}
\end{enumerate}
\end{proposition}
\begin{proof}
Without loss of generality we assume $k \geq 1$. We observe
\begin{align*}
 \int_0^{\Vert W^{(0)}\Vert_2} \sqrt{\ln (\mathcal{N}(\mathcal{L}, \Vert \cdot \Vert_2, \eps))} \ \dd\eps \overset{\text{Lemma}~\ref{lem:cov_num_bound}}&{\leq}  \int_0^{\Vert W^{(0)} \Vert_2}\sqrt{C_1 \cdot k} \cdot \sqrt{\ln \left( \frac{9 \Vert W^{(0)} \Vert_2}{\eps}\right)} \ \dd\eps \\
&= \sqrt{C_1} \cdot \sqrt{k} \cdot 9 \Vert W^{(0)} \Vert_2 \cdot \int_0^{\frac{1}{9}} \sqrt{\ln (1/\sigma)} \ \dd\sigma \\
& \leq C_2 \cdot \sqrt{k} \cdot \Vert W^{(0)} \Vert_2.
\end{align*}
Here, $C_1>0$ is the absolute constant from \Cref{lem:cov_num_bound} and $C_2 \defeq 9\cdot\sqrt{C_1} \cdot \int_0^{1/9} \sqrt{\ln (1/\sigma)} \ \dd \sigma$. At the equality, we applied the substitution $\frac{1}{\sigma} = \frac{9 \Vert W^{(0)} \Vert_2}{\eps}$. We combine this estimate with \Cref{prop:key} and get for any $u \geq 0$ that
\begin{align*}
\underset{x \in \RR^d}{\sup} \left\Vert W^{(1)} \cdot D^{(0)}(x) \cdot W^{(0)}\right\Vert_2 &\leq C_3 \cdot \left(C_2 \cdot \sqrt{k} \cdot \Vert W^{(0)} \Vert_2 + u \cdot \Vert W^{(0)} \Vert_2 \right) \\
&\leq C_3 \cdot \max\{1, C_2\} \cdot \Vert W^{(0)} \Vert_2 \cdot (\sqrt{k} + u)
\end{align*}
with probability at least $1 - 2\exp(-u^2)$, as well as
\begin{equation*}
\underset{W^{(1)}}{\EE} \left[ \underset{x \in \RR^d}{\sup} \Vert W^{(1)}D^{(0)}(x) W^{(0)} \Vert_2\right] \leq C_3 \cdot C_2 \cdot \sqrt{k} \cdot \Vert W^{(0)} \Vert_2,
\end{equation*}
where $C_3 > 0$ is the absolute constant from \Cref{prop:key}. Hence, the claim follows by letting $C \defeq C_3\max\{C_2,1\}$.
\end{proof}
Until now, we have conditioned on $W^{(0)}, b^{(0)}$. Reintroducing the randomness with respect to $W^{(0)}, b^{(0)}$ leads to the following statement.
\begin{proposition} \label{thm:pre_main}
There exist absolute constants $C, c_1 > 0$ such that, writing $k \defeq \min\{d,N\}$, the following hold:
\begin{enumerate}
\item{ \label{item1:pre_main}
For any $u,t \geq 0$ it holds
\begin{equation*}
\underset{x \in \RR^d}{\sup} \Vert W^{(1)} \cdot D^{(0)}(x) \cdot W^{(0)}\Vert_2 \leq C\cdot \left(1 + \frac{\sqrt{d} + t}{\sqrt{N}}\right)(\sqrt{k} + u)
\end{equation*}
with probability at least $(1-2\exp(-u^2))_+ \cdot (1-2\exp(-c_1t^2))_+$ with respect to the choice of $W^{(0)}, W^{(1)}, b^{(0)}, b^{(1)}$.
}
\item{ \label{item2:pre_main}
$ \displaystyle
\EE \left[ \underset{x \in \RR^d}{\sup} \left\Vert W^{(1)} \cdot D^{(0)}(x) \cdot W^{(0)}\right\Vert_2\right] \leq C \cdot \left(1+ \frac{\sqrt{d} }{\sqrt{N}} \right) \cdot \sqrt{k}.
$
}
\end{enumerate}
\end{proposition}
\begin{proof}
We first note that for any matrix $W^{(0)}$ it holds $\rang(W^{(0)}) \leq k$. 

Let us first deal with Part \eqref{item1:pre_main}. Let $C_2 > 0$ be the (absolute) constant from \Cref{thm:lower_bound_1} and $C \defeq \sqrt{2}C_2$. For fixed $u,t \geq 0$ let 
\begin{equation*}
A \defeq \left\{ (W^{(1)}, W^{(0)}, b^{(0)}): \ \underset{x \in \RR^d}{\sup} \Vert W^{(1)} \cdot D^{(0)}(x) \cdot W^{(0)}\Vert_2 \leq C \left(1 + \frac{\sqrt{d} + t}{\sqrt{N}}\right)(\sqrt{k} + u)\right\}.
\end{equation*}
Furthermore let 
\begin{equation*}
A_1 \defeq \left\{ (W^{(0)}, b^{(0)}) : \ \Vert W^{(0)} \Vert_2 \leq \sqrt{2} \left(1 + \frac{\sqrt{d} + t}{\sqrt{N}}\right)\right\}
\end{equation*}
and for fixed $(W^{(0)}, b^{(0)})$, let
\begin{equation*}
A_2\left(W^{(0)}, b^{(0)}\right) \defeq  \left\{ W^{(1)}: \ \underset{x \in \RR^d}{\sup} \Vert W^{(1)} \cdot D^{(0)}(x) \cdot W^{(0)}\Vert_2 \leq C_2 \cdot \Vert W^{(0)}  \Vert_2 \cdot (\sqrt{k} + u)\right\}.
\end{equation*}
Note that then
\begin{equation*}
(W^{(0)}, b^{(0)}) \in A_1, \ W^{(1)} \in A_2\left(W^{(0)}, b^{(0)}\right) \quad \Longrightarrow \quad  (W^{(1)}, W^{(0)}, b^{(0)}) \in A.
\end{equation*}
From \Cref{thm:lower_bound_1} and since probabilities are always non-negative we infer
\begin{equation*}
\PP^{W^{(1)}} \left(A_2\left(W^{(0)}, b^{(0)}\right)\right) \geq (1 - 2\exp(-u^2))_+
\end{equation*}
for any $(W^{(0)}, b^{(0)})$. Furthermore, it holds 
\begin{equation*}
\Vert W^{(0)} \Vert_2 = \sqrt{\frac{2}{N}} \cdot \left\Vert \sqrt{\frac{N}{2}} \cdot W^{(0)} \right\Vert_2 \leq \sqrt{\frac{2}{N}}\cdot (\sqrt{N} + \sqrt{d} + t) = \sqrt{2}\left(1 + \frac{\sqrt{d} + t}{\sqrt{N}}\right)
\end{equation*}
with probability at least $(1 - 2 \exp(-c_1 t^2))_+$ for some absolute constant $c_1 > 0$, as follows from \cite[Corollary 7.3.3]{vershynin_high-dimensional_2018} by noting that the matrix $\sqrt{\frac{N}{2}} W^{(0)}$ has independent $\mathcal{N}(0,1)$-entries. The claim of Part \eqref{item1:pre_main} then follows from \Cref{prop:highprob}.

Let us now deal with Part \eqref{item2:pre_main}. Using \Cref{thm:lower_bound_1} we derive
\begin{equation*}
\underset{W^{(0)},b^{(0)},W^{(1)}}{\EE} \left[ \underset{x \in \RR^d}{\sup} \left\Vert W^{(1)} \cdot D^{(0)}(x) \cdot W^{(0)}\right\Vert_2 \right]\leq C_2 \cdot \sqrt{k} \cdot \underset{W^{(0)}}{\EE} \ \Vert W^{(0)} \Vert_2.
\end{equation*}
From \cite[Theorem 7.3.1]{vershynin_high-dimensional_2018} we get
\begin{equation*}
\underset{W^{(0)}}{\EE} \ \Vert W^{(0)} \Vert_2 = \sqrt{\frac{2}{N}} \cdot \underset{W^{(0)}}{\EE} \   \left\Vert \sqrt{\frac{N}{2}} \cdot W^{(0)} \right\Vert_2 \leq\sqrt{\frac{2}{N}}(\sqrt{N} + \sqrt{d}) =\sqrt{2} \left(1 + \sqrt{\frac{d}{N}}\right).
\end{equation*}
This yields the claim.
\end{proof}

The transfer of \Cref{thm:pre_main} to obtain an upper bound of the Lipschitz constant of shallow ReLU networks follows directly from \eqref{eq:lowbound}.
\begin{theorem} \label{thm:1_main}
There exist absolute constants $C, c_1 > 0$ such that if $\Phi: \RR^d \to \RR$ is a random shallow ReLU network with width $N$ satisfying \Cref{assum:1} and writing $k \defeq \min\{d,N\}$, the following hold:
\begin{enumerate}
\item{For any $u,t \geq 0$, we have
\begin{equation*}
\lip(\Phi) \leq C \cdot \left(1 + \frac{\sqrt{d} + t}{\sqrt{N}}\right)(\sqrt{k} + u)
\end{equation*}
with probability at least $(1-2\exp(-u^2))_+ \cdot (1-2\exp(-c_1t^2))_+$.}
\item{$ \displaystyle
\EE \left[\lip(\Phi) \right]  \leq C \cdot \left(1+ \frac{\sqrt{d} }{\sqrt{N}} \right) \cdot \sqrt{k}.
$}
\end{enumerate}
\end{theorem}


By plugging in certain values for $u$ and $t$, we can now prove \Cref{thm:main_1}.
\renewcommand*{\proofname}{Proof of \Cref{thm:main_1}}
\begin{proof}
Let $\widetilde{C}$ and $\widetilde{c_1}$ be the relabeled constants from \Cref{thm:1_main} and $k \defeq \min\{d,N\}$, as well as $\ell \defeq \max\{d,N\}$. To show Part (1), we set $u = \sqrt{k}$ and $t = \sqrt{\ell}$. Then, since the inequality $\sqrt{\ln(2)} < \sqrt{\ln(\ee)} = 1$ holds, we get $u \geq \sqrt{\ln(2)}$ and thus $1-2\exp(-u^2) \geq 0$. \Cref{thm:1_main} shows 
\begin{equation*}
\lip(\Phi) \leq 2\widetilde{C} \cdot \left(1 + \frac{\sqrt{d} +\sqrt{\ell}}{\sqrt{N}}\right) \sqrt{k} 
\end{equation*}
with probability at least $(1-2\exp(-k)) \cdot (1-2\exp(-\widetilde{c_1} \ell))_+$. If $d \leq N$ we get
\begin{equation*}
2\widetilde{C}\cdot \left(1 + \frac{\sqrt{d} +\sqrt{\ell}}{\sqrt{N}}\right) \sqrt{k} \leq 2\widetilde{C}\cdot  \left(1 + \frac{\sqrt{N} + \sqrt{N}}{\sqrt{N}}\right) \cdot \sqrt{d} = 6\widetilde{C} \cdot \sqrt{d}
\end{equation*}
and if $d \geq N$ we infer
\begin{align*}
2\widetilde{C}\cdot \left(1 + \frac{\sqrt{d} +\sqrt{\ell}}{\sqrt{N}}\right) \sqrt{k} = 2\widetilde{C}\cdot \left(1 + \frac{\sqrt{d} +\sqrt{d}}{\sqrt{N}}\right) \sqrt{N}= 2\widetilde{C}\cdot \left(\sqrt{N} + \sqrt{d} + \sqrt{d}\right) \leq 6\widetilde{C} \cdot \sqrt{d}.
\end{align*}
For Part (2) note that, if $d \leq N$ is satisfied, \Cref{thm:1_main} gives us
\begin{equation*}
\EE \left[\lip(\Phi) \right]  \leq \widetilde{C}\cdot \left(1+ \frac{\sqrt{d} }{\sqrt{N}} \right) \cdot \sqrt{d} \leq 2 \widetilde{C} \cdot \sqrt{d}
\end{equation*}
and if $d \geq N$ we get
\begin{equation*}
\EE \left[\lip(\Phi) \right]  \leq \widetilde{C} \cdot\left(1+ \frac{\sqrt{d} }{\sqrt{N}} \right) \cdot \sqrt{N} = \widetilde{C}\cdot \left(\sqrt{N} + \sqrt{d}\right) \leq 2 \widetilde{C}\cdot \sqrt{d}.
\end{equation*}
Overall, the claim is satisfied with $C \defeq  6\widetilde{C}$ and $c_1 \defeq \widetilde{c_1}$.
\end{proof}
\renewcommand*{\proofname}{Proof}


\subsection{The deep case}
\newcommand{\D}{\mathscr{D}}
\label{sec:deep}
In the following, we treat the case of deep networks, meaning $L > 1$. The proofs of this section also apply in the case of shallow networks, but are only relevant in the case of deep networks, since in the case of shallow networks better bounds have been derived in the preceding subsection. Again, we first assume that the matrices $W^{(0)}, ..., W^{(L-1)}$ and the biases $b^{(0)}, ..., b^{(L-1)}$ are fixed and the matrix $W^{(L)}$ is initialized randomly according to \Cref{assum:1}. In this setting, we denote
\begin{equation*}
\Lambda \defeq \Vert W^{(L-1)} \Vert_2 \cdots \Vert W^{(0)} \Vert_2.
\end{equation*}

As the first step, we derive a bound on the cardinality of the set of all possible combinations $(D^{(L-1)}(x), ..., D^{(0)}(x))$ of the $D$-matrices occurring in the formula for $\nabla \Phi(x)$ from \Cref{prop:grad_relu}. The finiteness of this set is immediate since it consists of tuples of diagonal $N \times N$-matrices with zeros and ones on the diagonal. Naively, one can bound the cardinality by $2^{LN}$ but in fact, it is possible to prove a much stronger bound. 
\begin{lemma}\label{lem:D_card}
Assume $d+2 <N$. For fixed $W^{(0)},..., W^{(L-1)}$ and $b^{(0)}, ..., b^{(L-1)}$ we define
\begin{equation*}
\mathscr{D} \defeq \left\{ \left(D^{(L-1)}(x),..., D^{(0)}(x)\right): \ x \in \RR^d\right\},
\end{equation*}
with $D^{(0)}(x),..., D^{(L-1)}(x)$ as defined in \Cref{subsec:gradient}.
Then it holds that
\begin{equation*}
\vert \mathscr{D} \vert \leq \left(\frac{\ee N}{d+1}\right)^{L(d+1)}.
\end{equation*}
\end{lemma}
\begin{proof}
For any $0\leq \ell \leq L-1$ we define
\begin{equation*}
\D^{(\ell)} \defeq \left\{ \left(D^{(\ell)}(x), ..., D^{(0)}(x)\right): \ x \in \RR^d\right\}
\end{equation*}
and claim that it holds
\begin{equation}\label{eq:ind_d}
\vert \D^{(\ell)} \vert \leq \left(\frac{\ee N}{d+1}\right)^{(\ell + 1)(d+1)},
\end{equation}
which we will show by induction over $\ell$. 

To this end, we start with the case $\ell = 0$. Using the notation introduced in \cref{prop:vc_half_spaces_2}, we see for any $i \in \{1,...,N\}$ and $x \in \RR^d$ that 
\begin{equation*}
\left(D^{(0)}(x)\right)_{i,i} = f_{(x,1)^T}\left((W^{(0)}_{i,-}, b^{(0)}_{i})^T\right).
\end{equation*}
By definition of $D^{(0)}(x)$, this yields
\begin{align*}
\vert\D^{(0)} \vert &= \left\vert \left\{ \left(f_{(x,1)^T}\left((W^{(0)}_{1,-}, b^{(0)}_{1})^T\right),..., f_{(x,1)^T}\left((W^{(0)}_{N,-}, b^{(0)}_{N})^T\right)\right): \ x \in \RR^d\right\}\right\vert \\
&\leq \left\vert \left\{ \left(f_{\alpha}\left((W^{(0)}_{1,-}, b^{(0)}_{1})^T\right),..., f_{\alpha}\left((W^{(0)}_{N,-}, b^{(0)}_{N})^T\right)\right): \ \alpha \in \RR^{d+1}\right\}\right\vert \leq \left(\frac{\ee N}{d+1}\right)^{d+1}.
\end{align*}
To obtain the last inequality we employed Sauer's lemma (cf. \cite[Lemma 6.10]{shalev2014understanding}) and the estimate for the VC-dimension of halfspaces (cf. \Cref{prop:vc_half_spaces_2}) and the assumption $d+2 < N$.

We now assume that the claim holds for some fixed $0 \leq \ell \leq L-2$. We then see 
\begin{align*}
\D^{(\ell + 1)} &= \left\{ \left(D^{(\ell+1)}(x), ..., D^{(0)}(x)\right): \ x \in \RR^d\right\} \\
&= \! \!\biguplus_{(C^{(\ell)}, ..., C^{(0)}) \in \D^{(\ell)}} \! \left\{ \left(D^{(\ell + 1)}(x), C^{(\ell)}, ..., C^{(0)}\right): \! \ x \in \RR^d \!\text{ with } \! D^{(j)}(x)=C^{(j)} \ \!\text{for } \! \text{all } \! j\in \{0,...,\ell\}\right\}
\end{align*}
and hence
\begin{equation} \label{eq:D_bound}
\abs{\D^{(\ell + 1)}} = \sum_{(C^{(\ell)}, ..., C^{(0)}) \in \D^{(\ell)}} \abs{\left\{ D^{(\ell + 1)}(x): \ x \in \RR^d \text{ with } D^{(j)}(x)=C^{(j)} \ \text{for all } j\in \{0,...,\ell\}\right\}}.
\end{equation}
We thus fix $(C^{(\ell)}, ..., C^{(0)}) \in \D^{(\ell)}$ and seek to bound $\abs{\mathcal{A}}$ where
\begin{equation*}
\mathcal{A} \defeq \left\{ D^{(\ell + 1)}(x): \ x \in \RR^d, \  D^{(j)}(x) = C^{(j)} \ \text{for all } j\in \{0,...,\ell\}\right\}.
\end{equation*}
With $x^{(\ell)}$ as in \Cref{subsec:gradient} (for $1 \leq \ell \leq L$), it is immediate that
\begin{align*}
&\norel\abs{\mathcal{A}} \\
&= \abs{\left\{\left(\mathbbm{1}_{W^{(\ell + 1)}_{1,-}x^{(\ell+1)} + b^{(\ell + 1)}_1 > 0}, ..., \mathbbm{1}_{W^{(\ell + 1)}_{N,-}x^{(\ell+1)} + b^{(\ell + 1)}_N > 0}\right):  x \in \RR^d \!\text{ with }  \! \forall \  0\leq j \leq \ell\!:\!D^{(j)}(x)=C^{(j)}\right\}}.
\end{align*}
Fix $x \in \RR^d$ with $D^{(j)}(x)=C^{(j)}$ for every $j \in \{0,..., \ell\}$. From the definition of $x^{(\ell+1)}$ (see \Cref{subsec:gradient}) we may write
\begin{equation*}
x^{(\ell+1)} = C^{(\ell)}W^{(\ell)} \cdots C^{(0)}W^{(0)}x + \bar{c},
\end{equation*} 
where $\bar{c} = \bar{c}(C^{(0)}, ..., C^{(\ell)}, W^{(0)}, ..., W^{(\ell)}, b^{(0)}, ..., b^{(\ell)})\in \RR^N$ is a fixed vector. We thus get for any $i \in \{1,...,N\}$ that
\begin{equation*}
W^{(\ell + 1)}_{i,-}x^{(\ell+1)} + b^{(\ell+1)}_i = W^{(\ell + 1)}_{i,-}C^{(\ell)}W^{(\ell)} \cdots C^{(0)}W^{(0)}x + c_i,
\end{equation*}
where $c = c(C^{(0)}, ..., C^{(\ell)}, W^{(0)}, ..., W^{(\ell+1)}, b^{(0)}, ..., b^{(\ell+1)}) \in \RR^N$ is fixed. Writing 
\begin{equation*}
V \defeq W^{(\ell + 1)}C^{(\ell)}W^{(\ell)} \cdots C^{(0)}W^{(0)} \in \RR^{N \times d} 
\end{equation*}
we infer
\begin{align*}
\abs{\mathcal{A}} &= \left\vert \left\{ \left(f_{(x,1)^T}\left((V_{1,-}, c_{1})^T\right),..., f_{(x,1)^T}\left((V_{N,-}, c_{N})^T\right)\right): \ x \in \RR^d\right\}\right\vert \\
&\leq \left\vert \left\{ \left(f_{\alpha}\left((V_{1,-}, c_{1})^T\right),..., f_{\alpha}\left((V_{N,-}, c_{N})^T\right)\right): \ \alpha \in \RR^{d+1}\right\}\right\vert \leq \left(\frac{\ee N}{d+1}\right)^{d+1}
\end{align*}
where we again used \cite[Lemma 6.10]{shalev2014understanding} and \Cref{prop:vc_half_spaces_2} for the last inequality, again noting that we assumed $d+2 < N$. 

Combining this result with \eqref{eq:D_bound} and the induction hypothesis, we see
\begin{equation*}
\abs{\D^{(\ell+1)}} \leq \abs{\D^{(\ell)}} \cdot \left(\frac{\ee N}{d+1}\right)^{d+1} \leq \left(\frac{\ee N}{d+1}\right)^{(\ell + 2)(d+1)}.
\end{equation*}
By induction this shows that \eqref{eq:ind_d} holds for all $\ell \in \{0,...,L-1\}$. The statement of the lemma then follows by noting that $\D = \D^{(L-1)}$.
\end{proof}

Having established a bound for the cardinality of the set $\mathscr{D}$, we can now bound the covering numbers of $\mathcal{L}$ in the case of deep networks. 
\begin{lemma}\label{lem:cov_bound}
Let $N > d+2$ and $W^{(0)},..., W^{(L-1)}$ and $b^{(0)}, ..., b^{(L-1)}$ be fixed. Moreover, let
\begin{equation*}
\Lambda \defeq \Vert W^{(L-1)} \Vert_2 \cdots \Vert W^{(0)} \Vert_2.
\end{equation*}
Then for every arbitrary $\varepsilon \in (0, \Lambda)$ it holds that
\begin{equation*}
\mathcal{N}(\mathcal{L}, \Vert \cdot \Vert_2, \varepsilon) \leq \left(\frac{3 \Lambda}{\eps} \right)^d \cdot \left(\frac{\ee N}{d+1}\right)^{L(d+1)}.
\end{equation*}
Here, $\mathcal{L}$ is as defined in \Cref{prop:key}. 
\end{lemma}
\begin{proof}
We can assume that $\Lambda > 0$ since otherwise $(0, \Lambda) = \emptyset$. Using the notation $\mathscr{D}$ as introduced previously in \cref{lem:D_card}, we see immediately that
\begin{equation*}
\mathcal{L} = \bigcup_{(C^{(L-1)}, ..., C^{(0)}) \in \D} \left[ C^{(L-1)}W^{(L-1)} \cdots C^{(0)}W^{(0)} \B_d(0,1) \right]
\end{equation*}
and hence it holds for every $\eps > 0$ that
\begin{equation}\label{eq:cov_num}
\mathcal{N}(\mathcal{L}, \Vert \cdot \Vert_2, \eps) \leq \sum_{(C^{(L-1)}, ..., C^{(0)}) \in \D} \mathcal{N}\left(C^{(L-1)}W^{(L-1)} \cdots C^{(0)}W^{(0)} \B_d(0,1), \Vert \cdot \Vert_2, \eps \right).
\end{equation}
This can be seen from the fact that the union of $\eps$-nets of $C^{(L-1)}W^{(L-1)} \cdots C^{(0)}W^{(0)} \B_d(0,1)$, where $(C^{(L-1)}, ..., C^{(0)})$ runs through all elements of $\D$, is an $\eps$-net of $\mathcal{L}$.

Fix $(C^{(L-1)}, ..., C^{(0)}) \in \D$ and define $V \defeq C^{(L-1)}W^{(L-1)} \cdots C^{(0)}W^{(0)}$. 
From \Cref{prop:covering_ball} we infer that there are $w_1, ..., w_M \in \B_d(0,1)$ such that
\begin{equation*}
\B_d(0,1) \subseteq \bigcup_{i=1}^M \overline{B}_d\left(w_i,\frac{\eps}{\Lambda}\right)
\end{equation*}
with $M \leq \left(\frac{2 \Lambda}{\eps} + 1\right)^d$. 
Let $v_i \defeq Vw_i$ for every $i \in \{1,...,M\}$. 
Hence, it holds $v_i \in V\B_d(0,1)$ for every $i \in \{1,...,M\}$. Let $u \in V \B_d(0,1)$ be arbitrary and choose $u' \in \B_d(0,1)$ satisfying $ u = Vu'$. 
By choice of $w_1, ..., w_M$ there exists $i \in \{1,...,M\}$ with 
\begin{equation*}
\Vert u' - w_i \Vert_2 \leq \frac{\eps}{\Lambda}.
\end{equation*}
But then it holds
\begin{align*}
  \Vert u - v_i \Vert_2 =\Vert u - Vw_i \Vert_2 
&\leq \Vert V  \Vert_2 \cdot \Vert u' - w_i \Vert_2 \\
&\leq \underbrace{\Vert C^{(L-1)} \Vert_2}_{\leq 1} \cdot \Vert W^{(L-1)} \Vert_2 \cdots \underbrace{\Vert C^{(0)} \Vert_2}_{\leq 1} \cdot \Vert W^{(0)} \Vert_2 \cdot \frac{\eps}{\Vert W^{(L-1)} \Vert_2 \cdots \Vert W^{(0)} \Vert_2} \\
&\leq \eps.
\end{align*}
Hence, we conclude
\begin{equation*}
\mathcal{N}\left(V\B_d(0,1), \Vert \cdot \Vert_2, \eps \right) \leq \left(\frac{2 \Lambda}{\eps} + 1\right)^d \leq \left(\frac{3 \Lambda}{\eps}\right)^d,
\end{equation*}
since $\eps \leq \Lambda$. Combining this estimate with Equation \eqref{eq:cov_num} and \cref{lem:D_card} yields the claim.
\end{proof}

The following proposition establishes the final bound when the expectation is calculated only with respect to $W^{(L)}$, i.e., conditioning on $W^{(0)}, ..., W^{(L-1)}, b^{(0)}, ..., b^{(L-1)}$. 
\begin{proposition}\label{thm:deep_final}
Let the matrices $W^{(L-1)}, ..., W^{(0)}$ and the biases $b^{(L-1)}, ..., b^{(0)}$ be fixed and define $\Lambda \defeq \Vert W^{(L-1)} \Vert_2 \cdots \Vert W^{(0)} \Vert_2$. Moreover, assume that $d+2 <  N$. Then the following hold, with an absolute constant $C>0$:
\begin{enumerate}
\item{ \label{item:deep_highprob} For any $u \geq 0$, we have
\begin{equation*}
\underset{x \in \RR^d}{\sup} \Vert W^{(L)}D^{(L-1)}(x)W^{(L-1)}\cdots D^{(0)}(x) W^{(0)} \Vert_2 \leq C \cdot \Lambda \cdot \sqrt{L} \cdot \sqrt{\ln \left(\frac{\ee N}{d+1}\right)} \cdot (\sqrt{d} + u)
\end{equation*}
with probability at least $(1 - 2\exp(-u^2))$ (with respect to the choice of $W^{(L)}$).
}
\item{
$\displaystyle
\underset{W^{(L)}}{\EE} \left[\underset{x \in \RR^d}{\sup} \Vert W^{(L)}D^{(L-1)}(x)W^{(L-1)}\cdots D^{(0)}(x) W^{(0)} \Vert_2 \right] \leq C \cdot \Lambda \cdot \sqrt{L} \cdot \sqrt{\ln \left(\frac{\ee N}{d+1}\right)} \cdot \sqrt{d}.
$
}
\end{enumerate}
\end{proposition}
\begin{proof}
Using \Cref{lem:cov_bound} and the elementary inequality $\sqrt{x+y}  \leq \sqrt{x} + \sqrt{y}$ for $x,y \geq 0$ we infer
\begin{align*}
\sqrt{\ln \left(\mathcal{N}(\mathcal{L}, \Vert \cdot \Vert_2, \eps)\right)} &\leq \sqrt{\ln \left(\left(\frac{3 \Lambda}{\eps} \right)^d \cdot \left(\frac{\ee N}{d+1}\right)^{L(d+1)}\right)} \\
&\leq \sqrt{d} \cdot \sqrt{\ln \left(\frac{3 \Lambda}{\eps}\right)} + \sqrt{L}\cdot \sqrt{d+1} \cdot \sqrt{\ln \left(\frac{\ee N}{d+1}\right)}
\end{align*}
for any $\eps \in (0, \Lambda)$ where $\mathcal{L}$ is as in \Cref{prop:key}. This yields
\begin{equation} \label{eq:proof1}
\int_0^\Lambda \sqrt{\ln \left(\mathcal{N}(\mathcal{L}, \Vert \cdot \Vert_2, \eps)\right)} \ \dd \eps \leq \sqrt{d} \cdot \int_0^\Lambda \sqrt{\ln \left(\frac{3 \Lambda}{\eps}\right)} \ \dd \eps + \Lambda \cdot \sqrt{L}\cdot \sqrt{d+1} \cdot \sqrt{\ln \left(\frac{\ee N}{d+1}\right)}.
\end{equation}
Using the substitution $\sigma = \frac{\eps}{3\Lambda}$ we get
\begin{align} \label{eq:proof2}
\int_0^\Lambda \sqrt{\ln \left(\frac{3 \Lambda}{\eps}\right)} \ \dd \eps = 3\Lambda \cdot \int_0^{\frac{1}{3}} \sqrt{\ln \left( 1/\sigma\right)}\ \dd \sigma \leq C_1 \cdot \Lambda
\end{align}
with $C_1 \defeq 3 \cdot \int_0^{1/3} \sqrt{\ln \left( 1/\sigma\right)} \dd \sigma$. Overall, we thus see
\begin{align}
\int_0^\Lambda \sqrt{\ln \left(\mathcal{N}(\mathcal{L}, \Vert \cdot \Vert_2, \eps)\right)} \ \dd \eps \overset{\eqref{eq:proof1}, \eqref{eq:proof2}}&{\leq} \Lambda \cdot \left( C_1  \cdot \sqrt{d} + \sqrt{L}\cdot \sqrt{d+1} \cdot \sqrt{\ln \left(\frac{\ee N}{d+1}\right)}\right) \nonumber\\
\overset{d < d+1 \leq 2d}&{\leq} \sqrt{2} \cdot \max\{1, C_1\}  \cdot \Lambda \cdot \left( \sqrt{d} + \sqrt{L} \cdot \sqrt{d} \cdot \sqrt{\ln \left(\frac{\ee N}{d+1}\right)}\right) \nonumber\\
\label{alig:int}
\overset{L \geq 1, N \geq d+1}&{\leq} \underbrace{2\sqrt{2} \cdot \max\{1, C_1\}}_{=: C_2}  \cdot \Lambda \cdot \sqrt{L} \cdot \sqrt{\ln \left(\frac{\ee N}{d+1}\right)} \cdot \sqrt{d}. 
\end{align}
We can now prove \eqref{item:deep_highprob}. From \Cref{prop:key} we obtain an absolute constant $C_3>0$ such that for any $u \geq 0$ the estimate
\begin{align*}
\underset{x \in \RR^d}{\sup} \Vert W^{(L)} \cdot D^{(L-1)}(x)W^{(L-1)} \cdots D^{(0)}(x)W^{(0)}\Vert_2 &\leq C_3 \left( C_2  \cdot \Lambda \cdot \sqrt{L} \cdot \sqrt{\ln \left(\frac{\ee N}{d+1}\right)} \cdot \sqrt{d} + u \cdot \Lambda \right) \\
&\leq C_3C_2 \cdot \Lambda \cdot \sqrt{L} \cdot \sqrt{\ln \left(\frac{\ee N}{d+1}\right)} \cdot (\sqrt{d} + u)
\end{align*}
holds with probability at least $(1 - 2 \exp(-u^2))$ with respect to $W^{(L)}$. 

For the expectation bound, simply note that 
\begin{align*}
\underset{W^{(L)}}{\EE} \left[ \underset{x \in \RR^d}{\sup} \Vert W^{(L)} \cdot D^{(L-1)}(x)W^{(L-1)} \cdots D^{(0)}(x)W^{(0)}\Vert_2\right] \leq C_3C_2 \cdot \Lambda \cdot \sqrt{L} \cdot \sqrt{\ln \left(\frac{\ee N}{d+1}\right)} \cdot \sqrt{d}
\end{align*}
follows directly from \Cref{prop:key}. Hence, the claim follows letting $C \defeq C_2C_3$.
\end{proof}

Incorporating randomness in $W^{(0)}, ..., W^{(L-1)}$ and $b^{(0)}, ..., b^{(L-1)}$ leads to the following theorem.
\begin{theorem} \label{thm:deep_finall}
There exist absolute constants $C, c_1 > 0$ such that for $N > d+2$, random weight matrices $W^{(0)},...,W^{(L)}$ and random bias vectors $b^{(0)},...,b^{(L)}$ as in \Cref{assum:1} the following hold:
\begin{enumerate}
\item{
For every $u,t \geq 0$ we have 
\begin{align*}
&\norel \underset{x \in \RR^d}{\sup} \Vert W^{(L)}D^{(L-1)}(x)W^{(L-1)}\cdots D^{(0)}(x) W^{(0)} \Vert_2 \\
&\leq C \cdot \left(1 + \frac{\sqrt{d} + t}{\sqrt{N}}\right)\left(2\sqrt{2} + \frac{\sqrt{2} t}{\sqrt{N}}\right)^{L-1} \cdot \sqrt{L} \cdot \sqrt{\ln \left(\frac{\ee N}{d+1}\right)} \cdot (\sqrt{d} + u)
\end{align*}
with probability at least $(1-2\exp(-u^2))_+\left((1-2\exp(-c_1 t^2))_+\right)^L$ with respect to $W^{(0)},..., W^{(L)}$ and $b^{(0)},...,b^{(L)}$.
}
\item{
$\displaystyle
 \norel\EE\left[\underset{x \in \RR^d}{\sup} \Vert W^{(L)}D^{(L-1)}(x)W^{(L-1)}\cdots D^{(0)}(x) W^{(0)} \Vert_2 \right] $ \\
$\displaystyle \leq C \cdot \left( 1 + \frac{\sqrt{d}}{\sqrt{N}}\right) \cdot (2 \sqrt{2})^{L-1} \cdot \sqrt{L} \cdot \sqrt{\ln \left(\frac{\ee N}{d+1}\right)} \cdot \sqrt{d}.$
}
\end{enumerate}
\end{theorem}
\begin{proof}
Let $\widetilde{C}$ be the relabeled constant from \Cref{thm:deep_final} and define $C \defeq \sqrt{2} \widetilde{C}$. We start with (1): In view of \Cref{thm:deep_final,prop:highprob} it suffices to show that for every $t \geq 0$ the estimate
\begin{equation*}
\Lambda \leq \sqrt{2}\left(1 + \frac{\sqrt{d} + t}{\sqrt{N}}\right)\left(2\sqrt{2} + \frac{\sqrt{2}t}{\sqrt{N}}\right)^{L-1}
\end{equation*}
holds with probability at least $\left((1-2\exp(-c_1 t^2))_+\right)^L$, where
\begin{equation*}
\Lambda \defeq \Vert W^{(L-1)}\Vert_2 \cdots \Vert W^{(0)} \Vert_2.
\end{equation*}
To show this, note that \cite[Corollary 7.3.3]{vershynin_high-dimensional_2018} yields
\begin{equation*}
\Vert W^{(0)} \Vert_2 = \sqrt{\frac{2}{N}} \left\Vert \sqrt{\frac{N}{2}} W^{(0)}\right\Vert_2\leq \sqrt{2} \left(1 + \frac{\sqrt{d} + t}{\sqrt{N}}\right)
\end{equation*}
with probability at least $(1- 2\exp(-c_1 t^2))_+$, as well as
\begin{equation*}
\Vert W^{(\ell)} \Vert_2 = \sqrt{\frac{2}{N}} \left\Vert \sqrt{\frac{N}{2}} W^{(\ell)} \right\Vert_2 \leq \sqrt{\frac{2}{N}} (2 \sqrt{N} + t)= 2\sqrt{2} + \frac{\sqrt{2}t}{\sqrt{N}}
\end{equation*}
with probability at least $(1- 2\exp(-c_1 t^2))_+$ for any $1 \leq \ell \leq L-1$. Hence, the claim follows from an iterative application of \Cref{prop:highprob}.

To conclude (2) we apply \cite[Theorem 7.3.1]{vershynin_high-dimensional_2018} to the matrices $\sqrt{\frac{N}{2}} W^{(\ell)}$ for $0 \leq \ell \leq L-1$. This yields $\underset{W^{(\ell)}}{\EE} \Vert W^{(\ell)} \Vert_2 \leq 2 \sqrt{2}$ for every $1 \leq \ell \leq L-1$ and further $\underset{W^{(0)}}{\EE} \Vert W^{(0)}\Vert_2\leq \sqrt{2} \left(1 + \frac{\sqrt{d}}{\sqrt{N}} \right)$. The independence of the matrices $W^{(\ell)}$ combined with \Cref{thm:deep_final} then yields the claim.
\end{proof}

Now we obtain an upper bound on the Lipschitz constant directly using \Cref{thm:deep_finall} and \eqref{eq:lowbound}:
\begin{theorem}\label{thm:final_deep_lipschitz}
Let $\Phi: \RR^d \to \RR$ be a random ReLU network of width $N$ and with $L$ hidden layers satisfying \Cref{assum:1}. Moreover, let $d+2<N$. Then the following hold, for certain absolute constants $C, c_1 > 0:$
\begin{enumerate}
\item{For every $u,t \geq 0$, we have
\begin{align*}
\lip(\Phi)\leq C \cdot \left(1 + \frac{\sqrt{d} + t}{\sqrt{N}}\right)\left(2\sqrt{2} + \frac{\sqrt{2} t}{\sqrt{N}}\right)^{L-1} \cdot \sqrt{L} \cdot \sqrt{\ln \left(\frac{\ee N}{d+1}\right)} \cdot (\sqrt{d} + u)
\end{align*}
with probability at least $(1-2\exp(-u^2))_+\left((1-2\exp(-c_1 t^2))_+\right)^L$.
}
\item{
$\displaystyle
\EE \left[\lip(\Phi) \right]
\leq C \cdot \left( 1 + \frac{\sqrt{d}}{\sqrt{N}}\right) \cdot (2 \sqrt{2})^{L-1} \cdot \sqrt{L} \cdot \sqrt{\ln \left(\frac{\ee N}{d+1}\right)} \cdot \sqrt{d}.
$
}
\end{enumerate}
\end{theorem}
 By plugging in special values for $t$ and $u$ and using $d \leq N$ we can now prove \Cref{thm:main_2}.

\renewcommand*{\proofname}{Proof of \Cref{thm:main_2}}
\begin{proof}
Let $\widetilde{C}$ and $\widetilde{c_1}$ be the relabeled constants from \Cref{thm:final_deep_lipschitz} and let $C \defeq 6\widetilde{C}$ and $c_1 \defeq \widetilde{c_1}$. 
Part (1) follows from \Cref{thm:final_deep_lipschitz} by plugging in $u = \sqrt{d}$ and $t = \sqrt{N}$, which yields
\begin{equation*}
\lip(\Phi)\leq \widetilde{C} \cdot 3 \cdot (3\sqrt{2})^{L-1} \cdot \sqrt{L} \cdot \sqrt{\ln \left(\frac{\ee N}{d+1}\right)} \cdot 2\cdot \sqrt{d} = 6\widetilde{C} \cdot (3\sqrt{2})^{L-1} \cdot \sqrt{L}\cdot \sqrt{\ln \left(\frac{\ee N}{d+1}\right)} \cdot \sqrt{d}
\end{equation*}
with probability at least $(1-2\exp(-d))\left((1-2\exp(-c_1 N))_+\right)^L$,
where we also used $d \leq N$ and $1-2\exp(-u^2) = 1-2\exp(-d) \geq 0$.
Part (2) follows immediately from $d \leq N$ and part (2) of \Cref{thm:final_deep_lipschitz}.
\end{proof}
\renewcommand*{\proofname}{Proof}

