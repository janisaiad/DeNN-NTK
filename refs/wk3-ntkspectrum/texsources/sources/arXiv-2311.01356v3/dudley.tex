
\section{On Dudley's inequality} \label{app:dudley}

In this section we add important notes on Dudley's inequality,
which is crucial for deriving the upper bounds in \Cref{sec:upper}. 

We first define the notion of separability of a random process. 

\begin{definition}[{cf. \cite[Definition 5.22]{van2014probability}}]
A random process $(X_t)_{t \in T}$ on a probability space $(\Omega, \mathscr{A}, \PP)$ indexed by a metric space $(T, \varrho)$
is called \emph{separable} if there exists a countable set $T_0 \subseteq T$ and a $\PP$-null set $N \subseteq \Omega$
such that for all $\omega \in \Omega \setminus N$
and $t \in T$ there exists a sequence $(s_n)_{n \in \NN} \in T_0^\NN$
with $s_n \to t$ and 
\begin{equation*}
\lim_{n  \to \infty} X_{s_n}(\omega) = X_t(\omega).
\end{equation*} 
\end{definition}

\begin{proposition} \label{prop:equal}
Let $(\Omega, \mathscr{A}, \PP)$ be a probability space, $(T, \varrho)$ a separable metric space and $(X_t)_{t \in T}$ a random process with $X_t: \Omega \to \RR$ for every $t \in T$. Moreover, we assume that for every fixed $\omega \in \Omega$ the map
\begin{equation*}
T \to \RR, \quad t \mapsto X_t(\omega)
\end{equation*}
is continuous. Then $(X_t)_{t \in T}$ is a separable random process. 
\end{proposition}

\newcommand{\T}{\widetilde{T}}
\begin{proof}
Let $T_0 \subseteq T$ be a countable dense subset of $T$ and $\omega \in \Omega, \ t \in T$.
Since $T_0$ is dense there is a sequence $(s_n)_{n \in \NN} \in T_0^\NN$ with $s_n \to t$.
Then, using the continuity assumption, it holds
\begin{equation*}
\lim_{n \to \infty} \ X_{s_n}(\omega) = X_t (\omega),
\end{equation*} 
which yields the claim.
\end{proof}

We can now formulate and prove a slight extension of Dudley's inequality from \cite{van2014probability}.

\begin{theorem}[{cf.~\cite[Theorems~5.25~and~5.29]{van2014probability}}] \label{thm:dudleyyy}
There is a constant $C>0$ with the following property: If $(X_t)_{t \in T}$ is a mean-zero separable random process indexed by a metric space $(T,\varrho)$ satisfying
\begin{equation*}
\Vert X_t - X_s \Vert_{\psi_2} \leq K \, \varrho(t,s)
\end{equation*}
for every $t,s \in T$ with a constant $K>0$ which does not depend on $s$ and $t$, the following hold:
\begin{enumerate}
\item{For all $t_0 \in T$ and $u \geq 0$ it holds
\begin{equation*}
\underset{t \in T}{\sup} \ (X_t - X_{t_0}) \leq CK \left(\int_0^\infty \sqrt{\ln \left(\mathcal{N}(T, \varrho, \eps)\right)} \ \dd\eps + u\diam(T)\right)
\end{equation*}
with probability at least $1- 2\exp(-u^2)$.}
\item{
$ \displaystyle
\EE \left[ \underset{t \in T}{\sup}\ X_t\right] \leq CK \int_0^\infty \sqrt{\ln \left(\mathcal{N}(T, \varrho, \eps)\right)} \ \dd\eps.
$}
\end{enumerate}
\end{theorem}

\begin{proof}
Firstly, note that the separability assumption implies that $\underset{t \in T}{\sup} \ X_t = \underset{t \in T_0}{\sup} \ X_t$ almost surely for some countable set $T_0 \subseteq T$ so that no measurability issues arise (at least if we are willing to pass to the completion of the underlying probability space). 

Let $C_1 > 0$ be the absolute constant appearing in \cite[Equation (2.16)]{vershynin_high-dimensional_2018} and define $Y_t \defeq \frac{X_t}{K \sqrt{2C_1}}$ for $t \in T$. Then it holds
\begin{equation*}
\Vert Y_t - Y_s \Vert_{\psi_2} = \frac{\Vert X_t - X_s \Vert_{\psi_2}}{K \sqrt{2C_1}} \leq \frac{\varrho(t,s)}{\sqrt{2C_1}}
\end{equation*}
for every $t,s \in T$ and hence, using \cite[Equation (2.16)]{vershynin_high-dimensional_2018}, we get
\begin{align*}
\EE \left[ \exp(\lambda(Y_t - Y_s))\right] \leq \exp(C_1 \lambda^2 \Vert Y_t - Y_s\Vert_{\psi_2}^2) \leq \exp(\lambda^2 \varrho(t,s)^2/2)
\end{align*}
for all $\lambda \in \RR$. Hence, $(Y_t)_{t \in T}$ is a \emph{sub-gaussian} process (see \cite[Definition 5.20]{van2014probability}) and trivially inherits the property of separability from $(X_t)_{t \in T}$. Hence, \cite[Theorems~5.25~and~5.29]{van2014probability} yield the existence of a constant $C_2>0$ such that the following hold:
\begin{enumerate}
\item{For all $u \geq 0$ and $t_0 \in T$ it holds 
\begin{equation*}
\underset{t \in T}{\sup} \ (Y_t - Y_{t_0}) \leq C_2 \left(\int_0^\infty \sqrt{\ln(\mathcal{N}(T, \varrho, \eps))} \ \dd \eps + u \diam(T) \right)
\end{equation*}
with probability at least $1- C_2\exp(-u^2)$.}
\item{$ \displaystyle \EE \left[\underset{t \in T}{\sup} \  Y_t \right] \leq C_2 \int_0^\infty \sqrt{\ln(\mathcal{N}(T, \varrho, \eps))} \ \dd \eps $}.
\end{enumerate}
Note that in (1) we can bound the probability even by $1-2\exp(-u^2)$ which is due to the first inequality on p.137 in \cite{van2014probability}. Specifically, we have
\begin{equation*}
\sum_{k=1}^\infty \ee ^{-k /2} \leq \sum_{k=1}^\infty\left( \frac{9}{4} \right)^{-k/2} = \sum_{k=1}^\infty \left(\frac{3}{2}\right)^{-k} = \frac{2/3}{1- 2/3} = 2.
\end{equation*}
Using this fact and the exact definition of $(Y_t)_{t \in T}$ yields the claim by letting $C \defeq \sqrt{2C_1}C_2$.
\end{proof}

Now we apply Dudley's inequality to certain finite-dimensional Gaussian processes.

\begin{proposition}[{special version of Dudley's inequality}]\label{prop:dudley}
There exists an absolute constant $C>0$ with the following property: For any $T \subseteq \RR^N$ with $0 \in T$ and any random vector $X \in \RR^N$ with 
\begin{equation*}
X_i \iid \mathcal{N}(0,1) \quad \text{for} \quad 1 \leq i \leq N,
\end{equation*}
 the following hold:
 \begin{enumerate}
 \item{ For any $u \geq 0$, we have
 \begin{equation*}
 \underset{t \in T}{\sup} \ \langle X, t \rangle \leq  C \left(\int_0^\infty \sqrt{\ln \left(\mathcal{N}(T, \Vert \cdot \Vert_2, \eps)\right)} \ \dd \eps + u \cdot \diam(T)\right)
 \end{equation*}
 with probability at least $1 - 2 \exp(-u^2)$.
 }
 \item{
 $ \displaystyle
 \EE \left[ \underset{t \in T}{\sup} \ \langle X, t \rangle\right] \leq C\int_0^\infty \sqrt{\ln \left(\mathcal{N}(T, \Vert \cdot \Vert_2, \eps)\right)} \ \dd \eps.
 $
 }
 \end{enumerate}
\end{proposition}
\begin{proof}
For $t,s \in T$ we have
\begin{equation*}
\Vert \langle X,t \rangle - \langle X,s \rangle \Vert_{\psi_2} = \Vert \langle X, t-s \rangle \Vert_{\psi_2} \leq C_1 \Vert t -s \Vert_2
\end{equation*}
with an absolute constant $C_1>0$. Here we used that$\langle X, t-s \rangle \sim \mathcal{N}(0, \Vert t -s \Vert_2^2)$ (see for instance \cite[Exercise 3.3.3]{vershynin_high-dimensional_2018}) and \cite[Example 2.5.8 (i)]{vershynin_high-dimensional_2018}. Furthermore, it holds
\begin{equation*}
\EE \left[ \langle X,t \rangle\right] = 0
\end{equation*}
for fixed $t \in T$ since $\langle X,t \rangle \sim \mathcal{N}(0, \Vert  t \Vert_2^2)$. For a fixed realization of $X$ it is trivial that $t \mapsto \langle X,t \rangle$ is continuous and hence, the process $(\langle X,t \rangle)_{t \in T}$ is separable by \Cref{prop:equal}. Dudley's inequality and its high-probability version (cf. \Cref{thm:dudleyyy}) then show that:
\begin{enumerate}
\item{ For any $u \geq 0$, we have
\begin{equation*}
\underset{t \in T}{\sup} \ \langle X, t \rangle = \underset{t \in T}{\sup}  \big(\langle X,t \rangle - \langle X, 0 \rangle \big)\overset{0 \in T}{\leq}  C_1C_2 \left(\int_0^\infty \sqrt{\ln \left(\mathcal{N}(T, \Vert \cdot \Vert_2, \eps)\right)} \ \dd \eps + u \cdot \diam(T)\right)
\end{equation*}
with probability at least $1 - 2\exp(-u^2)$. 
}
\item{
$ \displaystyle
\EE \left[ \underset{t \in T}{\sup} \ \langle X, t \rangle\right] \leq C_1C_2\int_0^\infty \sqrt{\ln \left(\mathcal{N}(T, \Vert \cdot \Vert_2, \eps)\right)} \ \dd \eps
$}
\end{enumerate}
with an absolute constant $C_2 > 0$. Hence, the claim follows letting $C \defeq C_1C_2$. 
\end{proof}


