\begin{proof}[Proof Sketch of Theorem~\ref{thm:traj_u}]
Here we give a continuous time proof of Theorem~\ref{thm:traj_u}.
The discrete time proof follows similar ideas but involves more (non-trivial) perturbation analysis so we deferred it to appendix. 

Note for continuous time we have the following differential equations\begin{align*}
\frac{d \vect{u}(t)}{dt} = &-\mat{H}(t)\left(\vect{u}(t)-\vect{y}\right), \\
\frac{d \tilde{\vect{u}}(t)}{dt} = &- \mat{H}^{\infty}\left(\tilde{\vect{u}}(t)-\vect{y}\right).
\end{align*}

First observe that because $\tilde{\vect{u}}(0) = \vect{u}(0)$, we have \[
\norm{\tilde{\vect{u}}(t)-\vect{u}(t)}_2 \le \int_{\tau = 0}^t \norm{\frac{d\left(\tilde{\vect{u}}(\tau)-\vect{u}(\tau)\right)}{d \tau}}_2 d\tau.
\]
Note \begin{align*}
&\frac{d \left(\tilde{\vect{u}}(t)-\vect{u}(t)\right)}{dt} \\
= &-\mat{H}(0)\left(\tilde{\vect{u}}(t)-\vect{y}\right) + \mat{H}(t)\left(\vect{u}(t)-\vect{y}\right) \\
= &  -\mat{H}(0)\left(\tilde{\vect{u}}(t)-\vect{u}(t)\right) + \left(\vect{u}(t)-\vect{y}\right)\left(\mat{H}(t)-\mat{H}(0)\right).
\end{align*}
Since $\mat{H}(0)$ is positive semidefinite, $-\mat{H}(0)\left(\tilde{\vect{u}}(t)-\vect{u}(t)\right) $ term only makes $\norm{\tilde{\vect{u}}(t)-\vect{u}(t)}_2$ smaller.
Therefore, we have \begin{align*}
&\norm{\tilde{\vect{u}}(t)-\vect{u}(t)}_2 \\
\le &\int_{\tau =0}^t \norm{\vect{u}(\tau)-\vect{y}}_2 \norm{\mat{H}(t)-\mat{H}(0)}_2 \\
\le & t\norm{\vect{u}(0)-\vect{y}}_2 \frac{\poly(n,1/\lambda_0)}{\sqrt{m}} = \frac{t\poly(n,1/\lambda_0)}{\sqrt{m}} .
\end{align*}

\end{proof}




\begin{proof}[Proof Sketch of Theorem~\ref{thm:traj_W}]
	To illustrate the main proof idea we give the continuous time analysis below.
	We first consider the continuous time gradient flow.
	The key idea is to consider the following auxiliary ODEs\begin{align*}
	\frac{d\widetilde{\mat{W}}(t)}{dt} = &- \mat{Z}(0) \left(\tilde{\vect{u}}(t)-\vect{y}\right) \text{ with } \widetilde{\mat{W}} = \mat{W}(0) \\
	\frac{d\tilde{\vect{u}}(t)}{dt} = &- \mat{H}(0)\left(\tilde{\vect{u}}(t)-\vect{y}\right) \text{ and }\tilde{\vect{u}}(0) = \vect{u}(0).
	\end{align*} 
	
	Note $\widetilde{\mat{W}}(t)$ is the desired ODE, so we want to bound $\mat{W}-\widetilde{\mat{W}}(t)$.
	\begin{lem}\label{lem:gf_optimization}
		With high probability over initialization, we have for all $t\ge 0$\begin{align*}
		\norm{\mat{W}(t)-\widetilde{\mat{W}}(t)}_F \le \frac{\poly(n,1/\lambda_0,\log m)}{\sqrt{m}}.
		\end{align*}
	\end{lem}
	
	
	
	Also note, both $\vect{u}(t)-\tilde{\vect{u}}(t)$ will converge to $\vect{y}$ by the analysis in \cite{du2018provably}.
	We can compute \begin{align*}
	&\frac{d \left(\mat{W}(t)-\widetilde{\mat{W}}(t)\right)}{dt}\\
	= &-\mat{Z}(t)(\vect{u}(t)-\vect{y}) + \mat{Z}(0)\left(\tilde{\vect{u}}(t)-\vect{y}\right) \\
	= &\left(\mat{Z}(0)-\mat{Z}(t)\right)\left(\vect{u}(t)-\vect{y}\right) + \mat{Z}(0)\left(\tilde{\vect{u}}(t)-\vect{u}(t)\right).
	\end{align*}
	For the first term, recall in \cite{du2018provably}, they showed for all $t >0$ \begin{align*}
	\norm{\mat{Z}(t) - \mat{Z}(0)}_2 = \frac{\poly(n,1/\lambda_0)}{\sqrt{m}}.
	\end{align*}
	Therefore we can bound \begin{align*}
	&\int \norm{\left(\mat{Z}(0)-\mat{Z}(t)\right)\left(\vect{u}(t)-\vect{y}\right) }_F dt\\
	\le &\frac{\poly(n,1/\lambda_0)}{\sqrt{m}} \int \norm{\tilde{\vect{u}}(t)-\vect{y}}_2 dt= \frac{\poly(n,1/\lambda_0)}{\sqrt{m}}.
	\end{align*}
	For the second term, we thus need to bound\begin{align}
	\int_{t=0}^{\infty}\norm{\tilde{\vect{u}}(t)-\vect{u}(t)}_2 dt. \label{eqn:u_tilde_u}
	\end{align}
	To bound Equation~\eqref{eqn:u_tilde_u}, we observe that $\tilde{\vect{u}}(t) \rightarrow \vect{y}$ and $\vect{u}(t) \rightarrow \vect{y}$.
	Therefore, we can choose some $t_0$ such that \begin{align*}
	&\int_{t_0}^\infty\norm{\vect{\tilde{u}}(t)-\vect{u}(t)}_2 dt\\
	\le &\int_{t_0}^\infty(\norm{\vect{\tilde{u}}(t)-\vect{y}}_2  + \norm{\vect{y}-\vect{u}(t)}_2 )dt
	\end{align*}
	Note if we choose $t_0 = \frac{C}{\lambda_0}\left(\frac{m\lambda_0}{n}\right)$ so that \begin{align*}
	\int_{t_0}^\infty\norm{\vect{\tilde{u}}(t)-\vect{u}(t)}_2 dt \le \frac{2\norm{\vect{u}(t)-\vect{y}}_2}{\lambda_0} \le \frac{\poly(n,1/\lambda_0)}{\sqrt{m}}.
	\end{align*}
	Thus it suffices to bound \begin{align*}
	\int_{t=0}^{t_0}\norm{\tilde{\vect{u}}(t)-\vect{u}(t)}_2 dt \le t_0 \max_{0\le t\le t_0} \norm{\tilde{\vect{u}}(t)-\vect{u}(t)}_2.
	\end{align*}
	%First observe that \[
	%\norm{\tilde{\vect{u}}(t)-\vect{u}(t)}_2 \le \int_{\tau = 0}^t \norm{\frac{d\left(\tilde{\vect{u}}(\tau)-\vect{u}(\tau)\right)}{d \tau}}_2 d\tau.
	%\]
	%Note \begin{align*}
	%&\frac{d \left(\tilde{\vect{u}}(t)-\vect{u}(t)\right)}{dt} \\
	%= &-\mat{H}(0)\left(\tilde{\vect{u}}(t)-\vect{y}\right) + \mat{H}(t)\left(\vect{u}(t)-\vect{y}\right) \\
	%= &  -\mat{H}(0)\left(\tilde{\vect{u}}(t)-\vect{u}(t)\right) + \left(\vect{u}(t)-\vect{y}\right)\left(\mat{H}(t)-\mat{H}(0)\right).
	%\end{align*}
	%Since $\mat{H}(0)$ is positive semidefinite, $-\mat{H}(0)\left(\tilde{\vect{u}}(t)-\vect{u}(t)\right) $ term only makes $\norm{\tilde{\vect{u}}(t)-\vect{u}(t)}_2$ smaller.
	%Therefore, we have \begin{align*}
	%&\norm{\tilde{\vect{u}}(t)-\vect{u}(t)}_2 \\
	%\le &\int_{\tau =0}^t \norm{\vect{u}(\tau)-\vect{y}}_2 \norm{\mat{H}(t)-\mat{H}(0)}_2 \\
	%\le & t\norm{\vect{u}(0)-\vect{y}}_2 \frac{\poly(n,1/\lambda_0)}{\sqrt{m}} = \frac{t\poly(n,1/\lambda_0)}{\sqrt{m}} .
	%\end{align*}
	Therefore, we have \begin{align*}
	&\int_{t=0}^{t_0}\norm{\tilde{\vect{u}}(t)-\vect{u}(t)}_2 dt \\
	\le &t_0 \max_{0\le t\le t_0} \norm{\tilde{\vect{u}}(t)-\vect{u}(t)}_2 \\
	\le &t_0^2 \frac{\poly(n,1/\lambda_0)}{\sqrt{m}} \\
	\le &\frac{\poly(n,1/\lambda_0,\log(m))}{\sqrt{m}}.
	\end{align*}
\end{proof}