In this section we will prove \cref{lem:UnifConverge}.
Applying Proposition 3.2 and Proposition 3.3 in \citet{lai2023_GeneralizationAbility},
it suffices to show \cref{prop:2_KernelUniform}, that is, the kernel $K_t$ converges uniformly to $\NTK$.
This rest of this section is organized as follows:
We first introduce some more preliminaries;
in \cref{subsec:Init}, we discuss some properties of the network at initialization;
in \cref{subsec:Training}, we analyze the effect of small perturbation during training process of the network;
we then prove the lazy regime approximation of the neural network in \cref{subsec:Lazy};
we also show the Hölder continuity of $\NTK$ in \cref{subsec:Holder_NTK};
finally, we prove the kernel uniform convergence in \cref{subsec:KernelUniformConvergence}.


\paragraph{Further notations}
Let us denote $\tilde{B}_R = \left\{ x \in \R^d : \tilde{x} \leq R \right\}$ for $R\geq 1$.
For a vector $\bm{v}=(v_1, v_2, \cdots, v_m)\in\mb{R}^m$, we use $\norm{\bm{v}}_2$ (or simply $\norm{\bm{v}}$) to represent the Euclidean norm.
Additionally, if we have a univariate function $f:\mb{R}\to\mb{R}$, we define $f(\bm{v}) = (f(v_1), f(v_2),\cdots, f(v_m))\in\mathbb{R}^m$.
We denote by $\norm{\bm{M}}_2$ and $\norm{\bm{M}}_{\mr{F}}$ the spectral and Frobenius norm of a matrix $\bm{M}$ respectively.
Also, we use $\norm{\,\cdot\,}_0$ to represent the number of non-zero elements of a vector or matrix.
For matrices $\bm{A}\in\mb{R}^{n_1\times n_2}$ and $\bm{B}\in\mb{R}^{n_2\times n_1}$,
we define $\ang{\bm{A},\bm{B}} = \Tr(\bm{A}\bm{B}^T)$.
We remind that $\ang{\bm{M}, \bm{M}} = \norm{\bm{M}}_{\mr{F}}^2$ in this way.

\paragraph{Network Architecture} Let us recall the neural network in the main text.
% We fix the number of layers $L \geq 1$.
%(1) initialized by (2) and trained by gradient descent 
% \cref{eq:NN_Arch}, \cref{eq:NN_Init}
Since it can be shown easily that the bias term $\bm{b}^{(0,p)}$ in the first layer can be absorbed into $\bm{A}^{(p)}$
if we append an $1$ at the last coordinate of $\x$,
% for a little abuse of notation,
we denote $\bm{W}^{(0,p)}=(\bm{A}^{(p)}~\bm{b}^{(0,p)})$, $\tilde{\x}=(\x^T,1)^T\in\mb{R}^d \times \{1\}\subset\mb{R}^{d+1}$
and consider the following equivalent neural network:
% \begin{align} 
%   \label{eq:A_NN_Arch}
%   \begin{split}
%     \bm{\alpha}^{(0,p)}(\x)&=\tilde{\bm{\alpha}}^{(0,p)}(\x)=\tilde{\x}\in\mb{R}^{d+1},~p\in\dk{1,2},\\
%     \bm{\alpha}^{(l,p)}(\x) & = \sqrt{\tfrac{2}{m_{l}}} \bm{W}^{(l-1,p)}\tilde{\bm{\alpha}}^{(l-1,p)}(\x)\in\mb{R}^{m_l},\quad~\tilde{\bm{\alpha}}^{(l,p)}(\x)=\sigma\mpt{\bm{\alpha}^{(l,p)}(\x)}\in\mb{R}^{m_l},\qquad~ l \in [L],~p\in\dk{1,2},\\
%     g^{(p)}(\x;\bm{\theta}) & = \bm{W}^{(L,p)}\tilde{\bm{\alpha}}^{(L,p)}(\x)+b^{(L,p)}\in\mb{R},\quad~p\in\dk{1,2},\qquad~
%     f(\x;\bm{\theta}) = \frac{\sqrt{2}}{2}\bk{ g^{(1)}(\x;\bm{\theta}) - g^{(2)}(\x;\bm{\theta}) }\in\mb{R},
%   \end{split}
% \end{align}
\begin{align}
  \label{eq:A_NN_Arch}
  \begin{split}
    \bm{\alpha}^{(0,p)}(\x)&=\widetilde{\bm{\alpha}}^{(0,p)}(\x)=\tilde{\x}\in\mb{R}^{d+1},\\
    \widetilde{\bm{\alpha}}^{(l,p)}(\x) & = \sqrt{\tfrac{2}{m_{l}}} \bm{W}^{(l-1,p)}{\bm{\alpha}}^{(l-1,p)}(\x)\in\mb{R}^{m_l},\\
    {\bm{\alpha}}^{(l,p)}(\x)&=\sigma\xkm{\widetilde{\bm{\alpha}}^{(l,p)}(\x)}\in\mb{R}^{m_l},\\
    g^{(p)}(\x;\bm{\theta}) & = \bm{W}^{(L,p)}{\bm{\alpha}}^{(L,p)}(\x)+b^{(L,p)}\in\mb{R},\\
    f(\x;\bm{\theta}) &= \frac{\sqrt{2}}{2}\zk{ g^{(1)}(\x;\bm{\theta}) - g^{(2)}(\x;\bm{\theta}) }\in\mb{R}.
  \end{split}
\end{align}
for $p=1,2$ and $l=1,\cdots,L$.
% where $l \in \cl{ 0,1,\cdots,L }$ and $p \in \{1,2\}$ stand for the index of layers and parts respectively, $\sigma(x)= \max(x,0)$ is the ReLU activation,
Recall that the integers $m_1,m_2,\cdots, m_L$ are the width of $L$-hidden layers and $m_{L+1} = 1$ is the width of output layer.
Additionally, we have set $m = \min(m_1,m_2,\cdots,m_L)$ and made the assumption that $\max(m_1,m_2,\dots,m_{L}) \leq C_{\mathrm{m}} m$ for some absolute constant $C_{\mathrm{m}}$.
By setting $m_0=d+1$ for convenience, we have $\bm{W}^{(l,p)} \in \R^{m_{l+1} \times m_l}$ for $l\in\{0,1,2,\cdots,L\}$.

% We use $\bm{\theta}$ to represent the collection of all parameters flatten as a column vector,
% and use $\bm{\theta}^{(p)}$ for the parameters in part $p$.
% We also recall that $m = \min(m_1,\cdots,m_L)$, $M = \max(m_1,\cdots,m_L)$
% and we assume that $m\leq M\leq C_m m$ for some absolute constant $C_m\geq 1$.



% \subsubsection{Initialization and training}

%\paragraph{Initialization and training}
%At initialization, the parameters in one parity are i.i.d. normal and the parameters in the other parity are the same as the corresponding ones. For $l=0,1,\cdots,L$, we have
%%The initialization for $W$ is always given by independent
%\begin{align*}
%  \bm{W}^{(l,1)}_{i,j},b^{(L,1)} \iid N(0,1),\qquad~\bm{W}^{(l,2)} = \bm{W}^{(l,1)},\qquad~ b^{(L,2)} = b^{(L,1)}.
%\end{align*}
%In the training process, the parameter at time $t \geq 0$, denoted by $\bm{\theta}_t$, is determined according to the gradient flow equation \cref{eq:2_GD}.
% \begin{align}
%   \label{eq:a_GD}
%   \dot{\bm{\theta}}_t = - \nabla_{\bm{\theta}} \mathcal{L}(\bm{\theta}_t)= - \frac{1}{n} \nabla_{\bm{\theta}} f(\X;\bm{\theta}_t) (f(\X;\bm{\theta}_t) - \bm{y})
% \end{align}
% where $f(\X;\bm{\theta}_t) = \pt{f(\x_1;\bm{\theta}_t),\dots,f(\x_n;\bm{\theta}_t)}^T$ and $\nabla_{\bm{\theta}} f(\X;\bm{\theta}_t)$ is an $N_{\bm{\theta}} \times n$
% matrix where $N_{\bm{\theta}}$ is the number of parameters.
% %Quantities adding a subscript $t$ refer to the quantities
% %We also add a subscript $t$ for all the related quantities to refer to them in the training process.

For $p=1,2$, we define $g^{(p)}_t(\x) = g^{(p)}(\x;\bm{\theta}_t)$ and $f_t(\x) = f(\x;\bm{\theta}_t)$.
Similarly, we also add a subscript $t$ for all the related quantities (including those defined afterwards) to indicate their values at time $t$ during the training process.
% we add the subscript "$t$" to the other quantities in \cref{eq:A_NN_Arch} to indicate their values at time $t$.


% To track the training process, we will use $T$ to represent an interval $[0,t_0]$ or $[0,+\infty)$.

%\paragraph{Folding the network}
%Let us expand \cref{eq:a_GD} by $f(\x;\bm{\theta}) = \frac{\sqrt{2}}{2}\big[ g^{(1)}(\x;\bm{\theta}) - g^{(2)}(\x;\bm{\theta}) \big]$ and get
%\begin{align*}
%  \dot{\bm{\theta}}_t
%  &= - \frac{1}{n} \left[ \nabla_{\bm{\theta}} f(\X;\bm{\theta}_t) \right]^T
%\end{align*}
%We will show that for the the parameters evolve training process of our mirrored neural network is exactly the same as training only a parity of the network, except

% \subsubsection{The neural tangent kernel}

\paragraph{The neural tangent kernel}
%In \cref{subsec:UniformConv},
%we have introduced the concepts of NNK (Neural Network Kernel), denoted by $K_t$, and NTK (Neural Tangent Kernel), denoted by $\NTK$.
%
Let us consider the following neural network kernel
\begin{align*}
 % \label{eq:A_Kt_Formula}
 K_t(\x,\x') = \ang{\nabla_{\bm{\theta}} f(\x;\bm{\theta}_t),\nabla_{\bm{\theta}} f(\x';\bm{\theta}_t)}.
\end{align*}
Then, the gradient flow can be written as~\citep{jacot2018_NeuralTangent}
\begin{align*}
 % \label{eq:A_NN_GF}
 \dot{f}(\x;\bm{\theta}_t) = - \frac{1}{n} K_t(\x,\X)\xk{f(\X;\bm{\theta}_t) - \bm{y}}.
\end{align*}
As shown in \citet{jacot2018_NeuralTangent}, as the width $m$ goes to infinity, the kernel $K_t$ converges to the deterministic neural tangent kernel $\NTK$.
%
% The limit of $K_0$ as $m \to \infty$ yields the neural tangent kernel (NTK),
% \begin{align*}
%   % \label{eq:A_NTK_Def}
%   K_{0}(\x,\x') \overset{p}{\rightarrow} \NTK(\x,\x')  \qq{as} {m \to \infty}.
% \end{align*}
%
With the mirrored architecture given by \cref{eq:A_NN_Arch}, we can express the kernel $K_t(\x,\x')$ as follows:
\begin{align*}
  K_t(\x,\x')
  &=  \ang{\nabla_{\bm{\theta}} f(\x;\bm{\theta}_t), \nabla_{\bm{\theta}} f(\x;\bm{\theta}_t)} =\sum_{p=1}^2 \ang{\nabla_{\bm{\theta}^{(p)}} f(\x;\bm{\theta}_t), \nabla_{\bm{\theta}^{(p)}}f(\x;\bm{\theta}_t)} \\
  % &=  \frac{1}{2}\sum_{p=1}^2 \ang{\nabla_{\bm{\theta}^{(p)}} \mpt{g^{(1)}_t(\x) - g^{(2)}_t(\x)}, \nabla_{\bm{\theta}^{(p)}} \mpt{g^{(1)}_t(\x) - g^{(2)}_t(\x)}} \\
  &= \frac{1}{2} \sum_{p=1}^2 \ang{\nabla_{\bm{\theta}^{(p)}} g^{(p)}_t(\x),\nabla_{\bm{\theta}^{(p)}} g^{(p)}_t(\x')}=  \frac{1}{2} \xk{K_t^{(1)}(\x,\x') + K_t^{(2)}(\x,\x') },
  % &K_t(\x,\x')
  % =\sum_{p=1}^2 \ang{\nabla_{\bm{\theta}^{(p)}} f(\x;\bm{\theta}_t), \nabla_{\bm{\theta}^{(p)}}f(\x;\bm{\theta}_t)} \\
  % &~~= \frac{1}{2} \sum_{p=1}^2 \ang{\nabla_{\bm{\theta}^{(p)}} g^{(p)}_t(\x),\nabla_{\bm{\theta}^{(p)}} g^{(p)}_t(\x')}=\frac{1}{2}\sum_{p=1}^2 K_t^{(p)}(\x,\x'),
\end{align*}
where $K_t^{(p)}(\x,\x')=\ang{\nabla_{\bm{\theta}^{(p)}} g^{(p)}_t(\x),\nabla_{\bm{\theta}^{(p)}} g^{(p)}_t(\x')}$ is the neural network kernel of $g_t^{(p)}$ for $p=1,2$, which is a vanilla neural network.
Consequently, due to the mirror initialization, we have
% \begin{align*}
% \label{eq:TangentKernel_0}
$K_0^{(1)}(\x,\x') = K_0^{(2)}(\x,\x')= K_0(\x,\x')$.
% \end{align*}

% As $m\to\infty$, the neural network kernel $K_t(\x,\x')$ converges in probability to a kernel function known as the Neural Tangent Kernel (NTK), denoted by $\NTK(\x,\x')$:
% \begin{align*}
%   % \label{eq:A_NTK_Def}
%   K_{t}(\x,\x') \overset{\mb{P}}{\rightarrow} \NTK(\x,\x')  \qq{as} {m \to \infty}.
% \end{align*}
% This convergence result is the focus of \cref{thm:A_KernelUniform}, which we aim to prove in this section.

% Following \citet{jacot2018_NeuralTangent}, we can provide an explicit recursive formula for the limit of $K_0^{(p)}$ as well as $K_0$.
% % Here we note that since now $\x \in \tilde{B}_R \times\{1\}$, our formula is different from the formula in the main text in $\Sigma_0(\x,\x')$.
% Let us define
% \begin{align*}
%   \Sigma_0(\x,\x') &= N_0(\x,\x') = \ang{\x,\x'}+1;\\
%   \Sigma_{l}(\x,\x') &= 2 \E_{(u_l,v_l)}[ \sigma(u_l) \sigma(v_l)], \\
%   N_{l}(\x,\x') &= \Sigma_{l}(\x,\x') + 2 N_{l-1}(\x,\x') \E_{(u_l,v_l)}[ \dot{\sigma}(u_l) \dot{\sigma}(v_l)],
% \end{align*}
% for $l=1,2,\cdots,L$, where $(u_l,v_l)$ follows a multivariate normal distribution with mean 0 and covariance matrix $\bm{B}_{l-1}(\x,\x')$, i.e., $(u_l,v_l)\sim N(\bm{0},\bm{B}_{l-1}(\x,\x'))$, and the matrix $\bm{B}_l(\x,\x')$ is defined as:
% \begin{align*}
%   \bm{B}_l(\x,\x') =
%   \begin{pmatrix}
%     \Sigma_l(\x,\x)  & \Sigma_l(\x,\x')  \\
%     \Sigma_l(\x,\x') & \Sigma_l(\x',\x')
%   \end{pmatrix}.
% \end{align*}
% Additionally, we define $\dot{\sigma}(x) = \bm{1}_{\cl{x > 0}}$ as the weak derivative of $\sigma(x)$. With these definitions, it can be shown that the limit of $K_0$ is $N_L(\x,\x') + 1$. Actually, we can prove that it is also the limit of $K^{(p)}_t$ and $K_t$, which means that the NTK associated with $K_t^{(p)}$ and $K_t$ is
% \begin{align}
%   \label{eq:A_NTK_Def0}
%   &\NTK(\x,\x') = N_L(\x,\x') + 1.
% \end{align}
% Before completing the proof of \cref{thm:A_KernelUniform}, we can treat \cref{eq:A_NTK_Def0} as the definition of $\NTK$.

%Further computation (see \cref{subsec:A_ExplicitNTK}) shows that
%\begin{align}
%  \label{eq:NTK_Formula_new}
%  \NTK(\x,\x') =
%  \norm{\tilde{\x}}\norm{\tilde{\x}'} \sum_{r=0}^L \kappa^{(r)}_1(\bar{u}) \prod_{s=r}^{L-1} \kappa_0\mpt{\kappa^{(s)}_1(\bar{u})} + 1.
%\end{align}
%where $\tilde{\x} = (\x^T,1)^T$, $\tilde{\x}' = (\x'^T,1)^T$, $\bar{u} = \ang{\tilde{\x},\tilde{\x}'} / \pt{\norm{\tilde{\x}}\norm{\tilde{\x}'}}$,
%\begin{align}
%  \label{eq:A_Arccos_Formula}
%  \kappa_0(u) =  \frac{1}{\pi}\pt{\pi - \arccos u},\qquad
%  \kappa_1(u) = \frac{1}{\pi}\bk{\sqrt {1-u^2} + u (\pi - \arccos u)},
%\end{align}
%$\kappa_1^{(p)} = \kappa_1\underbrace{\circ \dots \circ}_{p \text{ times}} \kappa_1$ represents $p$ times composition of $\kappa_1$
%and $\kappa_1^{(0)}(u) = u$ by convention;
%if $r = L$, the product $\prod_{s=r}^{L-1}$ is understood to be $1$.
%%It is easy to check that \cref{eq:NTK_Formula_new} is different but equivalent to \cref{eq:NTK_Formula}.

% \subsubsection{An expanded matrix form}

\paragraph{An expanded matrix form}
Sometimes it is convenient to write the neural network \cref{eq:A_NN_Arch} in an expanded matrix form as introduced in \citet{allen-zhu2019_ConvergenceTheory}.
Let us define the activation matrix
\begin{align*}
  \bm{D}^{(l,p)}_{\x}=
  \begin{cases}
    \bm{I}_{d+1}, &l=0;\\
    \mr{diag}\xkm{\dot\sigma\xkm{\widetilde{\bm{\alpha}}^{(l,p)}\xkm{\x}}}, &l \geq 1,
  \end{cases}\quad\in\mb{R}^{m_l\times m_l}
\end{align*}
% \begin{align*}
%   \bm{D}^{(l,p)}_{\x}=\mr{diag}\mpt{\bm{1}\cl{\pt{\bm{\alpha}^{(l,p)}_{\x}}_1>0},\cdots,\bm{1}\cl{\pt{\bm{\alpha}^{(l,p)}_{\x}}_{m_l}>0}}
% \end{align*}
for $p=1,2$, where $\bm{I}_{d+1}$ is the identity matrix.
Then, we have
\begin{align*}
  \widetilde{\bm{\alpha}}^{(l,p)}(\x) &= \sqrt{\frac{2}{m_l}} \bm{W}^{(l-1,p)}\bm{D}^{(l-1,p)}_{\x}\widetilde{\bm{\alpha}}^{(l-1,p)}(\x),\quad
  {\bm{\alpha}}^{(l,p)}(\x)
  % =\bm{D}^{(l,p)}_{\x}\bm{\alpha}^{(l,p)}(\x)
  = \sqrt{\frac2{m_l}}\bm{D}^{(l,p)}_{\x}\bm{W}^{(l-1,p)}{\bm{\alpha}}^{(l-1,p)}(\x).
\end{align*}
% We will also add a subscript $t$ to indicate those quantities during the training.

To further write it as product of matrices, we first introduce the following notation to avoid confusion since matrix product is not commutative.
For matrices $\bm{A}_0,\bm{A}_1,\cdots,\bm{A}_L$, we define the left multiplication product
\begin{align*}
  \prod_{i=a}^b \bm{A}_i =
  \begin{cases}
    1,&0\leq b <a \leq L;\\
    \bm{A}_b \bm{A}_{b-1} \cdots \bm{A}_{a+1} \bm{A}_a,& 0 \leq a \leq b \leq L.
  \end{cases}
\end{align*}
% and set $\prod_{i=a}^b \bm{A}_i = 1$ if $a > b$. 
Since real number multiplication is commutative, the notation introduced above is compatible with the traditional usage when $\bm{A}_0,\bm{A}_1, \cdots, \bm{A}_L$ degenerate into real numbers.
In this way, we have
\begin{align}
  \label{eq:NN_Layer_MatrixProductForm}
  \widetilde{\bm{\alpha}}^{(l,p)}(\x) = \xk{\prod_{r=1}^l \sqrt{\frac{2}{m_r}} \bm{W}^{(r-1,p)}\bm{D}^{(r-1,p)}_{\x}} \tilde{\x}, \quad
  {\bm{\alpha}}^{(l,p)}(\x) = \xk{\prod_{r=1}^{l} \sqrt{\frac{2}{m_r}} \bm{D}^{(r,p)}_{\x} \bm{W}^{(r-1,p)}}\tilde{\x}.
\end{align}
Using the above expressions, we can obtain
\begin{align*}
  g^{(p)}(\x) &= \bm{W}^{(L,p)}{\bm{\alpha}}^{(L,p)}(\x) + b^{(L,p)}=\bm{W}^{(L,p)}\xk{\prod_{r=l+1}^{L} \sqrt{\frac{2}{m_r}} \bm{D}^{(r,p)}_{\x} \bm{W}^{(r-1,p)}}{\bm{\alpha}}^{(l,p)}(\x) + b^{(L,p)}\\
  &=\xk{ \prod_{r=l+1}^{L} \sqrt{\frac{2}{m_{r}}} \bm{W}^{(r,p)}\bm{D}^{(r,p)}_{\x} }\bm{W}^{(l,p)} {\bm{\alpha}}^{(l,p)}(\x) + b^{(L,p)}.
  % &=\bm{W}^{(L,p)} \pt{\prod_{r=1}^{L} \sqrt{\frac{2}{m_r}} \bm{D}^{(r,p)}_{\x} \bm{W}^{(r-1,p)} } \x+b^{(L,p)}
  % \nonumber\\
  % &=\pt{ \prod_{r=1}^{L} \sqrt{\frac{2}{m_{r}}} \bm{W}^{(r,p)}\bm{D}^{(r,p)}_{\x} }\bm{W}^{(0,p)} \x + b^{(L,p)}.
  % \label{eq:NN_Output_MatrixProductForm}
\end{align*}
Finally, we use the above results to calculate the gradient $\nabla_{\bm{W}^{(l,p)}} g^{(p)}(\x)$.
To simplify notation, we define:
\begin{align}
  \label{eq:def_gamma}
  \begin{split}
    \begin{gathered}
      \widetilde{\bm{\alpha}}^{(l,p)}_{\x}=\widetilde{\bm{\alpha}}^{(l,p)}\xkm{\x},
      \quad{\bm{\alpha}}^{(l,p)}_{\x}={\bm{\alpha}}^{(l,p)}\xkm{\x},
      \quad\bm{\gamma}^{(l,p)}_{\x}=\xk{\prod_{r=l+1}^{L} \sqrt{\frac2{m_r}}  \bm{W}^{(r,p)} \bm{D}^{(r,p)}_{\x}}^T\in\mb{R}^{m_{l+1}}.
    \end{gathered}
  \end{split}
\end{align}
Then we can obtain
\begin{align*}
  g^{(p)}(\x)=\bm{\gamma}^{(l,p),T}_{\x}\bm{W}^{(l,p)}{\bm{\alpha}}^{(l,p)}_{\x}+b^{(L,p)},
\end{align*}
% holds for $l\in\cl{0,1,\cdots,L}$ and $p\in\dk{1,2}$,
which can lead to
\begin{align}
  \label{eq:GradientExpr}
  \nabla_{\bm{W}^{(l,p)}} g^{(p)}(\x)=\bm{\gamma}^{(l,p)}_{\x}{\bm{\alpha}}^{(l,p),T}_{\x},\quad l=0,1,\cdots,L,~p=1,2.
\end{align}
Also, it is worth noting that for two vectors $\bm{a}$ and $\bm{b}$, we have
\begin{align*}
  \norm{\bm{a} \bm{b}^T}_{\mathrm{F}}^2 = \Tr(\bm{a} \bm{b}^T \bm{b} \bm{a}^T) = \Tr(\bm{a}^T \bm{a} \bm{b}^T \bm{b}) = \norm{\bm{a}}_2^2 \norm{\bm{b}}_2^2.
\end{align*}
Consequently, we can get
\begin{align}
  \label{eq:GradientExpr_Fnorm}
  \norm{\nabla_{\bm{W}^{(l,p)}} g^{(p)}(\x)}_{\mr{F}}=\norm{\bm{\gamma}^{(l,p)}_{\x}}_2\norm{{\bm{\alpha}}^{(l,p)}_{\x}}_2.
\end{align}

% \subsubsection{Further notations}
% \paragraph{Further notations}
% To track the training process, we will use $T$ to represent an interval $[0,t_0]$ or $[0,+\infty)$.
% For a vector $\bm{v}$, we use $\norm{\bm{v}}_2$ (or simply $\norm{\bm{v}}$) to represent the Euclidean norm
% and use $\norm{\bm{v}}_\infty$ to represent the maximum absolute value of its elements.
% We denote by $\norm{\bm{M}}_2$, $\norm{\bm{M}}_\infty$ and $\norm{\bm{M}}_{\mr{F}}$ the spectral, infinity and Frobenius norm of a matrix $\bm{M}$ respectively.
% Also, we use $\norm{\,\cdot\,}_0$ to represent the number of non-zero elements of a vector or matrix.
% For matrices $\bm{A}\in\mb{R}^{n_1\times n_2}$ and $\bm{B}\in\mb{R}^{n_2\times n_1}$,
% we define $\ang{\bm{A},\bm{B}} = \Tr(\bm{A}\bm{B}^T)$.
% We remind that $\ang{\bm{M}, \bm{M}} = \norm{\bm{M}}_{\mr{F}}^2$ in this way.

% \subsubsection{Organization}



\subsection{Initialization}\label{subsec:Init}
Since our neural network is mirrored, we can focus only on one part $g^{(p)}(\x)$ of the network at initialization.
For notational simplicity, we omit the superscript $p$ for $\bm{W}^{(l,p)}_t$ and other notations in the following if there is no ambiguity.
And unless otherwise stated, it is understood that the conclusions hold for both $p = 1$ and $p = 2$.
%We will also suppress the subscript $t$ when $t = 0$.

Since $K^{(p)}_0$ corresponds to the tangent kernel of a vanilla fully connected neural network,
\citet[Theorem 3.1]{arora2019_ExactComputation} shows the following convergence result.


%We set this constriction since we know $\norm{\tilde{\x}}\geq 1$ (recalling that $\tilde{\x} \in \mathbb{R}^d \times \{1\}$).

\begin{lemma}
[Convergence to the NTK at initialization]
  There exist some positive absolute constants $C_1>0$ and $C_2\geq 1$ such that if $\varepsilon\in\xk{0,1}$, $\delta\in\xk{0,1}$ and $m\geq C_1\varepsilon^{-4}\ln\xkm{C_2/\delta}$, then for any fixed $\bm{z}, \bm{z}'$ such that $\norm{\bm{z}} \leq 1$ and $\norm{\bm{z}'}\leq 1$, with probability at least $1-\delta$ with respect to the initialization, we have
  \[\abs{K^{(p)}_0\xkm{\bm{z},\bm{z}'}-\NTK\xkm{\bm{z},\bm{z}'}}\leq\varepsilon.\]

  % For any
  % $\varepsilon > 0$ and any $\delta\in(0, 1)$, if $m = \Omega(\varepsilon^{-4}\ln(1/\delta))$, we then have that
  % for any inputs $\x, \x'\in\mc{\X}\times\{1\}$, with probability at least $1-\delta$ we have:
  %  \begin{align*}
  %    \abs{K^{(p)}_0(\x,\x')-\NTK(\x,\x')}\leq \varepsilon.
  %  \end{align*}
\end{lemma}

Letting $\varepsilon=m^{-1/5}$ in the previous lemma, we can get the following corollary:
\begin{corollary}
  \label{cor:Init_Kt_NTK}

  There exist some positive absolute constants $C_1>0$ and $C_2\geq 1$ such that if $\delta\in\xk{0,1}$ and $m\geq C_1\xk{\ln\xkm{C_2/\delta}}^5$, 
  then for any fixed $\bm{z},\bm{z}'\in \tilde{B}_R$, with probability at least $1-\delta$ with respect to the initialization, we have
  \[\abs{K^{(p)}_0\xkm{\bm{z},\bm{z}'}-\NTK\xkm{\bm{z},\bm{z}'}}=O\xkm{ R^2 m^{-1/5}}.\]

%   For any $\delta\in(0,1)$,
%   if $m \geq \mr{poly}(\ln(1/\delta))$,
% %  There exists a polynomial $\mr{poly}_1$ such that for any  $m\geq \mr{poly}_1\big(\ln(1/\delta)\big)$.
%   then for any inputs $\x$, $\x'\in\mc{\X}\times\{1\}$, with probability at least $1-\delta$, we have
%   \begin{align}
%     \abs{K_0(\x,\x')-\NTK(\x,\x')} = O(m^{-1/5}).
%   \end{align}
\end{corollary}

Now we further provide some bounds about the magnitudes of weight matrices and layer outputs.
The following is a standard estimation of Gaussian random matrix, which is a direct consequence of \citet[Corollary 5.35]{vershynin2010_IntroductionNonasymptotic}.
\begin{lemma}
  \label{lem:Random matrix}At initialization, there exists a positive absolute constant $C$, 
  such that when $m\geq C$, with probability at least $1-\exp\xkm{-\Omega(m)}$ with respect to the initialization,
  % With probability at least $1-\exp(-\Omega(m))$ with respect to the initialization,
  % for $l=0,1,\cdots,L$, 
  we have
  \begin{align*}
    \norm{\bm{W}^{(l)}_0}_2 = O\xkm{\sqrt{m}},\quad l\in \dk{0,1,\dots,L}.
  \end{align*}
\end{lemma}
% \begin{proof}This lemma is a direct consequence of \citet[Corollary 5.35]{vershynin2010_IntroductionNonasymptotic}.\end{proof}

%\begin{lemma}
%  Let $A$ be an $a\times b$ matrix whose entries are independent standard normal random variables.
%  Then for every $t\geq 0$, with probability at least $1-2\exp(- t^2/2)$ one has
%  \[\sqrt{a}-\sqrt{b}-t\leq s_{\min}(A)\leq s_{\max}(A)\leq \sqrt{a}+\sqrt{b}+t.\]
%\end{lemma}

Noticing that $\norm{\bm{D}^{(l)}_{\x}}_2 \leq 1$ and combining \cref{lem:Random matrix} with \cref{eq:NN_Layer_MatrixProductForm}, \cref{eq:def_gamma} and \cref{eq:GradientExpr_Fnorm}, we have:

\begin{lemma}
  \label{lem:Init_WeightsProductBound}There exists a positive absolute constant $C$, 
  such that when $m\geq C$, with probability at least $1-\exp(-\Omega(m))$ with respect to the initialization, for any
  % $1 \leq a \leq b \leq L $, 
  $l\in\dk{0,1,\cdots,L}$ and $\x \in \tilde{B}_R$, we have
  \begin{align*}
    \begin{gathered}
      \norm{\widetilde{\bm{\alpha}}_{\x,0}^{(l)}}_2 = O(R),\quad~\norm{{\bm{\alpha}}_{\x,0}^{(l)}}_2 = O(R),\quad~\norm{\bm{\gamma}^{(l)}_{\x,0}}_2=O(1)
      \quad~\text{and}\quad~
      \norm{\nabla_{\bm{W}^{(l)}} g_0(\x) }_{\mr{F}} = O(R).
    \end{gathered}
    % \norm{\bm{\alpha}_{\x,0}^{(l)}}_2 = O(1),
  \end{align*}
  % \begin{align*}
  % \begin{gathered}
  %   \norm{\prod_{r=a}^b \sqrt{\frac2{m_r}}  \bm{W}_0^{(r)} \bm{D}^{(r)}_{\x,0}}_2 = O(1),\quad \norm{\bm{\gamma}^{(l)}_{\x,0}}_2=O(1),\\
  %   \norm{\tilde{\bm{\alpha}}_{\x,0}^{(l)}}_2 = O(1) \qand \norm{\bm{\alpha}_{\x,0}^{(l)}}_2 = O(1),
  %   \end{gathered}
  % \end{align*}
\end{lemma}

\cref{lem:Random matrix} and \cref{lem:Init_WeightsProductBound} provide some upper bounds, and the subsequent lemma provides a lower bound.
It is important to note that the previous lemma holds uniformly for $\x\in\tilde{B}_R$, while the following lemma only holds pointwisely.

% Combining \cref{lem:Chi2_Binomial} with \cref{eq:NN_Layer_MatrixProductForm},
% we can get the following lemma. 
% \citep[Lemma 7.1]{allen-zhu2019_ConvergenceTheory}.
\begin{lemma}[Lemma 7.1 in \citet{allen-zhu2019_ConvergenceTheory}]
  \label{lem:Init_LayerOutputBounds}
  There exists a positive absolute constant $C$ such that when $m\geq C$, for any fixed $\bm{z}\in\tilde{B}_R$, 
  with probability at least $1-\exp\xkm{-\Omega(m)}$ with respect to the initialization, we have $\norm{{\bm{\alpha}}^{(l)}_{\bm{z},0}}_{2} =\Theta(R)$ for $l\in\dk{0,1,\cdots,L}$.
  % \begin{align*}
  %   \norm{{\bm{\alpha}}^{(l)}_{\bm{z},0}}_{2} =\Theta(1)\qquad~\text{for}~l\in\cl{0,1,\cdots,L}.
  % \end{align*}
  % when $m$ is greater than the positive constant $C$.

%   \begin{proof}
%     We prove the lemma by induction. For $l = 0$, we can get
%     \[\norm{ \tilde{\bm{\alpha}}^{(0)}_{\bm{z},0}}_2 =\norm{\bm{\alpha}^{(0)}_{\bm{z},0}}_2 = \norm{(\z^T,1)}_2 \in \bk{1,\sqrt{R^2+1}},\] and thus the base case holds.

%     Assuming the statement holds for $l=k\in\cl{0,1,\cdots,L-1}$. When $l=k+1$, we fix $\tilde{\bm{\alpha}}^{(k)}_{\bm{z},0}$ and let $\bm{W}^{(k)}_0$ be the only source of randomness. According to \cref{eq:A_NN_Arch}, we have $\bm{\alpha}^{(k+1)}_{\bm{z},0} = \sqrt{{2}/{m_{k+1}}} \,\bm{W}^{(k)}_0 \tilde{\bm{\alpha}}^{(k)}_{\bm{z},0}$, which implies that
%     \begin{align*}
%       \sqrt{{m_{k+1}}/2}\,{\bm{\alpha}^{(k+1)}_{\bm{z},0}}/{\norm{\bm{\tilde{\alpha}}^{(k)}_{\bm{z},0}}_2}=\bm{W}^{(k)}_0 {\tilde{\bm{\alpha}}^{(k)}_{\bm{z},0}}/{\norm{\bm{\tilde{\alpha}}^{(k)}_{\bm{z},0} }_2}
%     \end{align*}
%     is an $m_{k+1}$-dimensional standard normal random vector with respect to the randomness of $\bm{W}^{(k)}_0$.
% % then we have 
% % \[\frac{m_{k+1}\norm{\bm{\alpha}^{(k+1)}_0(\bm{z}) }_2^2}{2\norm{\tilde{\bm{\alpha}}^{(k)}_0(\bm{z}) }_2^2}= \sum_{j=1}^{m_{k+1}}\frac{ {m_{k+1}}\pt{\alpha^{(k+1)}_{0,j}(\bm{z})}^2}{2\norm{\tilde{\bm{\alpha}}^{(k)}_0(\bm{z}) }_2^2} \sim\chi^2_{m_{k+1}}.\]
% % Based on the inductive assumption, we have $\norm{\tilde{\bm{\alpha}}^{(k)}_0(\bm{z})}_2 = \Theta(1)$ with probability at least $1- \exp\mpt{-\Omega(m)} $.
% %  By applying tail probability estimation of the chi-square distribution, we can conclude that 
% %  $\norm{\bm{\alpha}^{(k+1)}_0(\bm{z})}_2 = \Theta(1)$ holds with probability at least $1-\exp\mpt{-\Omega(m)}$.
%     Recall the symmetry of the Gaussian distribution, we can see that \[\sigma\mpt{{\alpha}^{(k+1)}_{\bm{z},0,j}} = w_j \cdot  \abs{{\alpha}^{(k+1)}_{\bm{z},0,j}},\] where ${\alpha}^{(k+1)}_{\bm{z},0,j}$ denotes the $j$-entry of $\bm{\alpha}^{(k+1)}_{\bm{z},0}$ and $w_j\iid\mathcal{B}(1,1/2)$. By combining $\tilde{\bm{\alpha}}^{(k+1)}_{\bm{z},0}=\sigma\mpt{{\bm{\alpha}}^{(k+1)}_{\bm{z},0}}$ with the above symmetry, we have
%     \begin{align*}
%       \frac{m_{k+1}}{2}{\norm{\tilde{\bm{\alpha}}^{(k+1)}_{\bm{z},0} }_2^2}/{\norm{\tilde{\bm{\alpha}}^{(k)}_{\bm{z},0} }_2^2}= \sum_{j:w_j\not=0}\frac{m_{k+1}}{2}{ \pt{\alpha^{(k+1)}_{\bm{z},0,j}}^2}/{\norm{\tilde{\bm{\alpha}}^{(k)}_{\bm{z},0} }_2^2} \sim\chi^2_\omega
%     \end{align*}
%     where $\omega \sim \mathcal{B}(m_{k+1},{1}/{2}) $. Then, apply \cref{lem:Chi2_Binomial}, we can conclude that $\norm{\bm{\alpha}^{(k+1)}_{\bm{z},0}}_2 = \Theta(1)$ holds with probability at least $1-\exp\mpt{-\Omega(m)}$.

%     This completes the proof of the lemma by induction.
%   \end{proof}

  % With probability at least $1-\exp(-\Omega(m))$ with respect to the initialization, for any $0 \leq l \leq L$ and $\x \in \tilde{B}_Rl$, we have
  % \begin{align}
  %   \norm{\tilde{\bm{\alpha}}_0^{(l)}(\x)}_2 = \Theta(1) \qand \norm{\bm{\alpha}_0^{(l)}(\x)}_2 = O(1).
  % \end{align}
\end{lemma}

% The gradient at initialization can also be bounded by the following lemma.

% \begin{lemma}
%   \label{lem:Init_GradientBound}There exists a positive absolute constant $C$ such that with probability at least $1-\exp(-\Omega(m))$
%   with respect to the initialization,
%   for any $l \in\cl{ 0,1,\cdots,L}$ and $\x\in \tilde{B}_R$, we have
%   \begin{align*}
%     % \label{eq:Init_GradientBound}
%     \norm{\nabla_{\bm{W}^{(l)}} g_0(\x) }_{\mr{F}} = O(1)
%   \end{align*}
%   when $m$ is greater than the positive constant $C$.
% \end{lemma}
% \begin{proof}
%   Recalling \cref{eq:GradientExpr}, we can get the following result:
%   \begin{align*}
% \nabla_{\bm{W}^{(l)}} g_0(\x)=\bm{\gamma}^{(l)}_{\x,0}\tilde{\bm{\alpha}}^{(l),T}_{\x,0}.
% \end{align*}
% % \begin{align*}
% %     g(\x) = \pt{\prod_{r=l+1}^{L} \sqrt{\frac{2}{m_{r+1}}}  \bm{W}^{(r)} \bm{D}^{(r)}_{\x}} 
% %     \sqrt{\frac{1}{m_{l+1}}} \bm{W}^{(l)} \tilde{\bm{\alpha}}^{(l)}
% %     + b^{(L)},
% %   \end{align*}
% %   and hence
% %   \begin{align}
% %     \nabla_{\bm{W}^{(l)}} g(\x) &=
% %     \sqrt{\frac{1}{m_{l+1}}} \tilde{\bm{\alpha}}^{(l)} \pt{\prod_{r=l+1}^{L} \sqrt{\frac{2}{m_{r+1}}}  \bm{W}^{(r)} \bm{D}^{(r)}_{\x}}\nonumber\\
% %     &= \tilde{\bm{\alpha}}^{(l)} \pt{\prod_{r=l+1}^{L} \sqrt{\frac2{m_r}}  \bm{W}^{(r)} \bm{D}^{(r)}_{\x} }.\label{eq:GradientExpr}
% %   \end{align}
%   Then, to bound the two terms separately, we can apply \cref{lem:Init_WeightsProductBound}. Additionally, it is worth noting that for two vectors $\bm{a}$ and $\bm{b}$, we have
% \begin{align*}
% \norm{\bm{a} \bm{b}^T}_{\mathrm{F}}^2 = \Tr(\bm{a} \bm{b}^T \bm{b} \bm{a}^T) = \Tr(\bm{a}^T \bm{a} \bm{b}^T \bm{b}) = \norm{\bm{a}}_2^2 \norm{\bm{b}}_2^2.
% \end{align*}
% The result follows from these observations.
%   % Then the result follows from \cref{lem:Init_WeightsProductBound} to bound the two terms separately,
%   % and the fact that for two vectors $\bm{a},\bm{b}$,
%   % \begin{align*}
%   %   \norm{\bm{a} \bm{b}^T}_{\mathrm{F}}^2 = \Tr(\bm{a} \bm{b}^T \bm{b} \bm{a}^T) = \Tr(\bm{a}^T \bm{a} \bm{b}^T \bm{b}) = \norm{\bm{a}}_2^2 \norm{\bm{b}}_2^2.
%   % \end{align*}
% \end{proof}

\subsection{The training process}\label{subsec:Training}

In this subsection we will show that as long as the parameters and input do not change much, some the relative quantities can also be bounded.
We still focus on one parity in this subsection and suppress the superscript $p$ for convenience.

The most crucial result we will obtain in this subsection is the following proposition:
\begin{proposition}
  \label{prop:Training_KernelBound}
  % For any $\delta\in(0,1)$, suppose $\norm{\bm{W}^{(l)}_t - \bm{W}^{(l)}_0 }_{\mr{F}}= O(m^{1/4})$ holds for all $t\in T$.
  % Then for any $\x,\x' \in \mc{\X}$, with probability at least $1 - \exp(-\Omega(m^{5/6}))$, we have
  % \begin{align}
  %   \sup_{t\in T}\abs{K^{(p)}_t(\x,\x') - K^{(p)}_0(\x,\x')} = O \left( m^{-1/12}\sqrt{\ln m}\right).
  % \end{align}

  Fix $ \bm{z},\bm{z}' \in \tilde{B}_R$ and $T\subseteq[0,\infty)$.
  Suppose that $\norm{\bm{W}^{(l)}_t - \bm{W}^{(l)}_0}_{\mr{F}}= O(m^{1/4})$ holds for all $t\in T$ and $l\in\dk{0,1,\cdots,L}$.
  Then there exists a positive absolute constant $C$ such that when $m\geq C$, with probability at least $1 - \exp\xkm{-\Omega(m^{5/6})}$, for any $\x,\x' \in \mc{\X}$ such that $\norm{\x-\z}_2,\norm{\x'-\z'}_2 \leq O( 1/m)$, we have
  \[\sup_{t\in T}\abs{K^{(p)}_t(\x,\x') - K^{(p)}_0(\z,\z')}= O \xkm{R^2 m^{-{1}/{12}}\sqrt{\ln m}}.\]
\end{proposition}
% \begin{proposition}
%   \label{prop:Training_KernelBound}
%   % For any $\delta\in(0,1)$, suppose $\norm{\bm{W}^{(l)}_t - \bm{W}^{(l)}_0 }_{\mr{F}}= O(m^{1/4})$ holds for all $t\in T$.
%   % Then for any $\x,\x' \in \mc{\X}$, with probability at least $1 - \exp(-\Omega(m^{5/6}))$, we have
%   % \begin{align}
%   %   \sup_{t\in T}\abs{K^{(p)}_t(\x,\x') - K^{(p)}_0(\x,\x')} = O \left( m^{-1/12}\sqrt{\ln m}\right).
%   % \end{align}

%   Fix $ \bm{z},\bm{z}' \in \tilde{B}_R$ and $T\subseteq[0,\infty)$. Suppose that $\norm{\bm{W}^{(l)}_t - \bm{W}^{(l)}_0}_{\mr{F}}= O(m^{1/4})$ holds for all $t\in T$ and $l\in\cl{0,1,\cdots,L}$. Then there exists a positive absolute constant $C$ such that with probability at least $1 - \exp\mpt{-\Omega(m^{5/6})}$, we have
%   \[\sup_{t\in T}\abs{K^{(p)}_t(\z,\z') - K^{(p)}_0(\z,\z')}= O \mpt{m^{-{1}/{12}}\sqrt{\ln m}}\]
%   when $m$ is greater than the positive constant $C$.
% \end{proposition}
The proof of this proposition will be presented at the end of this subsection.
Combining this proposition with \cref{cor:Init_Kt_NTK}, we can derive the following corollary:
\begin{corollary}
  \label{cor:Training_Kernel_uniform}
  Fix $ \bm{z},\bm{z}' \in \tilde{B}_R$ and let $\delta\in(0,1)$, $T\subseteq[0,\infty)$.
  Suppose that $\norm{\bm{W}^{(l)}_t - \bm{W}^{(l)}_0}_{\mr{F}}= O(m^{1/4})$ holds for all $t\in T$ and $l\in\dk{0,1,\cdots,L}$.
  Then there exist some positive absolute constants $C_1>0$ and $C_2\geq 1$ such that with probability at least $1 - \delta$, for any $\x,\x' \in \mc{\X}$ such that $\norm{\x-\z}_2,\norm{\x'-\z'}_2 \leq O( 1/m)$, we have
  \[\sup_{t\in T}\abs{K^{(p)}_t(\x,\x') - \NTK(\z,\z')}= O \xkm{R^2 m^{-{1}/{12}}\sqrt{\ln m}},~\text{when}~m\geq C_1\xk{\ln(C_2/\delta)}^5.\]
\end{corollary}
% \begin{corollary}
%   \label{cor:Training_Kernel_uniform}
%   Fix $ \bm{z},\bm{z}' \in \tilde{B}_R$ and let $\delta\in(0,1)$, $T\subseteq[0,\infty)$. Suppose that $\norm{\bm{W}^{(l)}_t - \bm{W}^{(l)}_0}_{\mr{F}}= O(m^{1/4})$ holds for all $t\in T$ and $l\in\cl{0,1,\cdots,L}$. Then there exist some positive absolute constants $C_1>0$ and $C_2\geq 1$ such that with probability at least $1 - \delta$, we have
%   \[\sup_{t\in T}\abs{K^{(p)}_t(\z,\z') - \NTK(\z,\z')}= O \mpt{m^{-{1}/{12}}\sqrt{\ln m}}\]
%   when $m\geq C_1\pt{\ln(C_2/\delta)}^5$.
% \end{corollary}

% To prove \cref{prop:Training_KernelBound}


% \begin{lemma}
%   \label{lem:Training_GradientBound}
%   Let $\tau\in\left[\Omega\big(m^{-1}(\ln m)^{-3/2}\big), O\big({\sqrt{m}}/{(\ln m)^3}\big)\right]$ and fix $\x \in \tilde{B}_R\times\{1\}$.
%   Suppose $\norm{\bm{W}^{(l)}_t - \bm{W}^{(l)}_0 }_{\mr{F}}\leq \tau$ holds for all $t\in T$ and $l = 0,\dots,L$.
%   Then, with probability at least $1-\exp(-\Omega( m^{2/3} \tau^{2/3}))$ over the randomness of initialization, we have
%   \begin{align}
%     \sup_{t\in T}\norm{\nabla_{\bm{W}^{(l)}} g_t(\x) - \nabla_{\bm{W}^{(l)}} g_0(\x)}_{\mr{F}}= O\left( m^{-\frac{1}{6}} \tau^\frac{1}{3} \sqrt{\ln m} \right)
%   \end{align}
%   for all $l = 0,\dots,L$.
% \end{lemma}

To prove \cref{prop:Training_KernelBound}, we need to introduce some necessary lemmas.
In \cref{lem:Init_WeightsProductBound}, we have provided upper bounds for the norms of ${\widetilde{\bm{\alpha}}_{\x,0}^{(l)}}$, ${{\bm{\alpha}}_{\x,0}^{(l)}}$, $\bm{\gamma}^{(l)}_{\x,0}$ and $\nabla_{\bm{W}^{(l)}} g_0(\x)$ at initialization.
Next, we aim to prove that under perturbations in the parameters and the input, the corresponding changes in these quantities will also be small.

In fact, similar lemmas can be found in \citet{allen-zhu2019_ConvergenceTheory}, although they have different conditions compared to the lemmas in this paper.
For example, in \citet{allen-zhu2019_ConvergenceTheory}, the input points are constrained to lie on a sphere, the input and output layers are not involved in training, and each hidden layer has the same width.

However, the most crucial point is that \citet{allen-zhu2019_ConvergenceTheory} did not consider the impact of small perturbations in the input, which is vital for proving uniform convergence.
In fact, the slight perturbation between $\x$ and $\z$ can be regarded as taking a slight perturbation on $\bm{W}^{(0)}$, with other $\bm{W}^{(l)} $ fixed.
Additionally, since this paper fixes the number of layers $L$, there is no need to consider the impact of $L$ on the bounds.
This simplifies the proof of the corresponding conclusions.

Inspired by \citet[Lemma 8.2]{allen-zhu2019_ConvergenceTheory}, we can prove the following lemma:

\begin{lemma}
  \label{lem:NN_forward_perturbation}
  Let $\Delta=O(1/\sqrt{m})$, $\tau\in\zk{\Delta\sqrt{m},O\xkm{{\sqrt{m}}/{(\ln m)^3}}}$, $T\subseteq [0,\infty)$ and fix ${\bm{z}} \in\tilde{B}_R$.
  Suppose that $\norm{\bm{W}^{(l)}_t - \bm{W}^{(l)}_0 }_{\mr{F}}\leq\tau$ holds for all $t\in T$ and $l\in\dk{0,1,\cdots,L}$.
  Then there exists a positive absolute constant $C$ such that with probability at least $1-\exp\xkm{-\Omega(m^{2/3}\tau^{2/3})}$, for all $t\in T$, $l\in\dk{0,1,\cdots,L}$ and $\x \in\tilde{B}_R$ such that $\norm{\x-\bm{z}}_2 \leq \Delta$, we have
  % \begin{enumerate}[$(a)$~]

  $(a)$ $\norm{\widetilde{\bm{\alpha}}^{(l)}_{\x,t} - \widetilde{\bm{\alpha}}_{\z,0}^{(l)}}_2 = O(R \tau/\sqrt{m})$ and thus $\norm{\widetilde{\bm{\alpha}}^{(l)}_{\x,t}}_2=O(R)$;

  $(b)$ $\norm{\bm{D}_{\x,t}^{(l)} - \bm{D}_{{\bm{z}},0}^{(l)}}_0 = O(m^{2/3} \tau^{2/3})$ and $\norm{\xk{\bm{D}_{\x,t}^{(l)} - \bm{D}_{{\bm{z}},0}^{(l)}}\widetilde{\bm{\alpha}}_{\x,t}^{(l)}}_2= O(R{\tau}/{\sqrt{m}})$;

  $(c)$ $\norm{{\bm{\alpha}}^{(l)}_{\x,t} - {\bm{\alpha}}_{\z,0}^{(l)}}_2 = O(R \tau/\sqrt{m})$ and thus $\norm{{\bm{\alpha}}^{(l)}_{\x,t}}_2=O(R)$,

  % \end{enumerate}
  \noindent when $m$ is greater than the positive constant $C$.
\end{lemma}
\begin{proof}
  We use mathematical induction to prove this lemma.
  When $l=0$, it can be easily verified that $\bs{D}_{\bs{x},t}^{(0)} - \bs{D}_{\bs{z},0}^{(0)} = \bm{O}$ and $\norm{\widetilde{\bm{\alpha}}^{(l)}_{\x,t} - \widetilde{\bm{\alpha}}_{\z,0}^{(l)}}_2=\norm{{{\bm{\alpha}}}_{\x,t}^{(0)} -{{\bm{\alpha}}}^{(0)}_{\bm{z},0}}_2=\norm{\tilde{\x}-\tilde{\z}}_2 = \norm{\x-{\bm{z}}}_2 \leq \Delta\leq \tau/\sqrt{m}$, where $\bm{O}$ represents the zero matrix, $\tilde{\x}$ and $\tilde{\bm{z}}$ are extended vectors with an additional coordinate of $1$.
  Thus, all the statements hold for $l=0$.
  Now we assume that this lemma holds for $l=k\in\dk{0,1,\cdots,L-1}$.

  $(a)$  First of all, we can decompose $\widetilde{\bm{\alpha}}^{(k+1)}_{\x,t} -\widetilde{\bm{\alpha}}^{(k+1)}_{\z,0}$ as following:
  \begin{equation*}
    \begin{aligned}
      &\widetilde{\bm{\alpha}}^{(k+1)}_{\x,t} -\widetilde{\bm{\alpha}}^{(k+1)}_{\z,0}=
      \sqrt{\frac{2}{m_{k+1}}} \bm{W}^{(k)}_t {\bm{\alpha}}^{(k)}_{\x,t} - \sqrt{\frac{2}{m_{k+1}}} \bm{W}_0^{(k)}{\bm{\alpha}}^{(k)}_{\z,0}\\
      &\qquad=\sqrt{\frac{2}{m_{k+1}}} \xk{\bm{W}^{(k)}_t - \bm{W}^{(k)}_0} {\bm{\alpha}}^{(k)}_{\x,t} + \sqrt{\frac{2}{m_{k+1}}}\bm{W}^{(k)}_0 \xk{{\bm{\alpha}}^{(k)}_{\x,t} - {\bm{\alpha}}^{(k)}_{\z,0}}.
    \end{aligned}
  \end{equation*}
  From the above equation, we can deduce that $(a)$ holds for $l=k+1$ by the induction hypothesis and \cref{lem:Init_WeightsProductBound}.
  % and $\norm{{\bm{\alpha}}^{(k)}_{\x,t}}_2\leq \norm{{\bm{\alpha}}^{(k)}_{\z,0}}_2+\norm{{\bm{\alpha}}^{(k)}_{\x,t}-{\bm{\alpha}}^{(k)}_{\z,0}}_2$.
  % $\norm{\widetilde{\bm{\alpha}}^{(k+1)}_{\x,t} -\widetilde{\bm{\alpha}}^{(k+1)}_{\z,0}}_2 \leq O\mpt{{\tau}/{\sqrt{m}}}$.

  $(b)$
% Next, we use \cref{lem:Allen_Claim 8.3} to prove part $(a)(b)$. 
  Let us consider the following choices in
  \cref{lem:Allen_Claim 8.3}:
  \begin{align*}
    \bm{g}&=\frac{\widetilde{\bm{\alpha}}^{(k+1)}_{\z,0}}{\sqrt{{2}/{m_{k+1}}}\norm{{\bm{\alpha}}_{\z,0}^{(k)}}_2}=\frac{\bm{W}_0^{(k)}{\bm{\alpha}}_{\z,0}^{(k)}}{\norm{{\bm{\alpha}}_{\z,0}^{(k)}}_2},\qquad~
    \bm{g}'=\frac{\widetilde{\bm{\alpha}}^{(k+1)}_{\x,t}-\widetilde{\bm{\alpha}}^{(k+1)}_{\z,0}}{\sqrt{{2}/{m_{k+1}}}\norm{{\bm{\alpha}}_{\z,0}^{(k)}}_2}.
  \end{align*}
  It follows that $\bm{g}\sim N(0,\bm{I})$ if we fix ${\bm{\alpha}}_{\z,0}^{(k)}$ and only consider the randomness of $\bm{W}^{(k)}_0$.
  Also, we have $\norm{\bm{g}'}_2\leq O(\tau/\sqrt{m})\cdot O(\sqrt{m})\leq O(\tau)$ holds for all $\x \in\tilde{B}_R$ such that $\norm{\x-\bm{z}}_2 \leq \Delta$ since we have previously shown that $\norm{{\bm{\alpha}}_{\z,0}^{(k)}}_2=\Theta(R)$ in \cref{lem:Init_LayerOutputBounds}.
  Therefore, we can choose $\delta=\Theta(\tau)$ such that $\norm{\bm{g}'}_2\leq\delta$.
  Then, we can obtain
  \begin{align*}
    \bm{g}+\bm{g}'=\frac{\widetilde{\bm{\alpha}}^{(k+1)}_{\x,t}}{\sqrt{\frac{2}{m_{k+1}}}\norm{{\bm{\alpha}}_{\z,0}^{(k)}}_2},~~\bm{D}'=\bm{D}_{\x,t}^{(k+1)} - \bm{D}_{{\bm{z}},0}^{(k+1)}~~\text{and}~~\bm{u}=\frac{\xk{\bm{D}_{\x,t}^{(k+1)} - \bm{D}_{{\bm{z}},0}^{(k+1)}}\widetilde{\bm{\alpha}}^{(k+1)}_{\x,t}}{\sqrt{\frac{2}{m_{k+1}}}\norm{{\bm{\alpha}}_{\z,0}^{(k)}}_2}&,
  \end{align*}
  where $\bm{D}'$ and $\bm{u}$ are defined as in \cref{lem:Allen_Claim 8.3}.
  By applying \cref{lem:Allen_Claim 8.3}, we can get $\norm{\bm{D}'}_0\leq O(m^{2/3}\tau^{2/3})$ and $\norm{\bm{u}}_2\leq O(\delta)$, which establish the conclusion of part $(b)$.


% As to the second statement, we define $\bs{u}^{(l)}$ as following:

% % Let $u^{(l)} = (\bs{D}^{(l)}_{\bs{x},t} - \bs{D}^{(l)}_{\bs{z},0}) \bs{\alpha}_t^{(l)}(x)$

% \begin{equation}
% \begin{aligned}
% \bm{u}^{(l)} &:= (\bm{D}_{\x,t}^{(l)} - \bm{D}_{{\bm{z}},0}^{(l)}) \bm{\alpha}^{(l)}_t(\x)=
% \left(\bm{D}_{\x,t}^{(l)} - \bm{D}_{{\bm{z}},0}^{(l)}\right)\left(\bm{\alpha}^{(l)}_0({\bm{z}})+(\bm{\alpha}^{(l)}_t(\x)-\bm{\alpha}^{(l)}_0({\bm{z}}))\right) \\
% &=(\bm{D}_{\x,t}^{(l)} - \bm{D}_{{\bm{z}},0}^{(l)})\left[\sqrt{\frac{2}{m_l}}\bm{W}^{(l-1)}_0{\tilde{\bm{\alpha}}}^{(l-1)}_0({\bm{z}})+
% \left(\bm{\alpha}^{(l)}_t(\x)-\bm{\alpha}^{(l)}_0({\bm{z}})\right)\right],
% \end{aligned}
% \end{equation}


% Now we re-scale $\bm{u}^{(l)}$ by ${1}/{\norm{\tilde{\bm{\alpha}}^{(l-1)}_0({\bm{z}})}_2}$, then with Lemma \ref{lem: Recur_D0} we know (let$\bm{D}'=\bm{D}_{\x,t}^{(l)} - \bm{D}_{{\bm{z}},0}^{(l)}$, $\bm{g}=\sqrt{{2}/{m_l}}\bm{W}^{(l-1)}_0{ \bm{\tilde{\alpha}}^{(l-1)}_0({\bm{z}})}/{\norm{\tilde{\bm{\alpha}}^{(l-1)}_0({\bm{z}})}_2}$ and $\bm{g}'=\big({\bm{\alpha}^{(l)}_t(\x)-\bm{\alpha}^{(l)}_0({\bm{z}})}\big)/{\norm{\tilde{\bm{\alpha}}^{(l-1)}_0({\bm{z}})}_2}$) with probability at least $1-\exp\big(-\Omega(m^{2/3} \tau^{2/3})\big)$, we have:
% \begin{align}
% \label{a2}
% \norm{\bm{u}^{(l)}}_0 \leq \norm{\bm{D}_{\x,t}^{(l)} - \bm{D}_{{\bm{z}},0}^{(l)}}_0 = O\big(m^{\frac{2}{3}}\tau^\frac{2}{3}\big); \quad \text{and} \quad
% \norm{\bm{u}^{(l)}}_2 = O\big(\frac{\tau}{\sqrt{m}}\big).
% \end{align}



  $(c)$ Further, the third statement can be directly obtained from the following inequality:
  \begin{align*}
    \norm{{\bm{\alpha}}^{(k+1)}_{\x,t} -{\bm{\alpha}}^{(k+1)}_{\bm{z},0}}_2 &\leq \norm{\bm{D}^{(k+1)}_{{\bm{z}},0}\xk{\widetilde{\bm{\alpha}}^{(k+1)}_{\x,t} - \widetilde{\bm{\alpha}}^{(k+1)}_{\bm{z},0}}}_2+\norm{\xk{\bm{D}^{(k+1)}_{\x,t} - \bm{D}^{(k+1)}_{{\bm{z}},0} }\widetilde{\bm{\alpha}}^{(k+1)}_{\x,t} }_2.
  \end{align*}
  Thus, the proof of this lemma is complete.
\end{proof}

In the proof of \cref{lem:NN_forward_perturbation}, we use the following result:
\begin{lemma}[\citet{allen-zhu2019_ConvergenceTheory} Claim 8.3]
  \label{lem:Allen_Claim 8.3}
  % Suppose $\delta_2 \in [0,O(1)]$, $\delta_\infty \in \big[0,1/({4\sqrt{m}})\big]$ and $m_1 \in [1,M]$.
  % Suppose $\bm{W} \in \R^{m\times m_1}$ is a random matrix with entries drawn i.i.d from $\mathcal{N}(0,1)$.
  % With probability at least $1- \exp(-\Omega\big(m^{3/2}\delta_\infty\big))$, the following holds.
  % Fix any unit vector $\bm{h} \in \R^m$, and for all $\bm{g}\in \R^{m_1}$ that can be written as
  % \begin{align*}
  %   \bm{g}= \bm{g}_1+\bm{g}_2\quad \text{where} \quad\norm{\bm{g}_1}_2\leq \delta_2 \quad \text{and} \quad \norm{\bm{g}_2}_\infty \leq \delta_\infty.
  % \end{align*}
  % Let $\bm{D}' \in \R^{m\times m}$ be the diagonal matrix where
  % \begin{align*}
  % (\bm{D}')
  %   _{k,k}  =  \bm{1}\big\{\big(\sqrt{{2}/{m}}\bm{Wh}+\bm{g}\big)_k \geq 0\big\} - \bm{1}\big\{\big(\sqrt{{2}/{m}}\bm{Wh}\big)_k \geq 0\big\},\quad~\forall k \in [m].
  % \end{align*}
  % Then, letting $\x=\bm{D}'\left(\sqrt{{2}/{m}}\bm{Wh}+\bm{g}\right) \in \R^m$, we have
  % \begin{align*}
  %   \norm{\x}_0 \leq \norm{\bm{D}'}_0 = O\big(m(\delta_2)^\frac{2}{3}+\delta_\infty m^\frac{3}{2}\big)\quad  and \quad \norm{\x}_2 = O\big(\delta_2+(\delta_\infty)^\frac{3}{2}m^\frac{3}{4}\big).
  % \end{align*}

  Suppose each entry of $\bm{g} \in \mathbb{R}^m$ follows $g_i \iid {N}(0,1)$.
  For any $\delta > 0$, with probability at least $1-\exp\xkm{-\Omega(m^{2/3}\delta^{2/3})} $, the following proposition holds:

  Select $\bm{g}' \in \mathbb{R}^m$ such that $ \norm{\bm{g}'}_2\leq\delta.$
  Let $\bm{D}' = \mr{diag}(D'_{kk})$ be a diagonal matrix, where the $k$-th diagonal element
  $D_{kk}' $ follows
  $$D_{k k}^{\prime}=\bs{1}\left\{\left(g+g^{\prime}\right)_{k} \geq 0\right\}-\bs{1}\left\{g_{k} \geq 0\right\},\quad k\in[m].$$
  If we define $\bm{u} = \bm{D}'(\bm{g}+\bm{g}')$, then it satisfies the following inequalities:
  \[\norm{\bm{u}}_0 \leq \norm{\bm{D}'}_0 \leq O\xkm{m^{2/3}\delta^{2/3}} \quad \text{ and }\quad \norm{\bm{u}}_2 \leq O(\delta).\]
\end{lemma}

% Before proving \cref{prop:Training_KernelBound}, let us introduce some lemmas from \citet{allen-zhu2019_ConvergenceTheory}. By inspecting the proof of \citet[Lemma 8.2]{allen-zhu2019_ConvergenceTheory}(perturb the first layer), we have the following lemma: 

% \begin{lemma}
%   [Forward perturbation, Lemma 8.2 in \citet{allen-zhu2019_ConvergenceTheory} ]
%   \label{lem:NN_forward_perturbation}
%   Fix $\x \in \mc{\X}$ and suppose $\norm{\bm{W}^{(l)}_t - \bm{W}^{(l)}_0 }_{\mr{F}}\leq\tau=O\Big({\sqrt{m}}/{(\ln m)^3}\Big)$ holds for all $t\in T$, with probability at least $1-\exp({-\Omega( m^{2/3} \tau^{2/3})})$, we have
%   \begin{align} 
%     \label{eq:NN_ForwardPerturbation}
%     \begin{split}
%     &\norm{\bm{D}^{(l)}_{\x,t}-\bm{D}^{(l)}_{\x,0}}_0=O(m^{2/3}\tau^{2/3}); \\ & 
%     \norm{\tilde{\bm{\alpha}}^{(l)}_t(\x) - \tilde{\bm{\alpha}}_0^{(l)}(\x)}_2= O\mpt{\frac{\tau\sqrt{\ln m} }{\sqrt{m}}}
%     \end{split}
%   \end{align}
%   hold for all $t\in T$ and $l \in \{0,1,\cdots, L\}$.
% \end{lemma}

% \begin{lemma}
%   [Forward perturbation, Lemma 8.2 in \citet{allen-zhu2019_ConvergenceTheory} ]\label{lem:NN_forward_perturbation}
%   Let $\tau=O\mpt{{\sqrt{m}}/{(\ln m)^3}}$, $T\subseteq [0,\infty)$ and fix ${\bm{z}} \in\tilde{B}_R$. Suppose that $\norm{\bm{W}^{(l)}_t - \bm{W}^{(l)}_0 }_{\mr{F}}\leq\tau$ holds for all $t\in T$ and $l\in\cl{0,1\cdots,L}$. Then there exists a positive absolute constant $C$ such that with probability at least $1-\exp\mpt{-\Omega(m^{2/3}\tau^{2/3})}$, for all $t\in T$ and $l\in\cl{0,1,\cdots,L}$, we have
%   \begin{enumerate}[$(a)$~]
% % \item $\norm{\bm{\alpha}^{(l)}_t(\x) - \bm{\alpha}^{(l)}_0({\bm{z}}) }_2= O\big(\tau/\sqrt{m}\big)$;
%     \item $\norm{\bm{D}_{\bm{z},t}^{(l)} - \bm{D}_{{\bm{z}},0}^{(l)}}_0 = O(m^{2/3} \tau^{2/3})$;
%     \item $\norm{\pt{\bm{D}_{\bm{z},t}^{(l)} - \bm{D}_{{\bm{z}},0}^{(l)}}\bm{\alpha}_{\z,t}^{(l)}}_2= O(\tau/\sqrt{m})$;
%     \item $\norm{\tilde{\bm{\alpha}}^{(l)}_{\z,t} - \tilde{\bm{\alpha}}_{\z,0}^{(l)}}_2 = O(\tau/\sqrt{m})$,
%   \end{enumerate}
%   when $m$ is greater than the positive constant $C$.
% \end{lemma}
% \begin{proof}
%   The proof of this lemma is similar to the proof of \cref{lem:NN_forward_perturbation2}.
% \end{proof}


% By inspecting the proof of \citet[Lemma 8.7]{allen-zhu2019_ConvergenceTheory}(perturbing the last layer), we can relax their condition $\norm{\x}_2 =1$ to $\norm{\x}_2 \leq R$ and $m_l = m$ to $m_l \in [m,C_m m]$, deriving the following lemma:

Inspired by \citet[Lemma 8.7]{allen-zhu2019_ConvergenceTheory}, we then have the following lemma:
\begin{lemma}
  \label{lem:backward perturbation}
  Let $\Delta=O(1/\sqrt{m})$, $\tau\in\zk{\Delta\sqrt{m},O\xkm{{\sqrt{m}}/{(\ln m)^3}}}$, $T\subseteq [0,\infty)$ and fix ${\bm{z}} \in\tilde{B}_R$.
  Suppose that $\norm{\bm{W}^{(l)}_t - \bm{W}^{(l)}_0 }_{\mr{F}}\leq\tau$ holds for all $t\in T$ and $l\in\dk{0,1,\cdots,L}$.
  Then there exists a positive absolute constant $C$ such that with probability at least $1-\exp\xkm{-\Omega(m^{2/3}\tau^{2/3})}$, for all $t\in T$, $l\in\dk{0,1,\cdots,L}$ and $\x \in\tilde{B}_R$ such that $\norm{\x-\bm{z}}_2 \leq \Delta$, we have
  \begin{align*}
    \norm{{\bm{\gamma}}^{(l)}_{\x,t} - {\bm{\gamma}}^{(l)}_{\z,0}}_2
    = O\xkm{m^{-1/6} \tau^{1/3}\sqrt{\ln m}}\qquad~\text{and thus}\qquad~ \norm{{\bm{\gamma}}^{(l)}_{\x,t}}_2=O(1)
  \end{align*}
\end{lemma}
when $m$ is greater than the positive constant $C$.
% \begin{proof}
%   We prove this lemma by mathematical induction. For $l=L$, we have ${\bm{\gamma}}^{(L)}_{\x,t} - {\bm{\gamma}}^{(L)}_{\z,0}=0$. Now we assume that this lemma holds for $l=k\in\zk{L}$, then our goal is to prove that the lemma also holds for $l=k-1$. First, we can obtain
% \begin{align*}
%     \bm{\gamma}^{(k-1)}_{\x,t}=\xk{\sqrt{\frac2{m_{k}}}\bm{W}^{(k)}_t \bm{D}_{\x,t}^{(k)}}^T\bm{\gamma}^{(k)}_{\x,t}=\sqrt{\frac2{m_{k}}} \bm{D}_{\x,t}^{(k)}\bm{W}^{(k),T}_t\bm{\gamma}^{(k)}_{\x,t}
%   \end{align*}
%   holds for $k\in[L]$. If we denote $\Delta\bm{\gamma}^{(l)}={\bm{\gamma}}^{(l)}_{\x,t}-\bm{\gamma}^{(l)}_{\z,0}$, $\Delta\bm{W}^{(l)}={\bm{W}}^{(l)}_{t}-\bm{W}^{(l)}_{0}$ and $\bm{D}^{(l)\prime}={\bm{D}}^{(l)}_{\x,t}-\bm{D}^{(l)}_{\z,0}$, then we have $\norm{\bm{W}^{(k)}_t}_2\leq\norm{\bm{W}^{(k)}_t}_2+\norm{\Delta\bm{W}^{(k)}}_2=O(\sqrt{m})$ and
%   \begin{align*}
%     \Delta\bm{\gamma}^{(k-1)} =\sqrt{\frac2{m_k}}\Big[{\bm{D}}^{(k)}_{\x,t}{\bm{W}}^{(k),T}_{t}\Delta{\bm{\gamma}}^{(k)}+{\bm{D}}^{(k)}_{\x,t}\Delta{\bm{W}}^{(k),T}{\bm{\gamma}}^{(k)}_{\z,0}
%     +{\bm{D}}^{(k)\prime}{\bm{W}}^{(k),T}_0{\bm{\gamma}}^{(k)}_{\z,0}&\Big].
% % +\bm{\gamma}_{k+1}\bm{W}^{(k)\prime}\bm{D}^{(k)\prime}}.
%   \end{align*}
%   Let us denote the three terms within the square brackets, excluding the factor `$\sqrt{2/m_k}$' outside the brackets, as $\bm{u}_1$ to $\bm{u}_3$ respectively.
% % Let us denote the three terms within the square brackets as $\bm{u}_1$ to $\bm{u}_3$ respectively, while ignoring the factor $\sqrt {2/m_k}$ outside the brackets. 
%   First of all, it is easy to check that, with probability at least $1-\exp\xkm{-\Omega(m)}$, we have
%   \begin{align*}
%     \norm{\bm{u}_1}_2=O\xkm{m^{1/3} \tau^{1/3}\sqrt{\ln m}},\qquad~ \norm{\bm{u}_2}_2=O\xkm{\tau}.
%   \end{align*}
% %   \begin{align*}
% %     \begin{gathered}
% % \norm{\tilde{\bm{W}}^{(r)}}_2\leq\norm{{\bm{W}}^{(r)}_0}_2+\norm{{\bm{W}}^{(r)\prime}}_2= O(\sqrt{m}),\qquad~\norm{\tilde{\bm{D}}^{(r)}_{\z}}_2=\norm{{\bm{D}}^{(r)}_{\z,0}}_2+\norm{{\bm{D}}^{(r)\prime}}_2=O(1).
% %     \end{gathered}
% %   \end{align*}

%   As for $\bm{u}_3$, if $\bm{\gamma}^{(k)}_{\z,0}=\bm{0}$ or $\norm{\bm{D}^{(k)\prime}}_0=0$, we have $\norm{\bm{u}_3}_2=0$. Therefore, we consider the case where $\bm{\gamma}^{(k)}_{\z,0}\not=\bm{0}$ and $\norm{\bm{D}^{(k)\prime}}_0\geq1$. Denote $\hat{\bm{\gamma}}={\bm{\gamma}^{(k)}_{\z,0}}/\norm{\bm{\gamma}^{(k)}_{\z,0}}_2$ for $\bm{\gamma}^{(k)}_{\z,0}\in\mb{R}^{m_{k+1}}\backslash\dk{\bm{0}}$, we can get
%   \begin{align*}
%     \norm{\bm{u}_3}_2&\leq \norm{{\bm{D}}^{(k)\prime}{\bm{W}}^{(k),T}_0\hat{\bm{\gamma}}}_2\norm{\bm{\gamma}^{(k)}_{\z,0}}_2\leq O\xkm{1}\norm{{\bm{D}}^{(k)\prime}{\bm{W}}^{(k),T}_0\hat{\bm{\gamma}}}_2.
%   \end{align*}
%   Using the randomness of $\bm{W}^{(k)}_0$, for fixed $\hat{\bm{\gamma}}$, we have $\bm{W}^{(k),T}_0\hat{\bm{\gamma}}\sim {N}(\bm{0},\bm{I}_{m_k})$. According to \cref{lem:NN_forward_perturbation}(b) and \cref{lem:Recur_uv}, choosing $s=\Theta(m^{2/3}\tau^{2/3})$ such that $\norm{\bm{D}^{(k)\prime}}\leq s$, then with probability at least $1-\exp\xkm{-\Omega(m^{2/3}\tau^{2/3})}$, we can get
%   \begin{align*}
%     &\norm{{\bm{D}}^{(k)\prime}{\bm{W}}^{(k),T}_0\hat{\bm{\gamma}}}_2=\sup_{\bm{u}:\norm{\bm{u}}_2=1}\norm{\bm{u}^T{\bm{D}}^{(k)\prime}{\bm{W}}^{(k),T}_0\hat{\bm{\gamma}}\cdot 1}_2\\
%     &\quad=\sup_{\bm{u}:\norm{\bm{u}}_2=1}\norm{\xk{{\bm{D}}^{(k)\prime}\bm{u}}^T{\bm{W}}^{(k),T}_0\hat{\bm{\gamma}}\cdot 1}_2 =O\xkm{\sqrt{s\ln m}}
%     % =O\mpt{m^{1/3}\tau^{1/3}\sqrt{\ln m}},
%   \end{align*}
%   because of $\norm{\bm{D}^{(k)\prime}\bm{u}}_0\leq\norm{\bm{D}^{(k)\prime}}_0\leq s$ and $\norm{1}_0=1\leq \norm{\bm{D}^{(k)\prime}}_0\leq s$. Combining the above discussions, we can conclude the result.
% \end{proof}

% % \begin{lemma}
% %   [Backward perturbation, Lemma 8.7 in \citet{allen-zhu2019_ConvergenceTheory}]\label{lem:backward perturbation}Let $\tau=O\mpt{\sqrt{m}/{(\ln m)^3}}$, $s \asymp m^{2/3} \tau^{2/3}$ and fix $\bm{z} \in \mathcal{X}$. Then, with probability at least $1-\exp\mpt{-\Omega(m^{2/3} \tau^{2/3})}$ over the random initialization, the following result holds:

% %   For any perturbation matrices $\bm{D}^{(r)\prime}$, $\bm{W}^{(r)\prime}$ (where $r = 0,1,\cdots,L$) satisfying the following conditions:
% % % , with probability at least $1-\exp(-\Omega(s\ln m)) $ over the randomness of $\bm{W}^{(l)}_0$ for all $l \in \{0,1,\cdots,L\}$, if the perturbation
% %   \begin{enumerate}
% %     \item each $ \bm{D}^{(r)\prime}\in[-3,3]^{m_r\times m_r}$ is diagonal, and $\norm{\bm{D}^{(r)\prime}}_0\leq s$;
% %     \item each $\bm{W}^{(r)\prime}$ satisfies $\norm{\bm{W}^{(r)\prime}}_F \leq \tau$,
% %   \end{enumerate}
% %   denote $\tilde{\bm{W}}^{(r)}={\bm{W}}^{(r)}_0+{\bm{W}}^{(r)\prime}$, $\tilde{\bm{D}}^{(r)}_{\z}={\bm{D}}^{(r)}_{\z,0}+{\bm{D}}^{(r)\prime}$ and
% %   \[\tilde{\bm{\gamma}}^{(l)}_{\z}=\pt{\prod_{r=l+1}^L \sqrt{\frac2{m_r}}\tilde{\bm{W}}^{(r)} \tilde{\bm{D}}^{(r)}_{\bm{z}}}^T\in\mb{R}^{m_{l+1}},\]
% %   then for all $l = 0,1,\cdots,L$, we have the following bound:
% %   \begin{align*}
% %     \norm{\tilde{\bm{\gamma}}^{(l)}_{\z} - {\bm{\gamma}}^{(l)}_{\z,0}}_2
% %     = O\mpt{m^{-1/6} \tau^{1/3}\sqrt{\ln m}}.
% %   \end{align*}
% % \end{lemma}
% % \begin{proof}
% %   We prove this lemma by mathematical induction. For $l=L$, we have $\tilde{\bm{\gamma}}^{(L)}_{\z} - {\bm{\gamma}}^{(L)}_{\z,0}=0$. Now we assume that this lemma holds for $l=k\in\bk{L}$, then our goal is to prove that the lemma also holds for $l=k-1$. First, we can obtain
% %   \begin{align*}
% %     \tilde{\bm{\gamma}}^{(k-1)}_{\z}=\pt{\sqrt{\frac2{m_{k}}}\tilde{\bm{W}}^{(k)} \tilde{\bm{D}}^{(k)}_{\bm{z}}}^T\tilde{\bm{\gamma}}^{(k)}_{\z}=\sqrt{\frac2{m_{k}}}\tilde{\bm{D}}^{(k)}_{\bm{z}}\tilde{\bm{W}}^{(k),T}\tilde{\bm{\gamma}}^{(k)}_{\z};\\
% %     \bm{\gamma}^{(k-1)}_{\z,0}=\pt{\sqrt{\frac2{m_{k}}}\bm{W}^{(k)}_0 \bm{D}_{\bm{z},0}^{(k)}}^T\bm{\gamma}^{(k)}_{\z,0}=\sqrt{\frac2{m_{k}}} \bm{D}_{\bm{z},0}^{(k)}\bm{W}^{(k),T}_0\bm{\gamma}^{(k)}_{\z,0}
% %   \end{align*}
% %   hold for $k=[L]$. If we denote $\Delta\bm{\gamma}^{(l)}_{\z}=\tilde{\bm{\gamma}}^{(l)}_{\z}-\bm{\gamma}^{(l)}_{\z,0}$, then we have
% %   \begin{align*}
% %     \Delta\bm{\gamma}^{(k-1)}_{\z}
% % % &=\sqrt{\frac2{m_{k}}}\tilde{\bm{D}}^{(k)}_{\bm{z}}\tilde{\bm{W}}^{(k),T}\tilde{\bm{\gamma}}^{(k)}_{\z}-\sqrt{\frac2{m_{k}}} \bm{D}_{\bm{z}}^{(k)}\bm{W}^{(k),T}\bm{\gamma}^{(k)}_{\z}\\
% % % &\quad=\sqrt{\frac2{m_k}}\bk{\Delta\bm{\gamma}_{k+1}{\mpt{\bm{W}^{(k)} +\bm{W}^{(k)\prime}} \mpt{\bm{D}^{(k)}_{\bm{z}}+\bm{D}^{(k)\prime}}}+\bm{\gamma}_{k+1}\mbk{\mpt{\bm{W}^{(k)} +\bm{W}^{(k)\prime}} \mpt{\bm{D}^{(k)}_{\bm{z}}+\bm{D}^{(k)\prime}}-\bm{W}^{(k)} \bm{D}_{\bm{z}}^{(k)}}}\\
% %     =\sqrt{\frac2{m_k}}\Big[\tilde{\bm{D}}^{(k)}_{\bm{z}}\tilde{\bm{W}}^{(k),T}\Delta{\bm{\gamma}}^{(k)}_{\z}+\tilde{\bm{D}}^{(k)}_{\bm{z}}{\bm{W}}^{(k)\prime,T}{\bm{\gamma}}^{(k)}_{\z,0}
% %     +{\bm{D}}^{(k)\prime}{\bm{W}}^{(k),T}_0{\bm{\gamma}}^{(k)}_{\z,0}&\Big].
% % % +\bm{\gamma}_{k+1}\bm{W}^{(k)\prime}\bm{D}^{(k)\prime}}.
% %   \end{align*}
% %   Let us denote the three terms within the square brackets, excluding the factor `$\sqrt{2/m_k}$' outside the brackets, as $\bm{u}_1$ to $\bm{u}_3$ respectively.
% % % Let us denote the three terms within the square brackets as $\bm{u}_1$ to $\bm{u}_3$ respectively, while ignoring the factor $\sqrt {2/m_k}$ outside the brackets. 
% %   First of all, it is easy to check that, with probability at least $1-\exp\mpt{-\Omega(m)}$, we have
% %   \begin{align*}
% %     \norm{\bm{u}_1}_2=O\mpt{m^{1/3} \tau^{1/3}\sqrt{\ln m}},\qquad~ \norm{\bm{u}_2}_2=O\mpt{\tau}
% %   \end{align*}
% %   since
% %   \begin{align*}
% %     \begin{gathered}
% % \norm{\tilde{\bm{W}}^{(r)}}_2\leq\norm{{\bm{W}}^{(r)}_0}_2+\norm{{\bm{W}}^{(r)\prime}}_2= O(\sqrt{m}),\qquad~\norm{\tilde{\bm{D}}^{(r)}_{\z}}_2=\norm{{\bm{D}}^{(r)}_{\z,0}}_2+\norm{{\bm{D}}^{(r)\prime}}_2=O(1).
% %     \end{gathered}
% %   \end{align*}
% %   As for $\bm{u}_3$, if $\bm{\gamma}^{(k)}_{\z,0}=\bm{0}$ or $\norm{\bm{D}^{(k)\prime}}_0=0$, we have $\norm{\bm{u}_3}_2=0$. Therefore, we consider the case where $\bm{\gamma}^{(k)}_{\z,0}\not=\bm{0}$ and $\norm{\bm{D}^{(k)\prime}}_0\geq1$. Denote $\hat{\bm{\gamma}}={\bm{\gamma}^{(k)}_{\z,0}}/\norm{\bm{\gamma}^{(k)}_{\z,0}}_2$ for $\bm{\gamma}^{(k)}_{\z,0}\in\mb{R}^{m_{k+1}}\backslash\cl{\bm{0}}$, we can get
% %   \begin{align*}
% %     \norm{\bm{u}_3}_2&\leq \norm{{\bm{D}}^{(k)\prime}{\bm{W}}^{(k),T}_0\hat{\bm{\gamma}}}_2\norm{\bm{\gamma}^{(k)}_{\z,0}}_2\leq O\mpt{1}\norm{{\bm{D}}^{(k)\prime}{\bm{W}}^{(k),T}_0\hat{\bm{\gamma}}}_2.
% %   \end{align*}
% %   Using the randomness of $\bm{W}^{(k)}_0$, for fixed $\hat{\bm{\gamma}}$, we have $\bm{W}^{(k),T}_0\hat{\bm{\gamma}}\sim\mathcal{N}(\bm{0},\bm{I}_{m_k})$. Thus, by \cref{lem:Recur_uv}, with probability at least $1-\exp\mpt{-\Omega(s\ln m)}$, we can get
% %   \begin{align*}
% %     &\norm{{\bm{D}}^{(k)\prime}{\bm{W}}^{(k),T}_0\hat{\bm{\gamma}}}_2=\sup_{\bm{u}:\norm{\bm{u}}_2=1}\norm{\bm{u}^T{\bm{D}}^{(k)\prime}{\bm{W}}^{(k),T}_0\hat{\bm{\gamma}}\cdot 1}_2\\
% %     &\quad=\sup_{\bm{u}:\norm{\bm{u}}_2=1}\norm{\pt{{\bm{D}}^{(k)\prime}\bm{u}}^T{\bm{W}}^{(k),T}_0\hat{\bm{\gamma}}\cdot 1}_2 =O\mpt{\sqrt{s\ln m}},
% %   \end{align*}
% %   because of $\norm{\bm{D}^{(k)\prime}\bm{u}}_0\leq\norm{\bm{D}^{(k)\prime}}_0\leq s$ and $\norm{1}_0=1\leq \norm{\bm{D}^{(k)\prime}}_0\leq s$. Combining the above discussions, we can conclude the result.
% % \end{proof}

% % \begin{lemma}
% %   \label{lem:backward perturbation} Fix $\x \in \mc{\X}$ and $s \in [\Omega({d}/{\ln m }), O({m}/{\ln m}) ]$. Then, with probability at least $1-\exp(-\Omega(s\ln m))$ over the random initialization, we have the following result:
% %   Let $\bm{D}^{(r)'}$, $\bm{W}^{(r)'}$, $r = 0,\dots,L$ be perturbation matrices satisfying
% %   % , with probability at least $1-\exp(-\Omega(s\ln m)) $ over the randomness of $\bm{W}^{(l)}_0$ for all $l \in \{0,1,\cdots,L\}$, if the perturbation
% %   \begin{enumerate}
% %     \item each $ \bm{D}^{(r)'}$ is diagonal with at most $s$ non-zero entries;
% %     \item each $\bm{W}^{(r)'}$ satisfies $\norm{\bm{W}^{(r)'}}_2 \leq \tau=O\big({\sqrt{m}}/{(\ln m)^3}\big)$.
% %   \end{enumerate}
% %   Denote $\tilde{\bm{W}}^{(r)}={\bm{W}}^{(r)}+{\bm{W}}^{(r)\prime}$ and $\tilde{\bm{D}}^{(r)}_{\x}={\bm{D}}^{(r)}_{\x}+{\bm{D}}^{(r)\prime}$. Then, we have
% %   \begin{equation*}
% %   \begin{aligned}
% %   \norm{\prod_{r=l}^L \sqrt{\frac2{m_r}}\tilde{\bm{W}}^{(r)} \tilde{\bm{D}}^{(r)}_{\x} -
% %     \prod_{r=l}^L \sqrt{\frac2{m_r}}  \bm{W}^{(r)} \bm{D}_{\x}^{(r)}}_2
% %     \\= O\mpt{\sqrt{\frac{s \ln m +\tau^2}{m}}}
% %   \end{aligned}     
% %   \end{equation*}
% %   for all $l = 0,1,\dots,L$.
% % \end{lemma}
% % \begin{remark}
% %   If we choose the parameter $s$ in \cref{lem:backward perturbation} to be $s \asymp m^{2/3} \tau^{2/3}$, the upper bound  becomes $O(m^{-1/6} \tau^{1/3}\sqrt{\ln m} )$ for $\tau\in\big[\Omega\big(m^{-1}(\ln m)^{-3/2}\big),O\big({\sqrt{m}}/{(\ln m)^3}\big)\big]$.
% % \end{remark}

% In the proof of \cref{lem:backward perturbation}, we use the following result:
% \begin{lemma}
%   \label{lem:Recur_uv}
%   Suppose each entry of matrix $\bm{W} \in \mathbb{R}^{a \times b}$ follows $\bm{W}_{ij} \iid\mathcal{N}(0,1)$. Let $c = \max(a,b)$. If $s\geq 0$, then with probability at least $1- \exp\xkm{-s \ln c}$, the following holds:
%   \[\forall \bm{u} \in \mathbb{R}^a, \bm{v} \in \mathbb{R}^b~\text{such that}~\norm{\bm{u}}_0, \norm{\bm{v}}_0 \leq s,~\text{we have}~
%     \abs{\bm{u}^T \bm{W v}}\leq 9 \sqrt{s \ln c } \norm{\bm{u}}_2 \norm{\bm{v}}_2.\]
% \end{lemma}
% \begin{proof}
%  First of all, it is easy to see that when $s<1 $ or $c=1$, the proposition is trivial. So we only need to consider the result under condition that $s \geq 1$ and $c \geq 2$. 

%  Let $A \subseteq [a]$ such that $\abs{A}= \min\dk{a,\fl{s}}$, and let $U_A = \dk{ \bm{u}\in \mathbb{R}^{\abs{A}} : \forall i \notin A, u_i = 0 } $ be a set that contains vectors of which non-zero entries are only located in $A$. In the same way, let $B \subseteq [b] $ such that $\abs{B}=\min\dk{b, \fl{s}}$, and let $V_B =
%  \dk{ \bm{v}\in \mathbb{R}^{\abs{B}} : \forall j \notin B, v_j = 0 } $. Then we have
%  \begin{equation*}
%  \bm{u}^T \bm{W v} = \sum^{a}_{i=1} \sum^{b}_{j=1} \bm{u}_i \bm{W}_{ij} \bm{v}_j = \sum_{i \in A,j \in B} \bm{u}_i \bm{W}_{ij} \bm{v}_j= \bm{u}^T_A \bm{W}_{AB} \bm{v}_B,
%  \end{equation*}
%  in which $\bm{u}_A = (\bm{u}_i)_{i \in A}^T, \bm{v}_B = (\bm{v}_j)_{j \in B}^T, \bm{W}_{AB} = (\bm{W}_{ij})_{i \in A, j \in B}$.
%  According to the definition of spectral norm, we know that
%  \begin{equation*}
%  \abs{\bm{u}^T \bm{W v}} = \abs{ \bm{u}^T_A \bm{\bm{W}}_{AB} \bm{\bm{v}}_B } \leq \norm{\bm{u}_A}_2 \norm{\bm{W}_{AB}}_2 \norm{\bm{v}_B} _2.
%  \end{equation*}

%  Now we consider the spectral norm of $\bm{W}_{AB} \in \mathbb{R}^{\abs{A} \times \abs{B}}$. When $t \geq \sqrt{\fl{s}}$, with probability at least $1- 2\exp\xkm{-t^2/2}$, we have $\norm{\bm{W}_{AB}}_2 \leq 3t$. Then we have
%  \begin{equation*}
%  \forall \bm{u} \in U_A, \forall \bm{v} \in V_B, \quad \lvert \bm{u}^T \bm{W v} \rvert \leq \norm{\bm{u}}_2 \norm{\bm{W}_{AB}}_2 \norm{\bm{v}}_2 \leq 3t \norm{\bm{u}}_2 \norm{\bm{v}}_2.
%  \end{equation*}

%  Now we consider all possible $A$ and $B$, or to say all possible location of non-zero entries. We know there are $\binom{a}{\abs{A}}$ kinds of $ A$ and $\binom{b}{\abs{B}}$ kinds of $B$ in total. Therefore, with probability at least $1- 2\binom{a}{\abs{A}}\binom{b}{\abs{B}} \exp\xkm{-t^2/2} $, the following proposition holds:
% \begin{equation*}
%  \forall \bm{u} \in \mathbb{R}^a, \forall \bm{v} \in \mathbb{R}^b, s.t. \norm{\bm{u}}_0,\norm{\bm{v}}_0 \leq s, \quad \text{we have } \lvert \bm{u}^T \bm{W v} \rvert \leq 3t \norm{\bm{u}}_2 \norm{\bm{v}}_2.
%  \end{equation*}

% With the trivial inequality $\binom{n}{k} \leq n^k $, we have a control for the probability above:
% \begin{align*}1-2\binom{a}{\abs{A}}\binom{b}{\abs{B}}\exp\xkm{-t^2/2} \geq1-2a^{\abs{A}}b^{\abs{B}}\exp\xkm{- t^2/2}\geq 1-2a^{\fl{s}}b^{\fl{s}}\exp\xkm{- t^2/2}&\\
% \qquad\geq 1-2c^{2s}\exp\xkm{- t^2/2}=1-\exp\xkm{-\xk{t^2/2-2s\log c-\log 2}}&.\end{align*}
% Finally, let $t=\sqrt{8s\log c} \geq \sqrt{s}$, and then we get the expected result.
% \end{proof}

By using \cref{lem:NN_forward_perturbation} and \cref{lem:backward perturbation}, we can prove the following lemma:
% , which is a key lemma in proving \cref{prop:Training_KernelBound}.
\begin{lemma}
  \label{lem:Training_GradientBound}
  Let $\Delta=O(1/\sqrt{m})$, $\tau\in\zk{\Delta\sqrt{m},O\xkm{{\sqrt{m}}/{(\ln m)^3}}}$, $T\subseteq [0,\infty)$ and fix ${\bm{z}} \in\tilde{B}_R$.
  Suppose that $\norm{\bm{W}^{(l)}_t - \bm{W}^{(l)}_0 }_{F}\leq\tau$ holds for all $t\in T$ and $l\in\dk{0,1,\cdots,L}$.
  
  Then there exists a positive absolute constant $C$ such that with probability at least $1-\exp\xkm{-\Omega(m^{2/3}\tau^{2/3})}$, for all $t\in T$, $l\in\dk{0,1,\cdots,L}$ and $\x\in\tilde{B}_R$ such that $\norm{\x-\bm{z}}_2\leq \Delta$, we have
  % Suppose $\sup_{t \geq 0} \norm{\bm{W}^{(l)}_t - \bm{W}^{(l)}_0 }_{\mr{F}}= O(m^{1/4}) $ holds for all $l$.
  % For fixed $\z \in  \tilde{B}_R$, with probability at least $1-\exp(-\Omega(m^{5/6}))$ over the random initialization,
  % for any $\x \in  \mc{\X}$ such that  $\norm{\x - \z}_2 =  O(1/m)  $, we have
  \begin{align*}
    \norm{\nabla_{\bm{W}^{(l)}} g_t(\x) - \nabla_{\bm{W}^{(l)}} g_0(\z)}_{\mr{F}} = O\xkm{R m^{-1/6} \tau^{1/3}\sqrt{\ln m}}~\text{and thus}~\norm{ \nabla_{\bm{W}^{(l)}} g_t(\x)}_{\mr{F}} = O(R).
  \end{align*}
  when $m$ is greater than the positive constant $C$.
\end{lemma}
% \begin{lemma}
%   \label{lem:Training_GradientBound}
%   Let $\tau=O\mpt{{\sqrt{m}}/{(\ln m)^3}}$ and $T\subseteq [0,\infty)$. Suppose that $\norm{\bm{W}^{(l)}_t - \bm{W}^{(l)}_0 }_{\mr{F}}\leq\tau$ holds for all $t\in T$ and $l\in\cl{0,1,\cdots,L}$. Then there exists a positive absolute constant $C$ such that for any fixed $\bm{z}\in\tilde{B}_R$, with probability at least $1-\exp\mpt{-\Omega(m^{2/3}\tau^{2/3})}$, for all $l\in\cl{0,1,\cdots,L}$, we have
%   \begin{align*}
%     \sup_{t\in T}\norm{\nabla_{\bm{W}^{(l)}} g_t(\z) - \nabla_{\bm{W}^{(l)}} g_0(\z)}_{\mr{F}}= O\mpt{m^{-1/6} \tau^{1/3} \sqrt{\ln m}}
%   \end{align*}
%   when $m$ is greater than the positive constant $C$.
% \end{lemma}
\begin{proof}
  % [Proof of \cref{lem:Training_GradientBound}]
  Recalling \cref{eq:GradientExpr}, we have
  $\nabla_{\bm{W}^{(l)}} g_t(\x)=\bm{\gamma}^{(l)}_{\x,t}\
    {\bm{\alpha}}^{(l),T}_{\x,t}$.
  Then, we have
  \begin{align*}
    &\norm{\nabla_{\bm{W}^{(l)}} g_t(\x) - \nabla_{\bm{W}^{(l)}} g_0(\z)}_{\mr{F}}=\norm{\bm{\gamma}^{(l)}_{\x,t}{\bm{\alpha}}^{(l),T}_{\x,t} -\bm{\gamma}^{(l)}_{\z,0}{\bm{\alpha}}^{(l),T}_{\z,0}}_{\mr{F}}\notag \\
    % \label{eq:Proof_Training_GradientBound_Decomp}
    &\qquad \leq \norm{\bm{\gamma}^{(l)}_{\x,t} - \bm{\gamma}^{(l)}_{\z,0}}_2\norm{{\bm{\alpha}}^{(l)}_{\x,t}}_2
    + \norm{\bm{\gamma}^{(l)}_{\z,0}}_2\norm{{\bm{\alpha}}^{(l)}_{\x,t} - {\bm{\alpha}}^{(l)}_{\z,0}}_2 \leq O\xkm{R m^{-1/6} \tau^{1/3}\sqrt{\ln m}}
  \end{align*}
  by \cref{lem:backward perturbation}, \cref{lem:NN_forward_perturbation} and \cref{lem:Init_WeightsProductBound}.

%   According to \cref{lem:Init_WeightsProductBound}, with probability at least $1-\exp(-\Omega(m))$, we can get
%   \[\norm{\bm{\gamma}^{(l)}_{\z,0}}_2 = O(1)\qquad\text{and}\qquad\norm{\tilde{\bm{\alpha}}_{\z,0}^{(l)}}_2 = O(1).\]
%   On the other hand, by \cref{lem:NN_forward_perturbation}, with probability at least $1-\exp\mpt{-\Omega(m^{2/3}\tau^{2/3})}$, we have
%   $\norm{\tilde{\bm{\alpha}}^{(l)}_{\x,t} - \tilde{\bm{\alpha}}_{\z,0}^{(l)}}_2 = O({\tau }/{\sqrt{m}})$, which implies that (since $\tau = O\big({\sqrt{m}}/{(\ln m)^3}\big)$)
%   \begin{align*}
%     \norm{\tilde{\bm{\alpha}}_{\x,t}^{(l)}}_2
%     \leq \norm{\tilde{\bm{\alpha}}_{\z,0}^{(l)}}_2 + \norm{\tilde{\bm{\alpha}}^{(l)}_{\x,t} - \tilde{\bm{\alpha}}_{\z,0}^{(l)}}_2 = O(1).
%   \end{align*}
% %  since $O\big({\sqrt{m}}/{(\ln m)^3}\big)$,
%   Moreover, the statement $(b)$ in \cref{lem:NN_forward_perturbation} also enable us to
%   apply \cref{lem:backward perturbation} for $\bm{W}^{(r)'} = \bm{W}_t^{(r)} - \bm{W}_0^{(r)}$,
%   $\bm{D}^{(r)'} =  \bm{D}^{(r)}_{\x,t} -  \bm{D}^{(r)}_{\z,0}$, yielding that with probability at least $1-\exp\mpt{-\Omega(m^{2/3}\tau^{2/3})}$, we have
%   \begin{align*}
%     \norm{{\bm{\gamma}}^{(l)}_{\x,t} - {\bm{\gamma}}^{(l)}_{\z,0}}_2=\norm{\tilde{\bm{\gamma}}^{(l)}_{\z} - {\bm{\gamma}}^{(l)}_{\z,0}}_2
%     = O\mpt{m^{-1/6} \tau^{1/3}\sqrt{\ln m}}.
%   \end{align*}
%   Plugging these bounds in \cref{eq:Proof_Training_GradientBound_Decomp}, we can get the result.
  % \begin{align*}
  %   &\norm{\nabla_{\bm{W}^{(l)}} g_t(\z) - \nabla_{\bm{W}^{(l)}} g_0(\z)}_{\mr{F}}
  %   = O\mpt{m^{-{1}/{6}}\tau^{1/3}  \sqrt{\ln m}} + O\Big(\frac{\tau\sqrt{\ln m} }{\sqrt{m}}\Big) \\&=
  %   O(\tau^\frac{1}{3} m^{-\frac{1}{6}} \sqrt{\ln m}).
  % \end{align*}
\end{proof}

After preparing these tools, now we are ready to give the proof of \cref{prop:Training_KernelBound}.
% \begin{proof}

\paragraph{Proof of \cref{prop:Training_KernelBound}}
By applying \cref{lem:Training_GradientBound} with $\Delta=O(1/m)\leq O(1/\sqrt{m})$ and $\tau \asymp m^{1/4}\geq \Delta\sqrt{m}$, with probability at least $1 - \exp(-\Omega(m^{5/6}))$, for all $\x\in\tilde{B}_R$ such that $\norm{\x-\z}_2\leq O(1/m)$, we can obtain the following result
\begin{align*}
  \norm{ \nabla_{\bm{W}^{(l)}} g_t(\x)- \nabla_{\bm{W}^{(l)}} g_0(\z) }_{\mr{F}}
  = O\xkm{ R m^{-1/12}\sqrt{ \ln m}}~\text{and}~\norm{ \nabla_{\bm{W}^{(l)}} g_t(\x)}_{\mr{F}} = O(R).
\end{align*}
The same conclusion holds if we replace $\x$ and $\z$ with $\x'$ and $\z'$.
% Furthermore, since
% \begin{align*}
%   K^{(p)}_t(\x,\x')
%   &= \ang{\nabla_{\bm{\theta}^{(p)}} g_t(\x),\nabla_{\bm{\theta}^{(p)}} g_t(\x')}
%   = \sum_{l=0}^L \ang{\nabla_{\bm{W}^{(l)}} g_t(\x), \nabla_{\bm{W}^{(l)}} g_t(\x') } + 1,
% \end{align*}
Thus, we have
\begin{align*}
  &\abs{K^{(p)}_t(\x,\x') - K^{(p)}_0(\z,\z')}\leq \sum_{l=0}^L\abs{\ang{\nabla_{\bm{W}^{(l)}} g_t(\x), \nabla_{\bm{W}^{(l)}} g_t(\x') } - \ang{\nabla_{\bm{W}^{(l)}} g_0(\z), \nabla_{\bm{W}^{(l)}} g_0(\z') }} \\
  \begin{split}
    &\qquad\qquad\qquad \leq \sum_{l=0}^L \Big[\norm{ \nabla_{\bm{W}^{(l)}} g_t(\x)}_{\mr{F}} \norm{ \nabla_{\bm{W}^{(l)}} g_t(\x')- \nabla_{\bm{W}^{(l)}} g_0(\z') }_{\mr{F}}\\
    &\qquad\qquad\qquad\qquad\,\, + \norm{ \nabla_{\bm{W}^{(l)}} g_t(\x)- \nabla_{\bm{W}^{(l)}} g_0(\z) }_{\mr{F}}\norm{ \nabla_{\bm{W}^{(l)}} g_0(\z')}_{\mr{F}}\Big]
  \end{split}
  = O\xkm{R m^{-1/12}\sqrt{ \ln m}}
\end{align*}
holds for all $t\in T$ with probability at least $1 - \exp\xkm{-\Omega(m^{5/6})}$ when $m$ is large enough.
% \end{proof}


% Let $\tau = (\ln m)^\frac{3}{2}$, we have
% \[\norm{\nabla_{\bm{W}^{(l)}} g^{(p)}_t(\x) - \nabla_{\bm{W}^{(l)}} g^{(p)}_0(\x)}_{\mr{F}} \leq O\left( m^{-\frac{1}{6}}  \ln m \right).\]

\subsection{Lazy regime}\label{subsec:Lazy}
% Via continuous time gradient descent, the evolution of the parameters $\bm{\theta}$ can be written as
% \begin{align*}
%     \frac{\dd{\bm{\theta}}_t}{\dd t}&=-\nabla_{{\bm{\theta}}}\mathcal{L}(f,\mc{S})=-\nabla_{{\bm{\theta}}}f({\X})^T\nabla_{f({\X})}\mathcal{L}(f,\mc{S})= -\frac1n\nabla_{{\bm{\theta}}} f({\X})^T (f({\X})-{Y}).
% \end{align*}

In this subsection, we will prove that during the process of gradient descent training, the parameters do not change much.
Therefore, the conditions for the lemmas stated in the previous subsection are satisfied.
Since training relies on the structure of the neural network, in this subsection, we will no longer omit the superscript $p$ (although the corresponding conclusions also hold for non-mirror neural networks).


Let $\lambda_0 = \lambda_{\min} (\NTK(\X,\X))$ and $\bm{u}(t)=f_t(\X)-\bm{y}$.
Denote $\tilde{M}_{\bm{X}} = \sum_{i=1}^n\norm{\tilde{\x_i}}_2 $.
Since we will show the positive definiteness of the NTK in \cref{prop:NTK_PD}, we can assume that $\lambda_0 > 0$ hereafter.
Similar to Lemma F.8 and Lemma F.7 in \citet{arora2019_ExactComputation}, we have the following lemmas:


\begin{lemma}
  \label{lem:GradientFlow_ExpDecay}
  Let $\delta\in(0,1)$ and $t\in[0,\infty)$.
  Suppose that $\norm{\bm{W}^{(l,p)}_s - \bm{W}^{(l,p)}_0}_{\mr{F}} = O(m^{1/4})$ holds for all $s \in [0,t] $, $l\in \dk{0,1,\cdots,L}$ and $p\in\dk{1,2}$.
  Then there exists a polynomial $\poly(\cdot)$ such that when $m\geq\poly\xkm{n,\lambda_0^{-1},\ln(1/\delta)}$,
% Then there exist some positive absolute constants $C_1>0$ and $C_2\geq 1$ such that when $m\geq C_1\bk{\pt{n^{2}\lambda_0^{-1}}^{15}+\pt{\ln(C_2n^2/\delta)}^5}$, 
  with probability at least $1-\delta$, we have
  \begin{align}
    \label{eq:GradientFlow_ExpDecay}\norm{\bm{u}(s)}^{2}\leq \exp\xkm{- \frac{\lambda_{0}}{n}s}\norm{\bm{u}(0)}^{2} = \exp\xkm{-\frac{\lambda_{0}}{n}s} \|\bm{y}\|^{2},\quad~\text{for all}~s\in [0,t].
  \end{align}
  % Suppose that $\sup_{s \in T} \norm{\bm{W}^{(l,p)}_s - \bm{W}^{(l,p)}_0}_{\mathrm{F}} = O(m^{1/4})$ holds for all $l$ and $p$.
  % Then, for any $\delta \in (0,1)$, if $m \geq \mathrm{poly}\big(n,\lambda_0^{-1},\ln(1/\delta)\big)$, with probability at least $1-\delta$, we have
  % \begin{align}
  %   \label{eq:GradientFlow_ExpDecay}
  %   \norm{\bm{u}(s)}^2 \leq \exp(-\frac{\lambda_{0}}{n}s)\norm{\bm{u}(0)}^{2} = \exp({-\frac{\lambda_{0}}{n}s}) \norm{\bm{y}}^2,\quad
  %   \forall s \in T.
  % \end{align}
\end{lemma}

\begin{proof}
  Denote $\tilde{\lambda}_0(s)=\lambda_{\min}\big(K_s(\X,\X)\big)$.
  By Weyl's inequality, we can get
  \begin{align*}
    \abs{\tilde{\lambda}_0(s)-\lambda_0} \leq \norm{ K_s(\X,\X) - \NTK(\X,\X) }_2\leq \norm{ K_s(\X,\X) - \NTK(\X,\X) }_{\mr{F}}&\\
    \leq\frac12\sum_{p=1}^2\sum^n_{i,j=1} \abs{K^{(p)}_s(\x_i,\x_j) - \NTK(\x_i,\x_j)}&.
  \end{align*}
  Applying \cref{cor:Training_Kernel_uniform} with $\delta'= {\delta}/{(2 n^2)}$ to each difference,
  with probability at least $1-2n^2\delta'=1-\delta$, we can obtain the following bound for all $s\in [0,t]$:
  \begin{align*}
    \abs{\tilde{\lambda}_0(s)-\lambda_0}\leq n^2 O\xkm{m^{-1/12} \sqrt{\ln m}}\leq n^2 O\xkm{m^{-1/15}}\leq\frac{\lambda_0}2,
  \end{align*}
  when $m\geq C_1\zk{\xk{n^{2}\lambda_0^{-1}}^{15}+\xk{\ln\xkm{C_2n^2/\delta}}^5}$ for some positive absolute constants $C_1>0$ and $C_2\geq 1$.
  This implies that $\tilde{\lambda}_0(s)\geq\lambda_0/2$ holds for all $s\in [0,t]$.
  Therefore, we have
  \begin{equation*}
    \frac{\dd}{\dd s}\|\bm{u}(s)\|^{2}=- \frac{2}{n}\bm{u}(s)^T K_{s}(\X,\X) \bm{u}(s)\le -\frac{\lambda_{0}}{n} \lVert \bm{u}(s)\rVert^{2},
  \end{equation*}
  which implies \cref{eq:GradientFlow_ExpDecay} by standard ODE theory.
  Finally, by choosing
  \begin{align*}
    \poly\xkm{n,\lambda_0^{-1},\ln(1/\delta)}=C_1\zkm{\xk{n^{2}\lambda_0^{-1}}^{15}+\xk{2n+\ln(1/\delta)+\ln C_2}^5}
  \end{align*}
  we can complete the proof of this lemma.
\end{proof}


% Similar to Lemma F.7 in \citet{arora2019_ExactComputation}, we have the following lemma:



\begin{lemma}
  \label{lem:A_lazy_W}
  Fix $l \in \{0,1,\cdots,L\}$, $p \in \{1,2\}$ and let $\delta\in(0,1)$, $t\in[0,\infty)$.
  Suppose that for $s \in [0,t]$, we have
  \begin{align*}
    \begin{gathered}
      \norm{f_s(\X)-\bm{y}}_2 \leq \exp(\frac{-\lambda_0 }{4n}s) \norm{\bm{y}}_2\quad~\text{and}\\
      \norm{ \bm{W}^{(l',p')}_s - \bm{W}^{(l',p')}_0 }_{\mr{F}} \leq \frac{\sqrt{m}}{(\ln m)^3},\qquad~\text{for all}~(l',p') \neq (l,p).
    \end{gathered}
  \end{align*}
  Then there exists a polynomial $\poly(\cdot)$ such that when $m\geq\poly\xkm{n, \tilde{M}_{\bm{X}}, \norm{\bm{y}}_2,\lambda_0^{-1},\ln(1/\delta)}$,
  % Then there exists a positive absolute constant $C>0$ such that when $m\geq C\bk{\pt{n\norm{\bm{y}}_2\lambda_0^{-1}}^5+\pt{\ln\mpt{n/\delta}}^{6/5}}$,
  with probability at least $1-\delta$, we have
  $\sup_{s\in[0,t]}\norm{\bm{W}^{(l,p)}_s-\bm{W}^{(l,p)}_0 }_{\mr{F}} = O\xkm{{n \tilde{M}_{\bm{X}}\norm{\bm{y}}_2}/{\lambda_0}}$.
\end{lemma}

\begin{proof}
  First of all, recalling \cref{eq:2_GD} we have
  \begin{align}
    \notag
    \norm{\bm{W}^{(l,p)}_{t_0} - \bm{W}^{(l,p)}_0}_{\mr{F}}&= \norm{\int^{t_0}_0 {\dd \bm{W}^{(l,p)}_s}}_{\mr{F}} \leq  \int^{t_0}_0 \norm{\frac{1}{n\sqrt{2}}\sum^n_{i=1} ( f_s(\x_i) - y_i) \nabla_{\bm{W}^{(l,p)}} g^{(p)}_s(\x_i)}_{\mr{F}}\dd s \\
    \notag
    &\leq \frac{1}{n\sqrt{2}} \sum^n_{i=1} \sup_{0\leq s \leq t_0}\norm{\nabla_{\bm{W}^{(l,p)}} g_s^{(p)}(\x_i)}_{\mr{F}} \int^{t_0}_0 \norm{f_s(\X)-\bm{y}}_2~\dd s \\
    \label{eq:Proof_Lazy_GradientControl0}
    &\leq  O\xkm{\frac{\norm{\bm{y}}}{\lambda_0}} \sum^n_{i=1}\sup_{0\leq s \leq t_0} \norm{\nabla_{\bm{W}^{(l,p)}} g_s^{(p)}(\x_i)}_{\mr{F}} .
  \end{align}
  for all $t_0\in[0,t]$.
  % By triangle inequality, we have
  % \begin{equation}
  %   \label{eq:Proof_Lazy_GradientControl1}
  %   \begin{aligned}
  %     \norm{\nabla_{\bm{W}^{(l,p)}} g_s^{(p)}(\x_i) }_{\mr{F}} &\leq \norm{\nabla_{\bm{W}^{(l,p)}} g_0^{(p)}(\x_i) }_{\mr{F}} + \norm{\nabla_{\bm{W}^{(l,p)}} g_s^{(p)}(\x_i) - \nabla_{\bm{W}^{(l,p)}} g_0^{(p)}(\x_i) }_{\mr{F}} .
  %   \end{aligned}
  % \end{equation}
  % For the first term in \cref{eq:Proof_Lazy_GradientControl1}, by \cref{lem:Init_WeightsProductBound}, with probability at least $1-\exp\big({-\Omega(m)}\big)$, we have
  % $\norm{\nabla_{ \bm{W}^{(l,p)}} g_0^{(p)}(\x_i) }_{\mr{F}} = O(1)$ for any $i\in[n]$. So it suffices to bound the second term in \cref{eq:Proof_Lazy_GradientControl1}. 
  Suppose that \[s_0 = \min\dkm{ s \in [0,t]:\norm{\bm{W}^{(l,p)}_s- \bm{W}^{(l,p)}_0}_{\mr{F}} \geq  {\sqrt{m}}/{(\ln m)^3}}\] exists,
%  If $T_0 \neq T$, set $t_0$
  then $\norm{\bm{W}^{(l',p')}_s - \bm{W}^{(l',p')}_0}_{\mr{F}} \leq {\sqrt{m}}/{(\ln m)^3}$ holds for all $s \in [0,s_0] $, $l'\in \dk{0,1,\cdots,L}$ and $p'\in\dk{1,2}$.
  Applying \cref{lem:Training_GradientBound} with $\Delta=0$ and $\tau={\sqrt{m}}/{(\ln m)^3}$, we know that for any $i\in[n]$, with probability at least $1-\exp\xkm{-\Omega\xkm{m/(\ln m)^{2}}}\geq 1-\exp\xkm{-\Omega(m^{5/6})}$, we have
  \begin{equation*}
    \sum_{i=1}^n\sup_{s \in [0,s_0]} \norm{\nabla_{ \bm{W}^{(l,p)}} g_s^{(p)}(\x_i)}_{\mr{F}} = O(\tilde{M}_{\bm{X}}).
  \end{equation*}
  Plugging it back to \cref{eq:Proof_Lazy_GradientControl0}, with probability at least $1-n\exp\xkm{-\Omega(m^{5/6})}$, we obtain
  \[\norm{\bm{W}^{(l,p)}_{s_0} - \bm{W}^{(l,p)}_0}_{\mr{F}} =  O({n \tilde{M}_{\bm{X}} \norm{\bm{y}}}/{\lambda_0}),\]
  which contradicts to $\norm{\bm{W}^{(l,p)}_{s_0}- \bm{W}^{(l,p)}_0}_{\mr{F}}  \geq {\sqrt{m}}/{(\ln m)^3}$ when $m\geq C_1\xk{n\tilde{M}_{\bm{X}}\norm{\bm{y}}_2\lambda_0^{-1}}^5$ for some positive constant $C_1$.
%  Here we have to note that the hidden constants in

  Now we show that $\norm{\bm{W}^{(l',p')}_s- \bm{W}^{(l',p')}_0}_{\mr{F}} \leq {\sqrt{m}}/{(\ln m)^3}$ holds for all $s \in [0,t]$ and any $(l',p')$.
  The desired bound then follows from applying \cref{lem:Training_GradientBound} again for the interval $[0,t]$.

  Also, it is easy to check that there exists a positive absolute constant $C$ such that when $m
  \geq C_2\ln(n/\delta)^{6/5}$, we have $1-n\exp\xkm{-\Omega(m^{5/6})}\geq 1-\delta$.
  Finally, by choosing
  \[\poly\xkm{n,\lambda_0^{-1},\ln(1/\delta)}=C\zk{\xk{n\tilde{M}_{\bm{X}}\norm{\bm{y}}_2\lambda_0^{-1}}^{5}+\xk{n+\ln(1/\delta)}^2+1}\]
  for some positive absolute constant $C>0$, we can complete the proof.
\end{proof}


The following lemma is the key lemma that we aim to prove in this subsection.
It serves as the prerequisite for establishing the conclusions of the lemmas in the preceding and subsequent subsections.

\begin{lemma}
[Lazy regime]
  \label{lem:A_lazy_regime}
  There exists a polynomial $\poly(\cdot)$ such that for any $\delta\in(0,1)$, with probability at least $1-\delta$, for all $p\in\dk{1,2}$ and $l\in\dk{0,1,\cdots,L}$, we have
  \begin{align*}
    \sup_{t \geq 0}\norm{\bm{W}^{(l,p)}_t - \bm{W}^{(l,p)}_0 }_{\mr{F}} =O(m^{1/4}).
  \end{align*}
  when $m\geq\poly\xkm{n,\tilde{M}_{\bm{X}},\norm{\bm{y}}_2,\lambda_0^{-1},\ln(1/\delta)}$.
\end{lemma}

\begin{proof}
  Let us assume that
  \begin{align*}
    t_0 = \min\dkm{t\geq 0:\exists l,p~\text{such that}~
    \norm{\bm{W}^{(l,p)}_t - \bm{W}^{(l,p)}_0}_{\mr{F}} \geq m^{1/4}~\text{or}~\norm{\bm{u}(t)} \geq \exp(\tfrac{-\lambda_0}{4n}t) \norm{\bm{y}}}
  \end{align*}
  exists.
  Then, for all $t \in  [0,t_0]$, we have
  \begin{align*}
    \begin{gathered}
      \norm{\bm{u}(t)} \leq \exp(\frac{-\lambda_0}{4n}t) \norm{\bm{y}}\qquad~\text{and}~\qquad
      \norm{\bm{W}^{(l,p)}_t - \bm{W}^{(l,p)}_0}_{\mr{F}} \leq m^{1/4}~\text{for all}~l, p.
    \end{gathered}
  \end{align*}
  According to \cref{lem:A_lazy_W} and \cref{lem:GradientFlow_ExpDecay}, we know that there exists a polynomial $\poly(\cdot)$ such that with probability at least $1-\delta$, we have
  \begin{align*}
    \begin{gathered}
      \norm{\bm{u}(t_0)} \leq \exp(\frac{-\lambda_{0}}{2n}t_0)\norm{\bm{y}}\qquad~\text{and}~\qquad
      \norm{\bm{W}^{(l,p)}_{t_0}-\bm{W}^{(l,p)}_0 }_{\mr{F}} = O\xkm{\frac{n\tilde{M}_{\bm{X}}\norm{\bm{y}}}{\lambda_0}}~\text{for all}~l, p
    \end{gathered}
  \end{align*}
  when $m\geq \poly\xkm{n,\norm{\bm{y}}_2,\lambda_0^{-1},\ln(1/\delta)}$,
  which contradicts to the definition of $t_0$ when $m\geq C\xkm{n\tilde{M}_{\bm{X}}\norm{\bm{y}}_2\lambda_0^{-1}}^5$ for some positive absolute constant $C>0$.
\end{proof}
We also have a simple corollary:
\begin{corollary}
  \label{cor:UpperboundNN}
  There exists a positive absolute constant $M$ and $C$ such that when $m \geq M$,
  with probability at least $1-\exp(-\Omega(m^{5/6}))$,
  \begin{align*}
    \abs{f_t^m(x)} \leq  C \norm{\tilde{\x}}, \quad \forall x \in \R^d,~ \forall t \geq 0.
  \end{align*}
\end{corollary}
\begin{proof}
  Recall that
  \begin{align*}
    f_t^m(x) = \frac{\sqrt{2}}{2}[\bm{W}^{(L,1)}_t\bm{\alpha}_t^{(L,1)}(x) - \bm{W}^{(L,2)}_t\bm{\alpha}_t^{(L,2)}(x) ].
  \end{align*}
  Since with probability at least $1-\exp(-\Omega(m^{5/6}))$ we have
  \begin{align*}
    \norm{\bm{W}_0^{(l,p)}}_2\leq O(\sqrt{m}),\quad \norm{\bm{D}_{x,t}^{(l,p)}}_2 \leq 1,\quad \sup_{t\geq 0}\norm{\bm{W}_t^{(l,p)} - \bm{W}_0^{(l,p)}}_2 \leq O(m^\frac{1}{4}),
  \end{align*}
  the corollary is proved by \cref{lem:Random matrix} and \cref{lem:A_lazy_regime}
\end{proof}
% Suppose at time $t_0$, we have $\norm{\bm{W}^{(l,q)}_{t_0} - \bm{W}^{(l,q)}_0}_{\mr{F}} \geq \tau$ holds for some $l$ and $p$. 

% Then according to \cref{lem:A_lazy_W}, with...... there exists $t\in[0,t_0)$ such that $\norm{ \bm{W}^{(l',p)}_t- \bm{W}^{(l',p)}_0}_{\mr{F}} > \frac{\sqrt{m}}{(\ln m)^3}$ for some $l' \neq l$ or $\norm{f_t(\X) - Y}_2 >\me^{-\frac{\lambda_0}{2n}t} \norm{Y}_2$.

%  Suppose that for all $t\geq 0$, we have $\norm{f_t(\X) - Y}_2 \leq \me^{-\frac{\lambda_0}{2n}t} \norm{Y}_2\leq\sqrt{n}\me^{-\frac{\lambda_0}{2n}t}C_y$ and $\norm{ \bm{W}^{(l',p)}(t) - \bm{W}^{(l',p)}(0) }_{\mr{F}} \leq \frac{\sqrt{m}}{(\ln m)^3}$ for all $l' \neq l$.


%\begin{remark}
%  With mirror initialization, We know that $f(\x) = \frac{1}{2}[g^{(1)}(\x;\bm{\theta}) - g^{(2)}(\x;\bm{\theta})]$ and
%  \begin{align*}
%    &K_t(\x,\x') =\sum_{p=1}^2 \sum^{L}_{l=0} \langle \nabla_{\bm{W}^{(l,p)}} f_t(\x) , \nabla_{\bm{W}^{(l,p)}} f_t(\x') \rangle + 1 \\
%    &\quad= \frac{1}{2} \sum^2_{p=1} \Big(\sum^{L}_{l=0}\langle \nabla_{\bm{W}^{(l,p)}} g^{(p)}_t(\x) , \nabla_{\bm{W}^{(l,p)}} g^{(p)}_t(\x')  \rangle +1 \Big) =  \frac{1}{2} \sum^2_{p=1} K_t^{(p)}(\x,\x').
%  \end{align*}
%  With the simple decomposition above, we can get the property of $K_t$ from the property of $K_t^{(p)}$ (namely the fully-connected network without mirror initialization). So in the following several section, we only consider one single network $g^{(p)}$ and we write $\bm{W}_t^{(l,p)}$, $\bm{\alpha}^{(l,p)}$, $\tilde{\bm{\alpha}}^{(l,p)}$, $\bm{D}^{(l,p)}$ as $\bm{W}_t^{(l)}$, $\bm{\alpha}^{(l)}$, $\tilde{\bm{\alpha}}^{(l)}$, $\bm{D}^{(l)}$ respectively.
%
%\end{remark}

% \subsection{Nearly Hölder Continuity of \texorpdfstring{$K_t$}{Kt}}\label{subsec:Holder_Kt}

% Since $K_t = (K^{(1)}_t+K^{(2)}_t )/2$, to show the continuity of $K_t$, it suffices to show it for $K^{(p)}_t$.
% For convenience, we will focus on one part and omit the superscript $p$ in this subsection.

% \begin{proposition}
%   \label{prop:Continuity_Kt}
%   % For any $\delta\in(0,1)$, suppose $\norm{\bm{W}^{(l)}_t - \bm{W}^{(l)}_0 }_{\mr{F}}= O(m^{1/4})$ holds for all $t\in T$.
%   % Then for any $\x,\x' \in \mc{\X}$, with probability at least $1 - \exp(-\Omega(m^{5/6}))$, we have
%   % \begin{align}
%   %   \sup_{t\in T}\abs{K^{(p)}_t(\x,\x') - K^{(p)}_0(\x,\x')} = O \left( m^{-1/12}\sqrt{\ln m}\right).
%   % \end{align}

%   Fix $ \bm{z},\bm{z}' \in \tilde{B}_R$ and $T\subseteq[0,\infty)$. Suppose that $\norm{\bm{W}^{(l)}_t - \bm{W}^{(l)}_0}_{\mr{F}}= O(m^{1/4})$ holds for all $t\in T$ and $l\in\cl{0,1,\cdots,L}$. Then there exists a positive absolute constant $C$ such that with probability at least $1 - \exp\mpt{-\Omega(m^{5/6})}$, for any $\x,\x' \in \mc{\X}$ such that $\norm{\x-\z}_2,\norm{\x'-\z'}_2 \leq O( 1/m)$, we have
%   \[\sup_{t\in T}\abs{K^{(p)}_t(\x,\x') - K^{(p)}_0(\z,\z')}= O \mpt{m^{-{1}/{12}}\sqrt{\ln m}}\]
%   when $m$ is greater than the positive constant $C$.
% \end{proposition}
% The proof of this proposition is similar to the proof of \cref{prop:Training_KernelBound}. We will also include its proof at the end of this subsection. Combining this proposition with \cref{cor:Init_Kt_NTK}, we can derive the following corollary:
% \begin{corollary}
%   \label{cor:Training_Kernel_uniform}
%   Fix $ \bm{z},\bm{z}' \in \tilde{B}_R$ and let $\delta\in(0,1)$, $T\subseteq[0,\infty)$. Suppose that $\norm{\bm{W}^{(l)}_t - \bm{W}^{(l)}_0}_{\mr{F}}= O(m^{1/4})$ holds for all $t\in T$ and $l\in\cl{0,1,\cdots,L}$. Then there exist some positive absolute constants $C_1>0$ and $C_2\geq 1$ such that with probability at least $1 - \delta$, for any $\x,\x' \in \mc{\X}$ such that $\norm{\x-\z}_2,\norm{\x'-\z'}_2 \leq O( 1/m)$, we have
%   \[\sup_{t\in T}\abs{K^{(p)}_t(\x,\x') - \NTK(\z,\z')}= O \mpt{m^{-{1}/{12}}\sqrt{\ln m}}\]
%   when $m\geq C_1\pt{\ln(C_2/\delta)}^5$.
% \end{corollary}


% To prove \cref{prop:Continuity_Kt}, we need to introduce several necessary lemmas. These lemmas can be seen as generalizations of the lemmas in \cref{subsec:Training}. The main difference compared to the previous lemmas is that now we also need to consider the impact of perturbations in the input points.
% \begin{proposition} 
%   \label{prop:Continuity_Kt}
%   Let $\delta \in (0,1)$.
%   Suppose $\sup_{t \geq 0} \norm{\bm{W}^{(l)}_t - \bm{W}^{(l)}_0 }_{\mr{F}} =O(m^{1/4})$ holds for all $l$.
%   Fix $\z,\z' \in \mc{\X}$.
%   Then, with probability at least $1- \exp(-\Omega(m^{5/6}))$ over initialization,
%   for any $\x,\x' \in \mc{\X}$ such that $\norm{\x-\z}_2,\norm{\x'-\z'}_2 \leq O( 1/m)$,  we have
%   \begin{align}
%       \sup_{t\geq 0}\abs{ K^{(p)}_t(\x,\x') - K^{(p)}_t(\z,\z') } = O\left( m^{-1/12}  \sqrt{\ln m}\right).
%   \end{align}
% \end{proposition}

% 
% In fact, the slight perturbation between $\x$ and $\z$ can be regarded as taking a slight perturbation on $\bm{W}^{(1)}$, with other $\bm{W}^{(l)} $ fixed.
% Inspired by \citet[Lemma 8.2]{allen-zhu2019_ConvergenceTheory} (see \cref{lem:NN_forward_perturbation}), we can prove the following lemma:

% \begin{lemma}
%   \label{lem:NN_forward_perturbation2}
%   Let $\tau\in\bk{\Omega\mpt{1/\sqrt{m}},O\mpt{{\sqrt{m}}/{(\ln m)^3}}}$, $T\subseteq [0,\infty)$ and fix ${\bm{z}} \in\tilde{B}_R$. Suppose that $\norm{\bm{W}^{(l)}_t - \bm{W}^{(l)}_0 }_{\mr{F}}\leq\tau$ holds for all $t\in T$ and $l\in\cl{0,1\cdots,L}$. Then there exists a positive absolute constant $C$ such that with probability at least $1-\exp\mpt{-\Omega(m^{2/3}\tau^{2/3})}$, for all $t\in T$, $l\in\cl{0,1,\cdots,L}$ and $\x \in\tilde{B}_R$ such that $\norm{\x-\bm{z}}_2 \leq O(1/m)$, we have
%   \begin{enumerate}[$(a)$~]
% % \item $\norm{\bm{\alpha}^{(l)}_t(\x) - \bm{\alpha}^{(l)}_0({\bm{z}}) }_2= O\big(\tau/\sqrt{m}\big)$;
%     \item $\norm{\bm{D}_{\x,t}^{(l)} - \bm{D}_{{\bm{z}},0}^{(l)}}_0 = O(m^{2/3} \tau^{2/3})$;
%     \item $\norm{\pt{\bm{D}_{\x,t}^{(l)} - \bm{D}_{{\bm{z}},0}^{(l)}}\bm{\alpha}_{\x,t}^{(l)}}_2= O({\tau}/{\sqrt{m}})$;
%     \item $\norm{\tilde{\bm{\alpha}}^{(l)}_{\x,t} - \tilde{\bm{\alpha}}_{\z,0}^{(l)}}_2 = O(\tau/\sqrt{m})$,
%   \end{enumerate}
%   when $m$ is greater than the positive constant $C$.
% \end{lemma}
% \begin{proof}
%   We use mathematical induction to prove this lemma. When $l=0$, it can be easily verified that $\bs{D}_{\bs{x},t}^{(0)} - \bs{D}_{\bs{z},0}^{(0)} = \bm{O}$ and
%   \[\norm{{\tilde{\bm{\alpha}}}_{\x,t}^{(0)} -{\tilde{\bm{\alpha}}}^{(0)}_{\bm{z},0}}_2=\norm{\tilde{\x}-\tilde{\z}}_2 = \norm{\x-{\bm{z}}}_2 \leq O(1/m),\] where $\bm{O}$ represents the zero matrix, $\tilde{\x}$ and $\tilde{\bm{z}}$ are extended vectors with an additional coordinate of $1$. Thus, all the statements hold for $l=0$. Now we assume that this lemma holds for $l=k\in\cl{0,1,\cdots,L-1}$.

%   First of all, we can decompose $\bm{\alpha}^{(k+1)}_{\x,t} -\bm{\alpha}^{(k+1)}_{\z,0}$ as following:
%   \begin{equation*}
%     \begin{aligned}
%       &\bm{\alpha}^{(k+1)}_{\x,t} -\bm{\alpha}^{(k+1)}_{\z,0}=
%       \sqrt{\frac{2}{m_{k+1}}} \bm{W}^{(k)}_t \tilde{\bm{\alpha}}^{(k)}_{\x,t} - \sqrt{\frac{2}{m_{k+1}}} \bm{W}_0^{(k)}\tilde{\bm{\alpha}}^{(k)}_{\z,0}\\
%       &\quad=\sqrt{\frac{2}{m_{k+1}}} {\Delta\bm{W}^{(k)}} \tilde{\bm{\alpha}}^{(k)}_{\x,t} + \sqrt{\frac{2}{m_{k+1}}}\bm{W}^{(k)}_0 \pt{\tilde{\bm{\alpha}}^{(k)}_{\x,t} - \tilde{\bm{\alpha}}^{(k)}_{\z,0}},
%     \end{aligned}
%   \end{equation*}
%   where $\Delta\bm{W}^{(k)} = \bm{W}^{(k)}_t - \bm{W}^{(k)}_0$.
%   From the above equation, we can deduce that
%   $\norm{\bm{\alpha}^{(k+1)}_{\x,t} -\bm{\alpha}^{(k+1)}_{\z,0}}_2 \leq O\mpt{{\tau}/{\sqrt{m}}}$.

%   Next, we use \cref{lem:Allen_Claim 8.3} to prove part $(a)(b)$. Let us consider the following choices in
%   \cref{lem:Allen_Claim 8.3}:
%   \begin{align*}
%     \bm{g}&=\frac{\bm{\alpha}^{(k+1)}_{\z,0}}{\sqrt{{2}/{m_{k+1}}}\norm{\tilde{\bm{\alpha}}_{\z,0}^{(k)}}_2}=\frac{\bm{W}_0^{(k)}\tilde{\bm{\alpha}}_{\z,0}^{(k)}}{\norm{\tilde{\bm{\alpha}}_{\z,0}^{(k)}}_2}\sim\mathcal{N}(0,\bm{I}),\qquad~
%     \bm{g}'=\frac{\bm{\alpha}^{(k+1)}_{\x,t}-\bm{\alpha}^{(k+1)}_{\z,0}}{\sqrt{{2}/{m_{k+1}}}\norm{\tilde{\bm{\alpha}}_{\z,0}^{(k)}}_2}.
%   \end{align*}
%   It follows that $\norm{\bm{g}'}_2\leq O(\tau/\sqrt{m})\cdot O(\sqrt{m})\leq O(\tau)$ since we have previously shown that $\norm{\tilde{\bm{\alpha}}_{\z,0}^{(k)}}_2=\Theta(1)$ in \cref{lem:Init_LayerOutputBounds}. Therefore, we can choose $\delta=\Theta(\tau)$ such that $\norm{\bm{g}'}_2\leq\delta$. Then, we can obtain
%   \begin{align*}
%     \bm{g}+\bm{g}'=\frac{\bm{\alpha}^{(k+1)}_{\x,t}}{\sqrt{{2}/{m_{k+1}}}\norm{\tilde{\bm{\alpha}}_{\z,0}^{(k)}}_2},\quad~\bm{D}'=\bm{D}_{\x,t}^{(k+1)} - \bm{D}_{{\bm{z}},0}^{(k+1)}\quad~\text{and}\quad~\bm{u}=\frac{\pt{\bm{D}_{\x,t}^{(k+1)} - \bm{D}_{{\bm{z}},0}^{(k+1)}}\bm{\alpha}^{(k+1)}_{\x,t}}{\sqrt{{2}/{m_{k+1}}}\norm{\tilde{\bm{\alpha}}_{\z,0}^{(k)}}_2}&,
%   \end{align*}
%   where $\bm{D}'$ and $\bm{u}$ are defined as in \cref{lem:Allen_Claim 8.3}. By applying \cref{lem:Allen_Claim 8.3}, we can get $\norm{\bm{D}'}_0\leq O(m^{2/3}\tau^{2/3})$ and $\norm{\bm{u}}_2\leq 2\delta$, which establish the conclusion of part $(a)(b)$.


% % As to the second statement, we define $\bs{u}^{(l)}$ as following:

% % % Let $u^{(l)} = (\bs{D}^{(l)}_{\bs{x},t} - \bs{D}^{(l)}_{\bs{z},0}) \bs{\alpha}_t^{(l)}(x)$

% % \begin{equation}
% % \begin{aligned}
% % \bm{u}^{(l)} &:= (\bm{D}_{\x,t}^{(l)} - \bm{D}_{{\bm{z}},0}^{(l)}) \bm{\alpha}^{(l)}_t(\x)=
% % \left(\bm{D}_{\x,t}^{(l)} - \bm{D}_{{\bm{z}},0}^{(l)}\right)\left(\bm{\alpha}^{(l)}_0({\bm{z}})+(\bm{\alpha}^{(l)}_t(\x)-\bm{\alpha}^{(l)}_0({\bm{z}}))\right) \\
% % &=(\bm{D}_{\x,t}^{(l)} - \bm{D}_{{\bm{z}},0}^{(l)})\left[\sqrt{\frac{2}{m_l}}\bm{W}^{(l-1)}_0{\tilde{\bm{\alpha}}}^{(l-1)}_0({\bm{z}})+
% % \left(\bm{\alpha}^{(l)}_t(\x)-\bm{\alpha}^{(l)}_0({\bm{z}})\right)\right],
% % \end{aligned}
% % \end{equation}


% % Now we re-scale $\bm{u}^{(l)}$ by ${1}/{\norm{\tilde{\bm{\alpha}}^{(l-1)}_0({\bm{z}})}_2}$, then with Lemma \ref{lem: Recur_D0} we know (let$\bm{D}'=\bm{D}_{\x,t}^{(l)} - \bm{D}_{{\bm{z}},0}^{(l)}$, $\bm{g}=\sqrt{{2}/{m_l}}\bm{W}^{(l-1)}_0{ \bm{\tilde{\alpha}}^{(l-1)}_0({\bm{z}})}/{\norm{\tilde{\bm{\alpha}}^{(l-1)}_0({\bm{z}})}_2}$ and $\bm{g}'=\big({\bm{\alpha}^{(l)}_t(\x)-\bm{\alpha}^{(l)}_0({\bm{z}})}\big)/{\norm{\tilde{\bm{\alpha}}^{(l-1)}_0({\bm{z}})}_2}$) with probability at least $1-\exp\big(-\Omega(m^{2/3} \tau^{2/3})\big)$, we have:
% % \begin{align}
% % \label{a2}
% % \norm{\bm{u}^{(l)}}_0 \leq \norm{\bm{D}_{\x,t}^{(l)} - \bm{D}_{{\bm{z}},0}^{(l)}}_0 = O\big(m^{\frac{2}{3}}\tau^\frac{2}{3}\big); \quad \text{and} \quad
% % \norm{\bm{u}^{(l)}}_2 = O\big(\frac{\tau}{\sqrt{m}}\big).
% % \end{align}


%   Further, the third statement can be directly obtained from the following inequality:
%   \begin{align*}
%     \norm{\tilde{\bm{\alpha}}^{(l)}_{\x,t} -\tilde{\bm{\alpha}}^{(l)}_{\bm{z},0}}_2 &\leq \norm{\bm{D}^{(l)}_{{\bm{z}},0}\pt{\bm{\alpha}^{(l)}_{\x,t} - \bm{\alpha}^{(l)}_{\bm{z},0}}}_2+\norm{\pt{\bm{D}^{(l)}_{\x,t} - \bm{D}^{(l)}_{{\bm{z}},0} }\bm{\alpha}^{(l)}_{\x,t} }_2.
%   \end{align*}
%   Thus, the proof of this lemma is complete.
% \end{proof}

% \begin{lemma}
%   \label{lem:NN_forward_perturbation2}
%   Fix ${\z} \in  \mc{\X}\times\{1\}$.
%   Suppose that $\tau \in[\Omega(1/\sqrt{m}),  O\Big({\sqrt{m}}/{(\ln m)^3}\Big)]$ and
%   $\sup_{t \geq 0}\norm{\bm{W}^{(l)}_t - \bm{W}^{(l)}_0 }_{\mr{F}}\leq\tau$ holds for all $l$.
%   Then, with probability at least $1-\exp({-\Omega( m^{2/3} \tau^{2/3})})$, for any $\x \in  \mc{\X}\times\{1\}$ such that $\norm{\x-\z}_2 = O(1/m)$ and $t\geq0$, we have:
%   \begin{enumerate}[(a)]
%     \item We can decompose $\bm{\alpha}^{(l)}_t(\x) - \bm{\alpha}^{(l)}_0({\z}) = \bm{y}^{(l)}_1 + \bm{y}^{(l)}_2$ such that
%     \begin{align*}
%       \norm{\bm{y}^{(l)}_1}_2 = O\big({\tau}/{\sqrt{m}}\big) \qand \norm{\bm{y}^{(l)}_2}_{\infty} = O\big({\tau\sqrt{\ln m }}/{m}\big);
%     \end{align*}
%     \item $\norm{\bm{D}_{\x,t}^{(l)} - \bm{D}_{{\z},0}^{(l)}}_0 = O\big(m^{2/3} \tau^{2/3}\big)$ and $\norm{\big(\bm{D}_{\x,t}^{(l)} - \bm{D}_{{\z},0}^{(l)}\big)\bm{\alpha}_t^{(l)}(\x)}_2= O\big({\tau}/{\sqrt{m}}\big)$;
%     \item $\norm{\tilde{\bm{\alpha}}^{(l)}_t(\x) - \tilde{\bm{\alpha}}_0^{(l)}({\z})}_2 = O\big(\tau\sqrt{\ln m}/\sqrt{m}\big)$ and $\norm{\bm{\alpha}^{(l)}_t(\x) - \bm{\alpha}^{(l)}_0({\z}) }_2= O\big(\tau\sqrt{\ln m}/\sqrt{m}\big)$.
%   \end{enumerate}
% \end{lemma}
% \begin{proof}
%   [Proof of \cref{lem:NN_forward_perturbation2}]
%   We prove the lemma by induction.
%   When $l=0$, it is easy to get that $\norm{{\tilde{\bm{\alpha}}}_t^{(0)}(\x) -{\tilde{\bm{\alpha}}}^{(0)}_0({\z})}_2=\norm{\bm{\alpha}_t^{(0)}(\x) -\bm{\alpha}^{(0)}_0({\z})}_2 = \norm{\x-{\z}}_2 = O(1/m)$. Thus, all the statements hold for $l=0$.

%   When $l \in \{1,2\cdots,L\}$, we assume that the lemma holds for $0,1,2\cdots,l-1$.

%   $(a)$: First of all, we have
%   \begin{align*}
%     &\bm{\alpha}^{(l)}_t(\x) -\bm{\alpha}^{(l)}_0({\z})=
%     \sqrt{\frac{2}{m_l}} \bm{W}^{(l-1)}_t \bm{D}^{(l-1)}_{\x,t} \bm{\alpha}^{(l-1)}_t(\x) - \sqrt{\frac{2}{m_l}} \bm{W}_0^{(l-1)} \bm{D}_{{\z},0}^{(l-1)} \bm{\alpha}^{(l-1)}_0({\z})=\cdots\\
%     &\qquad=\sum^{l}_{a=1}\Big(\prod_{b=a+1}^{l} \sqrt{\frac{2}{m_{b}}}\bm{W}_0^{(b-1)} \bm{D}_{{\z},0}^{(b-1)}\Big)\Big(\sqrt{\frac{2}{m_a}}\big(\bm{W}^{(a-1)}_t - \bm{W}^{(a-1)}_0\big)\bm{D}^{(a-1)}_{\x,t} \bm{\alpha}^{(a-1)}_t(\x)\\
%     &\qquad\qquad+ \sqrt{\frac{2}{m_a}}\bm{W}_0^{(a-1)}\big(\bm{D}^{(a-1)}_{\x,t} - \bm{D}^{(a-1)}_{{\z},0}\big) \bm{\alpha}^{(a-1)}_t(\x) \Big) + \Big( \prod_{b=1}^{l}\sqrt{\frac{2}{m_b}} \bm{W}_0^{(b-1)} \bm{D}^{(b-1)}_{\z,0}\Big)(\x-\z)

%   \end{align*}

%   holds for $l \in \{1,2,\cdots,L\}$. For the first part in $\tilde{\bm{\alpha}}^{(l)}_t(\x) -\tilde{\bm{\alpha}}^{(l)}_0({\z})$, we have
%   \begin{align*}
%     &\norm{\Big(\prod_{b=a+1}^{l} \sqrt{\frac{2}{m_b}}\bm{W}_0^{(b-1)} \bm{D}_{{\z},0}^{(b-1)}\Big)\Big(\sqrt{\frac{2}{m_a}}\big(\bm{W}^{(a-1)}_t - \bm{W}^{(a-1)}_0\big)\bm{D}^{(a-1)}_{\x,t} \bm{\alpha}^{(a-1)}_t(\x)\Big)}_2\\
%     &\qquad\leq \norm{\prod_{b=a+1}^{l} \sqrt{\frac{2}{m_b}}\bm{W}_0^{(b-1)} \bm{D}_{{\z},0}^{(b-1)} }_2 \cdot \norm{\sqrt{\frac{2}{m_a}}\big(\bm{W}^{(a-1)}_t - \bm{W}^{(a-1)}_0\big)}_2 \cdot \norm{\bm{D}^{(a)}_{\x,t}}_2 \\
%     &\qquad\qquad\qquad\qquad\cdot \Big(\norm{\bm{\alpha}^{(a-1)}_0({\z})}_2 + \norm{\bm{\alpha}^{(a-1)}_t(\x) -\bm{\alpha}^{(a-1)}_0({\z})}_2 \Big)\\
%     &\qquad\leq O(1) \cdot O(\frac{\tau}{\sqrt{m}}) \cdot O(1) \cdot \bigg(O(1)+O\Big(\frac{\tau \sqrt{\ln m}}{\sqrt{m}}\Big)\bigg) \leq O\big(\frac{\tau}{\sqrt{m}}\big).
%   \end{align*}


%   For the second part in $\bm{\alpha}^{(l)}_t(\x) -\bm{\alpha}^{(l)}_0({\z})$, let
%   \begin{align*}
%     \bm{v}^{(a)}& \coloneqq \prod_{b=a+1}^{l} \left(\sqrt{\frac{2}{m_b}}\bm{W}_0^{(b-1)} \bm{D}_{{\z},0}^{(b-1)}\right)
%     \sqrt{\frac{2}{m_a}}\bm{W}_0^{(a-1)}(\bm{D}^{(a-1)}_{\x,t} - \bm{D}^{(a-1)}_{{\z},0}) \bm{\alpha}^{(a-1)}_t(\x);\\
%     \bm{u}^{(a)} &\coloneqq (\bm{D}_{\x,t}^{(a)} - \bm{D}_{{\z},0}^{(a)}) \bm{\alpha}^{(a)}_t(\x)=
%     \left(\bm{D}_{\x,t}^{(a)} - \bm{D}_{{\z},0}^{(a)}\right)\left(\bm{\alpha}^{(a)}_0({\z})+(\bm{\alpha}^{(a)}_t(\x)-\bm{\alpha}^{(a)}_0({\z}))\right) \\
%     &=(\bm{D}_{\x,t}^{(a)} - \bm{D}_{{\z},0}^{(a)})\left[\sqrt{\frac{2}{m_a}}\bm{W}^{(a-1)}_0{\tilde{\bm{\alpha}}}^{(a-1)}_0({\z})+
%     \left(\bm{\alpha}^{(a)}_t(\x)-\bm{\alpha}^{(a)}_0({\z})\right)\right],
%   \end{align*}
%   for each $a=1,2,3,\cdots,l$.
%   Then we can see
%   \begin{align*}
%     \bm{v}^{(1)}=0;\qquad 
%     \bm{v}^{(a)} = \prod_{b=a+1}^{l} \left(\sqrt{\frac{2}{m_b}}\bm{W}_0^{(b-1)} \bm{D}_{{\z},0}^{(b-1)}\right)
%     \sqrt{\frac{2}{m_a}}\bm{W}^{(a-1)}_0\bm{u}^{(a-1)},\quad a=2,3,\cdots,l.
%   \end{align*}
%   When $a\geq 2$, we have assumed that $\bm{\alpha}^{(a-1)}_t(\x)-\bm{\alpha}^{(a-1)}_0({\z}) $ 
%   can be written as $\bm{\alpha}^{(a-1)}_t(\x) - \bm{\alpha}^{(a-1)}_0({\z}) = \bm{y}^{(a-1)}_1 + \bm{y}^{(a-1)}_2 $ 
%   where $\norm{\bm{y}^{(a-1)}_1}_2 = O\big({\tau}/{\sqrt{m}}\big)$ and $\norm{\bm{y}^{(a-1)}_2}_{\infty} = O\big({\tau\sqrt{\ln m}}/{m}\big)$. If we re-scale $\bm{u}^{(a-1)}$ by ${1}/{\norm{\tilde{\bm{\alpha}}^{(a-2)}_0({\z})}_2}$, then with Lemma \ref{lem:Allen_Claim 8.3} (let $\bm{D}'=\bm{D}_{\x,t}^{(a-1)} - \bm{D}_{{\z},0}^{(a-1)}$, $\bm{W}=\bm{W}^{(a-2)}_0$, $\bm{h}={\bm{\alpha}^{(a-2)}_0({\z})}/{\norm{\tilde{\bm{\alpha}}^{(a-2)}_0({\z})}_2}$ and $\bm{g}=\big({\bm{\alpha}^{(a-1)}_t(\x)-\bm{\alpha}^{(a-1)}_0({\z})}\big)/{\norm{\tilde{\bm{\alpha}}^{(a-2)}_0({\z})}_2}$) we can know with probability at least $1-\exp\big(-\Omega(m^{2/3} \tau^{2/3})\big)$:
%   \begin{align}
%     \label{a2}
%     \norm{\bm{u}^{(a-1)}}_0 \leq \norm{\bm{D}_{\x,t}^{(a-1)} - \bm{D}_{{\z},0}^{(a-1)}}_0 = O\big(m^{\frac{2}{3}}\tau^\frac{2}{3}\big); \quad \text{and} \quad
%     \norm{\bm{u}^{(a-1)}}_2 = O\big(\frac{\tau}{\sqrt{m}}\big).
%   \end{align}
%   Then using \cref{lem:Allen_Claim 8.5} (with $\bm{u}=\bm{u}^{(a-1)}$, $\bm{v}=\bm{v}^{(a)}$ and $s=C m^{2/3}\tau^{2/3}$ such that $\norm{\bm{u}^{(a-1)}}_0\leq s$),
%   we have probability at least $1- \exp\big(-\Omega(m^{2/3} \tau^{2/3})\big)$, one can write $\bm{v}^{(a)}=\bm{v}_1+\bm{v}_2$ for:
%   \begin{equation*}
%     \norm{\bm{v}_1}_2= O\big(m^{-\frac{2}{3}} \tau^\frac{4}{3}\ln m\big) \quad
%     \text{and} \quad \norm{\bm{v}_2}_{\infty} = O\Big(\frac{\tau\sqrt{\ln m}}{m}\Big).
%   \end{equation*}
%   For the third part in $\bm{\alpha}^{(l)}_t(\x) -\bm{\alpha}^{(l)}_0({\z})$, by \cref{lem:Init_WeightsProductBound} we have
%   \begin{align*}
%     \norm{\Big( \prod_{b=1}^{l}\sqrt{\frac{2}{m_b}} \bm{W}_0^{(b-1)} \bm{D}^{(b-1)}_{\x,t}\Big)(\x-\z)}_2  = O(1) \cdot O(\frac{1}{m}) = O(\frac{1}{m})
%   \end{align*}

%   $(b)$: Now we have proved that $(a)$ holds for $l$, so we can similarly get that (\ref{a2}) holds for $a=l+1$.

%   $(c)$: According to $(a)$ and the fact that for $\bm{u} \in \R^p$, $\norm{\bm{u}}_2 \leq \sqrt{p} \norm{\bm{u}}_{\infty}$,
%   we know that
%   \begin{equation*}
%     \norm{\bm{\alpha}_t^{(l)}(\x) -\bm{\alpha}_0^{(l)}({\z})}_2 =O\Big(\tau\sqrt{\frac{\ln m}{m}} \Big).
%   \end{equation*}

%   As for the other term in \cref{lem:NN_forward_perturbation2}$(c)$, by $(b)$, we have
%   \begin{equation*}
%     \norm{\tilde{\bm{\alpha}}^{(l)}_t(\x) -\tilde{\bm{\alpha}}^{(l)}_0({\z})}_2 \leq \norm{\bm{D}^{(l)}_{{\z},0}\left(\bm{\alpha}^{(l)}_t(\x) - \bm{\alpha}^{(l)}_0({\z})\right)}_2 +\norm{\left(\bm{D}^{(l)}_{\x,t} - \bm{D}^{(l)}_{{\z},0} \right)\bm{\alpha}^{(l)}_t(\x) }_2\\
%     = O\Big(\tau\sqrt{\frac{\ln m}{m}}\Big).
%   \end{equation*}
% \end{proof}

% \begin{lemma}[\citet{allen-zhu2019_ConvergenceTheory} Claim 8.3]\label{claim 8.3}
% Suppose $\delta_2 \in [0,O(1)]$ and $\delta_\infty \in \big[0,\frac{1}{4\sqrt{m}}\big]$. Suppose $W$ is a random matrix with entries drawn i.i.d from $\mathcal{N}(0,1)$. With probability at least $1- e^{-\Omega\big(m^\frac{3}{2}\delta_\infty\big)}$, the following holds. Fix any unit vector $h \in \R^m$, and for all $g\in \R^m$ that can be written  as
% \[g= g_1+g_2\quad where \quad\norm{g_1}_2\leq \delta_2 \quad and \quad \norm{g_2}_\infty \leq \delta_\infty.\]
% Let $D' \in \R^{m\times m}$ be the diagonal matrix where
% \[(D')_{k,k}  =  \bm{1}_{\big(\sqrt{\frac{2}{m}}Wh+g\big)_k \geq 0} - \bm{1}_{\big(\sqrt{\frac{2}{m}}Wh\big)_k \geq 0},\quad~\forall k \in [m].\]
% Then, letting $x=D'\left(\sqrt{\frac{2}{m}}Wh+g\right) \in \R^m$, we have
% \[\norm{\x}_0 \leq \norm{D'}_0 \leq O(m(\delta_2)^\frac{2}{3}+\delta_\infty m^\frac{3}{2})\quad  and \quad \norm{\x}_2 \leq O(\delta_2+(\delta_\infty)^\frac{3}{2}m^\frac{3}{4}).\]
% \end{lemma} 

% In the proof of \cref{lem:NN_forward_perturbation2}, we use the following result:

% \begin{lemma}[\citet{allen-zhu2019_ConvergenceTheory} Claim 8.3]
%   \label{lem:Allen_Claim 8.3}
%   % Suppose $\delta_2 \in [0,O(1)]$, $\delta_\infty \in \big[0,1/({4\sqrt{m}})\big]$ and $m_1 \in [1,M]$.
%   % Suppose $\bm{W} \in \R^{m\times m_1}$ is a random matrix with entries drawn i.i.d from $\mathcal{N}(0,1)$.
%   % With probability at least $1- \exp(-\Omega\big(m^{3/2}\delta_\infty\big))$, the following holds.
%   % Fix any unit vector $\bm{h} \in \R^m$, and for all $\bm{g}\in \R^{m_1}$ that can be written as
%   % \begin{align*}
%   %   \bm{g}= \bm{g}_1+\bm{g}_2\quad \text{where} \quad\norm{\bm{g}_1}_2\leq \delta_2 \quad \text{and} \quad \norm{\bm{g}_2}_\infty \leq \delta_\infty.
%   % \end{align*}
%   % Let $\bm{D}' \in \R^{m\times m}$ be the diagonal matrix where
%   % \begin{align*}
%   % (\bm{D}')
%   %   _{k,k}  =  \bm{1}\big\{\big(\sqrt{{2}/{m}}\bm{Wh}+\bm{g}\big)_k \geq 0\big\} - \bm{1}\big\{\big(\sqrt{{2}/{m}}\bm{Wh}\big)_k \geq 0\big\},\quad~\forall k \in [m].
%   % \end{align*}
%   % Then, letting $\x=\bm{D}'\left(\sqrt{{2}/{m}}\bm{Wh}+\bm{g}\right) \in \R^m$, we have
%   % \begin{align*}
%   %   \norm{\x}_0 \leq \norm{\bm{D}'}_0 = O\big(m(\delta_2)^\frac{2}{3}+\delta_\infty m^\frac{3}{2}\big)\quad  and \quad \norm{\x}_2 = O\big(\delta_2+(\delta_\infty)^\frac{3}{2}m^\frac{3}{4}\big).
%   % \end{align*}

%   Suppose each entry of $\bm{g} \in \mathbb{R}^m$ follows $g_i \iid \mathcal{N}(0,1)$. For any $\delta > 0$, with probability at least $1-\exp\mpt{-m^{2/3}\delta^{2/3}/25 } $, the following proposition holds:

%   Select $\bm{g}' \in \mathbb{R}^m$ such that $ \norm{\bm{g}'}_2\leq\delta.$
%   Let $\bm{D}' = \mr{diag}(D'_{kk})$ be a diagonal matrix, where the $k$-th diagonal element
%   $D_{kk}' $ follows
%   $$D_{k k}^{\prime}=\bs{1}\left\{\left(g+g^{\prime}\right)_{k} \geq 0\right\}-\bs{1}\left\{g_{k} \geq 0\right\},\quad k\in[m].$$
%   If we define $\bm{u} = \bm{D}'(\bm{g}+\bm{g}')$, then it satisfies the following inequalities:
%   \[\norm{\bm{u}}_0 \leq \norm{\bm{D}'}_0 \leq 3m^{2/3}\delta^{2/3} \quad \text{ and }\quad \norm{\bm{u}}_2 \leq 2 \delta.\]
% \end{lemma}
% \begin{proof}
%   First, we define $S_0 = \cl{ j\in\bk{m}:D'_{jj}\not = 0}$ as the set of indices $j$ for which $D'_{jj}$ is non-zero. It is evident that $\norm{\bm{D}'}_0$ is equal to the cardinality of $S_0$, denoted as $|S_0|$.

%   According to the definition of $\bm{D}'$, in order for $D'_{jj}$ to be non-zero, it must hold that $\abs{g_j}\leq\abs{g'_j}$. Therefore, we further define $S_1=\cl{j\in[m]:\abs{g_j}\leq\abs{g'_j}}$. Consequently, we have $S_0\subseteq S_1$ and $\norm{\bm{D}'}_0=|S_0|\leq|S_1|$.

%   For any $\delta_\infty > 0$, we define $S_2 = \cl{ j \in [m]: \lvert g_j\rvert\leq 2\delta_\infty }$. Since $g_j$ follows a standard normal distribution, we can get:
%   \begin{align*}
%     \mathbb{P}\left(\lvert g_{j}\rvert\leq2 \delta_{\infty}\right)=2 \int_{0}^{2 \delta_{\infty}} \frac{\exp\mpt{-{x^{2}}/{2}}}{\sqrt{2 \pi}} \mathrm{d} x
% \leq2 \int_{0}^{2 \delta_{\infty}} \frac{\mathrm{d} x}{\sqrt{2 \pi}} \leq\frac{4 \delta_{\infty}}{\sqrt{2 \pi}} 
%     \leq\frac{8}{5} \delta_{\infty}&.
%   \end{align*}
%   It is easy to see that $\bs{1}\{\lvert g_j \rvert\leq 2 \delta_\infty\} \iid \mathcal{B}(1,p)$ and $\lvert S_2 \rvert = \sum\limits^m_{j=1} \bs{1} \mcl{ \lvert g_j \rvert\leq 2\delta_\infty } \sim\mathcal{B}(m,p)$, where $ p = \mathbb{P}(\lvert g_j\rvert\leq 2 \delta_\infty) \leq {8}\delta_\infty/5 $. By combining this with Bernstein's inequality, we can obtain the following result:
%   \begin{align}
%     \label{eq: |S1|bound}
%     \begin{split}
%       \mathbb{P}\left(\lvert S_{2}\rvert\leq2 m \delta_{\infty}\right) \geq \mathbb{P}\left(\lvert S_{2}\rvert\leq mp+\frac{2}{5} m \delta_{\infty}\right)
%       \geq 1-\exp\mpt{-\frac{1}{25} m \delta_{\infty}}&.
%     \end{split}
%   \end{align}


%   Next, we partition $S_1$ as follows: $S_1 = S_{1} \cap (S_{2} \cup S_{2}^{\mathrm{c}}) = (S_{1} \cap S_{2}) \cup (S_{1} \cap S_{2}^{\mathrm{c}})$. Therefore, we have $\lvert S_{1}\rvert = \lvert S_{1} \cap S_{2}\rvert + \lvert S_{1} \cap S_{2}^{\mathrm{c}}\rvert \leq \lvert S_{2}\rvert + \lvert S_{1} \cap S_{2}^{\mathrm{c}}\rvert$. It is evident that the upper bound of $|S_1|$ can be controlled using the inequality \cref{eq: |S1|bound}, thus we only need to further control $|S_1 \cap S_2^c|$ in order to obtain an estimation for $\norm{\bm{D}'}_0$. For any $j \in S_1 \cap S_2^c $, we have $\abs{ g'_{j}}\geq \abs{g_{j}}\geq\delta_{\infty}$, and thus
%   \begin{align*}
%     \lvert S_{1} \cap S_{2}^{\mathrm{c}}\rvert\delta_{\infty}^{2}=\sum_{j \in S_{1} \cap S_{2}^{\mathrm{c}}} \delta_{\infty}^{2} \leq\sum_{j \in S_{1} \cap S_{2}^{\mathrm{c}}}\lvert g^{\prime}_{j}\rvert ^{2} \leq \left\|g^{\prime}\right\|^{2}
%     \qquad\Longrightarrow\qquad\lvert S_{1} \cap S_{2}^{\mathrm{c}}\rvert\leq{\left\|g^{\prime}\right\|^{2}}/{\delta_{\infty}^{2}}.
%   \end{align*}
%   Then we know that
%   \begin{equation*}
%     \begin{aligned}
%       \lvert S_{1}\rvert \leq \lvert S_{2}\rvert +\lvert S_{1} \cap S_{2}^{\mathrm{c}}\rvert\leq2 m \delta_{\infty}+\frac{\left\|g_{1}^{\prime}\right\|^{2}}{\delta_{\infty}^{2}} \leq2 m \delta_{\infty}+\frac{\delta^{2}}{\delta_{\infty}^{2}} .
%     \end{aligned}
%   \end{equation*}
%   When $m \delta_\infty = {\delta^2}/{\delta_\infty^2}$, which implies $ \delta_\infty = {\delta^{2/3}}/{m^{1/3}}$, the right-hand side of the inequality reaches its minimum value of $3m^{{2}/{3}} \delta^{{2}/{3}}$.

%   Finally, we try to obtain the estimation of $\norm{\bm{u}}_2$. For $j \in S_0$, we know that $D'_{jj} = \pm 1 $. Combining the definition of $\bm{u}$ with $\lvert g_j \rvert \leq \lvert g_j' \rvert$, we have
%   $\lvert u_{j}\rvert=\lvert g_{j}+ g^{\prime}_{j} \rvert \leq \lvert g_{j} \rvert+\lvert g^{\prime}_{j}\rvert \leq2\lvert g^{\prime}_{j} \rvert$. This implies that:
%   \begin{align*}
%     \norm{\bm{u}}^{2} & =\sum_{j \in S_{0}}u_{j}^{2} \leq\sum_{j \in S_{0}} 4 (g'_{j})^{2}\leq4 \delta^{2}.
%   \end{align*}
%   Thus, the proof of this lemma is complete.
% \end{proof}

% \begin{lemma}[\citet{allen-zhu2019_ConvergenceTheory} Claim 8.5]
%   \label{lem:Allen_Claim 8.5}
%   For any $2 \leq a \leq b \leq L $ and any positive integer $s= O({m}/{\ln m})$, with probability at least $1-\exp\big(-\Omega(s\ln m)\big)$, for all $\bm{u} \in \R^m$ with $\norm{\bm{u}}_2 \leq 1 $ and $\norm{\bm{u}}_0 \leq s $, letting $\bm{v}=\prod_{k=a+1}^{b} \Big(\sqrt{{2}/{m_k}}\bm{W}_0^{(k-1)} \bm{D}_{0}^{(k-1)}\Big)\sqrt{{2}/{m_a}}\bm{W}^{(a-1)}_0\bm{u}$, we can write $\bm{v}=\bm{v}_1 + \bm{v}_2$ with
%   \[\norm{\bm{v}_1}_2 = O\big(\sqrt{{s}/{m}} \ln m\big) \quad \text{and} \quad \norm{\bm{v}_2}_\infty \leq \frac{2\sqrt{\ln m}}{\sqrt{m}}.\]
% \end{lemma}

% \begin{lemma}
%   \label{lem:GradientBound_Deviance}
%   Let $\tau\in\bk{\Omega\mpt{1/\sqrt{m}},O\mpt{{\sqrt{m}}/{(\ln m)^3}}}$, $T\subseteq [0,\infty)$ and fix $\bm{z}\in\tilde{B}_R$. Suppose that $\norm{\bm{W}^{(l)}_t - \bm{W}^{(l)}_0 }_{F}\leq\tau$ holds for all $t\in T$ and $l\in\cl{0,1,\cdots,L}$. Then there exists a positive absolute constant $C$ such that with probability at least $1-\exp\mpt{-\Omega(m^{2/3}\tau^{2/3})}$, for all $l\in\cl{0,1,\cdots,L}$ and $\x\in\tilde{B}_R$ such that $\norm{\x-\bm{z}}_2\leq O\mpt{1/m}$, we have

%   % Suppose $\sup_{t \geq 0} \norm{\bm{W}^{(l)}_t - \bm{W}^{(l)}_0 }_{\mr{F}}= O(m^{1/4}) $ holds for all $l$.
%   % For fixed $\z \in  \tilde{B}_R$, with probability at least $1-\exp(-\Omega(m^{5/6}))$ over the random initialization,
%   % for any $\x \in  \mc{\X}$ such that  $\norm{\x - \z}_2 =  O(1/m)  $, we have
%   \begin{align*}
%     \sup_{t\in T}\norm{\nabla_{\bm{W}^{(l)}} g_t(\x) - \nabla_{\bm{W}^{(l)}} g_0(\z)}_{\mr{F}} = O\mpt{m^{-1/12} \sqrt{ \ln m }}.
%   \end{align*}
%   when $m$ is greater than the positive constant $C$.
% \end{lemma}

% \begin{proof}
%   Recalling \cref{eq:GradientExpr}, we have
%   $\nabla_{\bm{W}^{(l)}} g_t(\x)=\bm{\gamma}^{(l)}_{\x,t}\tilde{\bm{\alpha}}^{(l),T}_{\x,t}$.
%   Then,
%   \begin{align}
%     &\norm{\nabla_{\bm{W}^{(l)}} g_t(\x) - \nabla_{\bm{W}^{(l)}} g_0(\z)}_{\mr{F}}=\norm{\bm{\gamma}^{(l)}_{\x,t}\tilde{\bm{\alpha}}^{(l),T}_{\x,t} -\bm{\gamma}^{(l)}_{\z,0}\tilde{\bm{\alpha}}^{(l),T}_{\z,0}}_{\mr{F}}\notag \\
%     \label{eq:Proof_Training_GradientBound_Decomp2}
%     &\qquad \leq \norm{\bm{\gamma}^{(l)}_{\x,t} - \bm{\gamma}^{(l)}_{\z,0}}_2\norm{\tilde{\bm{\alpha}}^{(l)}_{\x,t}}_2
%     + \norm{\bm{\gamma}^{(l)}_{\z,0}}_2\norm{\tilde{\bm{\alpha}}^{(l)}_{\x,t} - \tilde{\bm{\alpha}}^{(l)}_{\z,0}}_2.
%   \end{align}

%   According to \cref{lem:Init_WeightsProductBound}, with probability at least $1-\exp(-\Omega(m))$, we can get
%   \[\norm{\bm{\gamma}^{(l)}_{\z,0}}_2 = O(1)\qquad\text{and}\qquad\norm{\tilde{\bm{\alpha}}_{\z,0}^{(l)}}_2 = O(1).\]
%   On the other hand, by \cref{lem:NN_forward_perturbation2}, with probability at least $1-\exp\mpt{-\Omega(m^{2/3}\tau^{2/3})}$, for all $l\in\cl{0,1,\cdots,L}$ and $\x\in\tilde{B}_R$ such that $\norm{\x-\bm{z}}_2\leq O\mpt{1/m}$, we have
%   $\norm{\tilde{\bm{\alpha}}^{(l)}_{\x,t} - \tilde{\bm{\alpha}}_{\z,0}^{(l)}}_2 = O({\tau }/{\sqrt{m}})$, which implies that (since $\tau = O\big({\sqrt{m}}/{(\ln m)^3}\big)$)
%   \begin{align*}
%     \norm{\tilde{\bm{\alpha}}_{\x,t}^{(l)}}_2
%     \leq \norm{\tilde{\bm{\alpha}}_{\z,0}^{(l)}}_2 + \norm{\tilde{\bm{\alpha}}^{(l)}_{\x,t} - \tilde{\bm{\alpha}}_{\z,0}^{(l)}}_2 = O(1).
%   \end{align*}
% %  since $O\big({\sqrt{m}}/{(\ln m)^3}\big)$,
%   Moreover, the statement $(a)$ in \cref{lem:NN_forward_perturbation2} also enable us to
%   apply \cref{lem:backward perturbation} for $\bm{W}^{(r)'} = \bm{W}_t^{(r)} - \bm{W}_0^{(r)}$,
%   $\bm{D}^{(r)'} =  \bm{D}^{(r)}_{\x,t} -  \bm{D}^{(r)}_{\z,0}$, yielding that with probability at least $1-\exp\mpt{-\Omega(m^{2/3}\tau^{2/3})}$, for all $l\in\cl{0,1,\cdots,L}$ and $\x\in\tilde{B}_R$ such that $\norm{\x-\bm{z}}_2\leq O\mpt{1/m}$, we have
%   \begin{align*}
%     \norm{{\bm{\gamma}}^{(l)}_{\x,t} - {\bm{\gamma}}^{(l)}_{\z,0}}_2
%     = O\mpt{m^{-1/6} \tau^{1/3}\sqrt{\ln m}}.
%   \end{align*}
%   Plugging these bounds in \cref{eq:Proof_Training_GradientBound_Decomp2}, we can get the result.
% \end{proof}
% We first decompose
% \begin{equation}
%   \label{eq:Proof_GradientDev_Decomp}
%   \begin{aligned}
%     &\norm{\nabla_{\bm{W}^{(l)}} g_t(\x) - \nabla_{\bm{W}^{(l)}} g_t(\z)}_{\mr{F}} \\
%     &\leq \norm{\nabla_{\bm{W}^{(l)}} g_t(\x) - \nabla_{\bm{W}^{(l)}} g_0(\z)}_{\mr{F}} +
%     \norm{\nabla_{\bm{W}^{(l)}} g_0(\z) - \nabla_{\bm{W}^{(l)}} g_t(\z)}_{\mr{F}}.
%   \end{aligned}
% \end{equation}
%   Since $\z$ is fixed, the second term in \cref{eq:Proof_GradientDev_Decomp} is already bounded by \cref{lem:Training_GradientBound}
%   with $\tau \asymp m^{1/4}$.
% %  with probability at least $1-\exp(-\Omega(m^{5/6}))$ .
%   For the first term in \cref{eq:Proof_GradientDev_Decomp}, we recall \cref{eq:GradientExpr} to get

% \onecolumngrid
%   \begin{align}
%     &\norm{\nabla_{\bm{W}^{(l)}} g_t(\x) - \nabla_{\bm{W}^{(l)}} g_0(\z)}_{\mr{F}}  = \norm{\tilde{\bm{\alpha}}_t^{(l)}(\x) \left(  \prod_{r=l+1}^{L} \sqrt{\frac{2}{m_r}}  \bm{W}_t^{(r)}  \bm{D}_{\x,t}^{(r)}  \right)
%     - \tilde{\bm{\alpha}}_0^{(l)}(\z) \left(  \prod_{r=l+1}^{L} \sqrt{\frac{2}{m_r}}  \bm{W}_0^{(r)} \bm{D}_{\z,0}^{(r)} \right)}_{\mr{F}} \nonumber \\
%     &\qquad = \norm{\tilde{\bm{\alpha}}_t^{(l)}(\x)} \norm{  \prod_{r=l+1}^{L} \sqrt{\frac{2}{m_r}}  \bm{W}_t^{(r)}  \bm{D}_{\x,t}^{(r)}   -
%     \prod_{r=l+1}^{L} \sqrt{\frac{2}{m_r}}  \bm{W}_0^{(r)} \bm{D}_{\z,0}^{(r)} }  + \norm{\tilde{\bm{\alpha}}_t^{(l)}(\x) - \tilde{\bm{\alpha}}_0^{(l)}(\z)}
%     \norm{  \prod_{r=l+1}^{L} \sqrt{\frac{2}{m_r}}  \bm{W}_0^{(r)} \bm{D}_{\z,0}^{(r)} }.\label{eq:Proof_GradientDev_Decomp2}
%   \end{align}
%   \twocolumngrid

% \begin{align}
%   \notag
%   &\norm{\nabla_{\bm{W}^{(l)}} g_t(\x) - \nabla_{\bm{W}^{(l)}} g_0(\z)}_{\mr{F}} \\
%   \notag
%   & = \left\lVert\tilde{\bm{\alpha}}_t^{(l)}(\x) \left(  \prod_{r=l+1}^{L} \sqrt{\frac{2}{m_r}}  \bm{W}_t^{(r)}  \bm{D}_{\x,t}^{(r)}  \right)\right.\\
%   &\qquad\qquad\qquad\qquad\qquad \left.- \tilde{\bm{\alpha}}_0^{(l)}(\z) \left(  \prod_{r=l+1}^{L} \sqrt{\frac{2}{m_r}}  \bm{W}_0^{(r)} \bm{D}_{\z,0}^{(r)} \right)\right\rVert_{\mr{F}} \notag\\
%   & = \norm{\tilde{\bm{\alpha}}_t^{(l)}(\x)} \norm{  \prod_{r=l+1}^{L} \sqrt{\frac{2}{m_r}}  \bm{W}_t^{(r)}  \bm{D}_{\x,t}^{(r)}   -
%   \prod_{r=l+1}^{L} \sqrt{\frac{2}{m_r}}  \bm{W}_0^{(r)} \bm{D}_{\z,0}^{(r)} } \nonumber\\
%   &\qquad + \norm{\tilde{\bm{\alpha}}_t^{(l)}(\x) - \tilde{\bm{\alpha}}_0^{(l)}(\z)}
%   \norm{  \prod_{r=l+1}^{L} \sqrt{\frac{2}{m_r}}  \bm{W}_0^{(r)} \bm{D}_{\z,0}^{(r)} }.\label{eq:Proof_GradientDev_Decomp2}
% \end{align}


% Then, \cref{lem:Init_WeightsProductBound} shows that
% with probability at least $1- \exp(-\Omega(m))$,
% \begin{align*}
%   \norm{  \prod_{r=l+1}^{L} \sqrt{\frac{2}{m_r}}  \bm{W}_0^{(r)} \bm{D}_{\z,0}^{(r)} } = O(1),
%   \quad \norm{\tilde{\bm{\alpha}}_0^{(l)}(\z)} = O(1).
% \end{align*}
% Moreover, applying \cref{lem:NN_forward_perturbation2} with $\tau  \asymp m^{1/4}$, we can control the perturbation by
% \begin{align*}
%   &\norm{\tilde{\bm{\alpha}}_t^{(l)}(\x) - \tilde{\bm{\alpha}}_0^{(l)}(\z)} = O\left( m^{-1/4}\sqrt {\ln m} \right),\\
%   &\norm{\bm{D}_{\x,t}^{(l)} - \bm{D}_{{\z},0}^{(l)}}_0 = O\left( m^{5/6} \right),
% \end{align*}
% which hold with probability at least $1-\exp(-\Omega(m^{5/6}))$.
% Combining the latter with the assumption of $\bm{W}_t^{(l)}$, by \cref{lem:backward perturbation} we get
% \begin{align*}
%     \norm{  \prod_{r=l+1}^{L} \sqrt{\frac{2}{m_r}}  \bm{W}_t^{(r)}  \bm{D}_{\x,t}^{(r)}
%       -\prod_{r=l+1}^{L} \sqrt{\frac{2}{m_r}}  \bm{W}_0^{(r)} \bm{D}_{\z,0}^{(r)} }
%   = O\left( m^{-\frac{1}{12}} \sqrt{\ln m} \right).
% \end{align*}
% Finally, plugging this estimations in \cref{eq:Proof_GradientDev_Decomp2}, we conclude
% \begin{align*}
%     \norm{\nabla_{\bm{W}^{(l)}} g_t(\x) - \nabla_{\bm{W}^{(l)}} g_0(\z)}_{\mr{F}}
%     &= O(1)\cdot O\left( m^{-\frac{1}{12}} \sqrt{\ln m} \right) + O\left( m^{-\frac{1}{4}}\sqrt {\ln m} \right) \cdot O(1) \\
%     &=  O\left( m^{-\frac{1}{12}} \sqrt{\ln m} \right).
% \end{align*}


% \begin{proof}
%   [Proof of \cref{prop:Continuity_Kt}]By applying \cref{lem:GradientBound_Deviance} with $\tau \asymp m^{1/4}$, with probability at least $1 - \exp(-\Omega(m^{5/6}))$, we can obtain the following result
%   \begin{align*}
%     \norm{ \nabla_{\bm{W}^{(l)}} g_t(\x)- \nabla_{\bm{W}^{(l)}} g_0(\z) }_{\mr{F}}
%     = O\mpt{ m^{-1/12}\sqrt{ \ln m}}
%   \end{align*}
%   when $m$ is large enough.
%   Combining this result with \cref{lem:Init_WeightsProductBound} further implies $\norm{ \nabla_{\bm{W}^{(l)}} g_t(\x)}_{\mr{F}} = O(1)$.
%   The same conclusion holds if we replace $\x$, $\z$ with $\x'$, $\z'$.
%   Furthermore, we have
%   \begin{align*}
%      \abs{K^{(p)}_t(\x,\x') - K^{(p)}_0(\z,\z')}&\leq \sum_{l=0}^L \norm{ \nabla_{\bm{W}^{(l)}} g_t(\x)}_{\mr{F}} \norm{ \nabla_{\bm{W}^{(l)}} g_t(\x')- \nabla_{\bm{W}^{(l)}} g_0(\z') }_{\mr{F}} \\
%     & \qquad\qquad + \sum_{l=0}^L  \norm{ \nabla_{\bm{W}^{(l)}} g_t(\x)- \nabla_{\bm{W}^{(l)}} g_0(\z) }_{\mr{F}}\norm{ \nabla_{\bm{W}^{(l)}} g_0(\z')}_{\mr{F}} \\
%     &\qquad = O\mpt{m^{-1/12}\sqrt{ \ln m}}
%   \end{align*}
%   holds for all $t\in T$ with probability at least $1 - \exp\mpt{-\Omega(m^{5/6})}$ when $m$ is large enough.
% \end{proof}

% Since
% \begin{align*}
%   &K^{(p)}_t(\x,\x') = \ang{\nabla_{\bm{\theta}^{(p)}} g_t(\x),\nabla_{\bm{\theta}^{(p)}} g_t(\x')}
%   \\&= \sum_{l=0}^L \ang{\nabla_{\bm{W}^{(l)}} g_t(\x), \nabla_{\bm{W}^{(l)}} g_t(\x') } + 1,
% \end{align*}
% we have
% \begin{align*}
%   &\abs{K^{(p)}_t(\x,\x')-K^{(p)}_t(\z,\z')} \\&
%   = \abs{\sum_{l=0}^L  \ang{\nabla_{\bm{W}^{(l)}} g_t(\x),  \nabla_{\bm{W}^{(l)}} g_t(\x')} -
%   \sum_{l=0}^L \ang{\nabla_{\bm{W}^{(l)}} g_t(\z),  \nabla_{\bm{W}^{(l)}} g_t(\z')}  } \\
%   & \leq \sum_{l=0}^L \abs{\ang{\nabla_{\bm{W}^{(l)}} g_t(\x),  \nabla_{\bm{W}^{(l)}} g_t(\x')} - \ang{\nabla_{\bm{W}^{(l)}} g_t(\z),  \nabla_{\bm{W}^{(l)}} g_t(\z')}} \\
%   & \leq \sum_{l=0}^L \abs{\ang{\nabla_{\bm{W}^{(l)}} g_t(\x),  \nabla_{\bm{W}^{(l)}} g_t(\x') -  \nabla_{\bm{W}^{(l)}} g_t(\z')}} \\
%   & \quad + \sum_{l=0}^L \abs{\ang{\nabla_{\bm{W}^{(l)}} g_t(\x) - \nabla_{\bm{W}^{(l)}} g_t(\z) , \nabla_{\bm{W}^{(l)}} g_t(\z')}} \\
%   & \leq \sum_{l=0}^L \norm{\nabla_{\bm{W}^{(l)}} g_t(\x) }_{\mr{F}} \norm{\nabla_{\bm{W}^{(l)}} g_t(\x') -\nabla_{\bm{W}^{(l)}} g_t(\z')}_{\mr{F}} \\
%   & \quad + \sum_{l=0}^L \norm{\nabla_{\bm{W}^{(l)}} g_t(\x) - \nabla_{\bm{W}^{(l)}} g_t(\z)}_{\mr{F}} \norm{\nabla_{\bm{W}^{(l)}} g_t(\z')}_{\mr{F}}.
% \end{align*}

% Noticing that
% \begin{align*}
%   \norm{\nabla_{\bm{W}^{(l)}} g_t(\x)}_{\mr{F}} \leq \norm{\nabla_{\bm{W}^{(l)}} g_t(\z)}_{\mr{F}} + \norm{\nabla_{\bm{W}^{(l)}} g_t(\x) -\nabla_{\bm{W}^{(l)}} g_t(\z)  }_{\mr{F}}
% \end{align*}
% and
% \begin{align*}
%   \norm{\nabla_{\bm{W}^{(l)}} g_t(\z)}_{\mr{F}} \leq \norm{\nabla_{\bm{W}^{(l)}} g_0(\z)}_{\mr{F}} + \norm{\nabla_{\bm{W}^{(l)}} g_t(\z) -\nabla_{\bm{W}^{(l)}} g_0(\z)  }_{\mr{F}},
% \end{align*}
% by \cref{lem:Init_WeightsProductBound}, \cref{lem:Training_GradientBound} ($\tau \asymp m^{1/4}$) and \cref{lem:GradientBound_Deviance},
% we know that
% \begin{align*}
%   \abs{K_t^{(p)}(\x,\x') - K_t^{(p)}(\z,\z')} = O\left(m^{-1/12} \sqrt{\ln m} \right)
% \end{align*}
% holds with probability at least $1- \exp(-\Omega(m^{5/6}))$.

\subsection{Hölder continuity of \texorpdfstring{$\NTK$}{NTK}}\label{subsec:Holder_NTK}
For convenience, let us first introduce the following definition of Hölder spaces~\citep{adams2003_SobolevSpaces}.
For an open set $\Omega \subset \R^p$ and a real number $\alpha \in [0,1]$, let us define a semi-norm for $f : \Omega \to \R$ by
\begin{align*}
  \abs{f}_{0,\alpha} = \sup_{x,y \in \Omega,~x\neq y}{\abs{f(x) - f(y)}}/{\norm{x-y}^\alpha}
\end{align*}
and define the Hölder space by $C^{0,\alpha}(\Omega) = \left\{ f \in C(\Omega) : \abs{f}_{0,\alpha} < \infty \right\},$
which is equipped with norm $\norm{f}_{C^{0,\alpha}(\Omega)} = \sup_{x \in \Omega} \abs{f(x)} + \abs{f}_{\alpha}$.
Then it is easy to show that
\begin{enumerate}[$(a)$~]
  \item $C^{0,\alpha}(\Omega) \subseteq C^{0,\beta}(\Omega)$ if $\beta \leq \alpha$;
  \item if $f,g \in C^{0,\alpha}(\Omega)$, then $f + g,~ fg \in C^{\alpha}(\Omega)$;
  \item if $f \in C^{0,\alpha}(\Omega_1)$ and $g \in C^{0,\beta}(\Omega_2)$ with $\ran g \subseteq \Omega_1$, then $f\circ g \in C^{0,\alpha\beta}(\Omega_2)$.
\end{enumerate}
Consequently, using the formula \cref{eq:NTK_Formula}, we can show the following proposition:
\begin{proposition}
  \label{prop:NTK_Continuity}
  We have $\NTK \in C^{0,s}( \tilde{B}_R \times  \tilde{B}_R)$ with $s = 2^{-L}$.
  Particularly, there is some absolute constant $C>0$ such that
  for any $x,x',z,z' \in  \tilde{B}_R$,
  \begin{align}
    \abs{\NTK(\x,\x') - \NTK(\z,\z')} \leq C R^2 \norm{(\x,\x') - (\z,\z')}^s.
  \end{align}
\end{proposition}
\begin{proof}
  Let us recall \cref{eq:NTK_Formula}.
%  % \begin{align*}
%  %   \NTK(\x,\x') =
%  %   \norm{\x}\norm{\x'} \sum_{r=0}^L \kappa^{(r)}_1(\bar{u}) \prod_{s=r}^{L-1} \kappa_0(\kappa^{(s)}_1(\bar{u})) + 1.
%  % \end{align*}
%  % where $\bar{u} =\ang{\x,\x'}/(\norm{\x}\norm{\x'})$ and
%  % \begin{align*}
%  %   \kappa_0(u) =  \frac{1}{\pi}\left( \pi - \arccos u \right),\quad
%  %   \kappa_1(u) = \frac{1}{\pi}\left[ \sqrt {1-u^2} + u (\pi - \arccos u)  \right].
%  % \end{align*}
%  \begin{align*}
%    \NTK(\x,\x') =
%    \norm{\tilde{\x}}\norm{\tilde{\x}'} \sum_{r=0}^L \kappa^{(r)}_1(\bar{u}) \prod_{s=r}^{L-1} \kappa_0\mpt{\kappa^{(s)}_1(\bar{u})} + 1.
%  \end{align*}
%  where $\tilde{\x} = (\x^T,1)^T$, $\tilde{\x}' = (\x'^T,1)^T$, $\bar{u} = \ang{\tilde{\x},\tilde{\x}'} / \pt{\norm{\tilde{\x}}\norm{\tilde{\x}'}}$ and
%  \begin{align*}
%    \begin{split}
%      % \label{eq:A_Arccos_Formula}
%      \kappa_0(u) &=  \frac{1}{\pi}\pt{\pi - \arccos u},\qquad~
%      \kappa_1(u) = \frac{1}{\pi}\bk{\sqrt {1-u^2} + u (\pi - \arccos u)},
%    \end{split}
%  \end{align*}
  Since $\NTK$ is symmetric, by triangle inequality it suffices to prove that $\NTK(\x_0,\cdot) \in C^{0,s}(\tilde{B}_R)$
  with $\abs{\NTK(\x_0,\cdot)}_{0,s}$ bounded by a constant independent of $\x_0$.
  Now, the latter is proven by
  \begin{enumerate}[$(a)$~]
    \item $\x \mapsto \bar{u} = \ang{{\tilde{\x}}/{\norm{\tilde{\x}}}, {\tilde{\x}_0}/{\norm{\tilde{\x}_0}}} \in C^{0,1}(\tilde{B}_R)$, where the bound of the Hölder
    norm is independent of $\x_0$;
    \item as functions of $\bar{u}$, both $\sqrt{1 - \bar{u}^2}$ and $\arccos \bar{u}$ belong to $C^{0,{1}/{2}}([-1,1])$, and thus
    $\kappa_0,\kappa_1 \in C^{0,{1}/{2}}([-1,1])$;
    \item the expression of NTK together with the properties of Hölder functions.
  \end{enumerate}
  % It is also easy to check that $\abs{\NTK(\x_0,\cdot)}_s$ is bounded by a constant only depending on $R$
%  Thus, we can get the conclusion.
\end{proof}

% For any $\x,\x',\z,\z'$ such that $\norm{\x-\z}_2\leq O(1/m)$, $\norm{\x'-\z'}_2\leq O(1/m)$, with probability at least $1-\delta$ with respect to the initialization, we have
% \begin{align*}
% \abs{\NTK(\x,\x')-\NTK(\z,\z')}&\leq\abs{\NTK(\x,\x')-K_0(\x,\x')}+\abs{K_0(\x,\x')-K_0(\z,\z')}\\
% &\qquad+\abs{K_0(\z,\z')-\NTK(\z,\z')}\\
% &\leq O(m^{-1/5})+O(m^{-1/12}\sqrt{\ln m}) + O(m^{-1/5})=O(m^{-1/12}\sqrt{\ln m})
% \end{align*}
% when $m\geq C_1(\ln(C_2/\delta))^5$. Since the NTK does not depend on the parameters, the above bounds hold for any $\x,\x',\z,\z'$ such that $\norm{\x-\z}_2\leq O(1/m)$, $\norm{\x'-\z'}_2\leq O(1/m)$.

\subsection{The kernel uniform convergence} \label{subsec:KernelUniformConvergence}

% \begin{theorem}[Kernel uniform convergence, Proposition 2.3 in the main text]
%   \label{thm:A_KernelUniform}
%   There exists a polynomial $\operatorname{poly}(\cdot)$ such that for any given training data $\{(\x_{i},y_{i}),i\in[n]\}$ and any $\delta\in(0,1)$, when  $m\geq \operatorname{poly}(n,\lambda_0^{-1},\|\bm{y}\|_2,\ln(1/\delta))$,  we have
%   \begin{equation*}
%     \label{eq:A_KernelUniformConvergence}
%     \sup_{t \geq 0} \sup_{\x,\x'\in \tilde{B}_R}\abs{K_t(\x,\x') - \NTK(\x,\x')} \leq O\left(m^{-\frac{1}{12}}\sqrt{\ln m}\right).
%   \end{equation*}
%   with probability at least $1-\delta$.

%   For any $\delta \in (0,1)$,
%   as long as $m \geq \mathrm{poly}\left(n,\lambda_0^{-1},\norm{\bm{y}},\ln(1/\delta)\right)$,
%   with probability at least $1-\delta$ we have
%   \begin{equation*}
%     \label{eq:A_KernelUniformConvergence}
%     \sup_{t \geq 0} \sup_{\x,\x'\in \tilde{B}_R}\abs{K_t(\x,\x') - \NTK(\x,\x')} \leq O\left(m^{-\frac{1}{12}}\sqrt{\ln m}\right).
%   \end{equation*}
%\end{theorem}

\begin{proposition}[Kernel uniform convergence]
  \label{prop:2_KernelUniform}
  Denote $B_r = \{x\in \mb{R}^{d}: 1 \leq \norm{\tilde{\x}} \leq r\}$.
  There exists a polynomial $\operatorname{poly}(\cdot)$ such that
  for any $\delta \in (0,1)$,
  as long as $m \geq \mathrm{poly}\left(n, \tilde{M}_{\bm{X}}, \lambda_0^{-1},\norm{\bm{y}},\ln(1/\delta),k\right)$ and $m \geq r^k$,
  with probability at least $1-\delta$ we have
  \begin{equation*}
    \sup_{t \geq 0} \sup_{\x,\x'\in B_r}\abs{K_t(\x,\x') - \NTK(\x,\x')} \leq O\xkm{ r^2 m^{-\frac{1}{12}}\sqrt{\ln m}}.
  \end{equation*}
\end{proposition}

\begin{proof}
  First, by \cref{lem:A_lazy_regime}, we know that there exists a polynomial $\poly_1(\cdot)$ such that for any $\delta\in(0,1)$, when $m\geq\poly_1\xkm{n, \tilde{M}_{\bm{X}},\norm{\bm{y}},\lambda_0^{-1},\ln(1/\delta)}$, then with probability at least $1-\delta/2$, for all $p\in\dk{1,2}$ and $l\in\dk{0,1,\cdots,L}$, we have
  \begin{align*}
    \sup_{t \geq 0} \norm{\bm{W}^{(l,p)}_t - \bm{W}^{(l,p)}_0}_{\mr{F}} = O(m^{1/4}).
  \end{align*}
  Next, we condition on this event happens.
  % holds with probability at least $1-\delta$ if $m\geq\mr{poly}\big(n,\lambda_0^{-1},\ln(1/\delta),\norm{\bm{y}}\big)$.
%  we know that with probability at least $1-\delta_0$,

  Since $B_r \subset \R^d$ is bounded, for any $\ep > 0$ we have an $\ep$-net $\caN_{\ep}$ (with respect to $\norm{\cdot}_2$)
  of $\tilde{B}_R$ such that the cardinality $\abs{\caN_{\ep}} = O(r^d\ep^{-d})$
  ~\citep[Section 4.2]{vershynin2018_HighdimensionalProbability}.
  Specifically, we choose $\ep = m^{-2^L}$ and thus $\ln \abs{\caN_{\ep}} = O(\ln m)$ if $m \geq r^k$ and $ m \geq \poly_3(k)$.
  Denote by
  \[B_{\z,\z'}(\varepsilon)=\dk{(\x,\x'):\norm{\x-\z}\leq\varepsilon,~\norm{\x'-\z'} \leq \ep,~\x,\x' \in\tilde{B}_R}.\]
  Then, fixing $\z,\z' \in \mc{N}_\varepsilon$, for any $(\x,\x') \in B_{\z,\z'}(\varepsilon)$, we have
  \begin{align*}
    \abs{K_t(\x,\x') - \NTK(\x,\x')}&\leq \abs{ K_t(\x,\x')  - \NTK(\z,\z') } + \abs{\NTK(\z,\z') - \NTK(\x,\x')}.
  \end{align*}
  Then, noticing that $K_t = (K_t^{(1)} + K_t^{(2)})/2$, we control the two terms on the right hand side by
  \cref{cor:Training_Kernel_uniform} and \cref{prop:NTK_Continuity} respectively,
  deriving that with probability at least $1-\delta/\xk{2|\mathcal{N}_{\varepsilon}|^2}$, for all $t\geq 0$, we have
  \begin{align*}
    % & \sup_{t\geq 0}\sup_{(\x,\x')\in B_{\z,\z'}(\varepsilon)} |K_t(\x,\x') - K_t(\z,\z')| = O\Big( m^{-1/12}\sqrt{\ln m}\Big), \\
    % & \sup_{t\geq 0}\abs{K_t(\z,\z') - K_0(\z,\z')} = O\Big( m^{-1/12}\sqrt{ \ln m}\Big), \\
    % & \abs{K_0(\z,\z') - \NTK(\z,\z')} = O(m^{-1/5}), \\
    \begin{gathered}
      \sup_{(\x,\x')\in B_{\z,\z'}(\varepsilon)} |K_t(\x,\x') - \NTK(\z,\z')| = O\xkm{r^2 m^{-1/12}\sqrt{\ln m}}, \\
      \abs{\NTK(\z,\z') - \NTK(\x,\x')} = O(r^2\ep^{2^{-L}}) = O(r^2m^{-1}),
    \end{gathered}
  \end{align*}
  if $m\geq C_1\ln\xkm{C_2|\mathcal{N}_{\varepsilon}|^2/\delta}^5$ for some positive absolute constants $C_1>0$ and $C_2\geq 1$.
  
  And there also exists a polynomial $\poly_2(\cdot)$ such that when $m\geq\poly_2\xkm{\ln(1/\delta)}$, we have $m\geq C_1\ln\xkm{C_2|\mathcal{N}_{\varepsilon}|^2/\delta}^5$, since $\ln |\mathcal{N}_{\varepsilon}| = O(\ln m)$.
  Combining these two terms, we have
  \begin{align*}
    % \label{eq:Proof_KernelUniform_Ball}
    \sup_{t \geq 0} \sup_{(\x,\x')\in B_{\z,\z'}(\ep)} \abs{K_t(\x,\x') - \NTK(\x,\x')} = O\xkm{r^2m^{-{1/12}}\sqrt{\ln m}}
  \end{align*}
  if $m\geq\poly_2(\ln(1/\delta))$.

  Combining all of the above results and applying the union bound for all pair $\bm{z},\bm{z}'\in\mathcal{N}_\varepsilon$, with probability at least $1-\delta$, we have
  \begin{align*}
    \sup_{t\geq 0}\sup_{\x,\x'\in B_r}\abs{ K_{t}(\x,\x') - \NTK(\x,\x') } = O\xkm{r^2 m^{-{1}/{12}}\sqrt{\ln m}}
  \end{align*}
  if $m\geq\poly_1\xkm{n,\tilde{M}_{\bm{X}},\norm{\bm{y}}_2,\lambda_0^{-1},\ln(1/\delta)}  + \poly_2(\ln(1/\delta)) + \poly_3(k)$.
  % with probability at least $1-\delta / \abs{\caN_\ep}^2$ if $m \geq\mathrm{poly}(\ln(4\abs{\caN_\ep}^2 /\delta))$,
  % which is satisfied for $m \geq\mathrm{poly}(\ln(1 /\delta))$ (with a large constant) since $\ln \abs{\caN_{\ep}} = O(\ln m)$.
  % Combining these four terms, we have
  % \begin{align}
  %   \label{eq:Proof_KernelUniform_Ball}
  %   \sup_{t \geq 0} \sup_{(\x,\x')\in B_{\z,\z'}(\ep)} \abs{K_t(\x,\x') - \NTK(\x,\x')} = O\left(m^{-\frac{1}{12}}\sqrt{\ln m}\right).
  % \end{align}
  % Then, applying the union bound we show that \cref{eq:Proof_KernelUniform_Ball} holds for all pairs $(\z,\z')\in \caN_\ep$ with probability at least $1-\delta$.
  % Since $\caN_{\ep}$ is an $\ep$-net of $\tilde{B}_R$, such $B_{\z,\z'}(\ep)$'s cover all $\x,\x'\in \tilde{B}_R$.
  % Consequently, we conclude that \cref{eq:A_KernelUniformConvergence} holds with probability at least $1-2\delta$
  % as long as $m\geq\mr{poly}\big(n,\lambda_0^{-1},\ln(1/\delta),\norm{\bm{y}}\big)$.
\end{proof}
% Similar to in \cite{lai2023_GeneralizationAbility}, we have the corollary:
% \begin{corollary}
%   \label{lem:UnifConverge}
%   Denote $B_r = \dk{x \in \R^d : \norm{x} \leq m^{-1/48}}$.
%   There exists a polynomial $\mathrm{poly}(\cdot)$ such that
%   for any $\delta \in (0,1)$,
%   when $m \geq \mathrm{poly}(n,\tilde{M}_{\bm{X}},\lambda_0^{-1}$, $\norm{\bm{y}},\ln(1/\delta))$,
%   with probability at least $1-\delta$ with respect to random initialization, we have
%   \begin{align*}
%     \sup_{t\geq 0} \sup_{\x \in B_r} \abs{\fNTK(\x) - \fNN(\x)} \leq O(m^{-1/24}\sqrt{\ln m}).
%   \end{align*}
% \end{corollary}
