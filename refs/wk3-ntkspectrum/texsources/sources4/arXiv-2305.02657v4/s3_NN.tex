%
%% \subsubsection{Kernel gradient flow}
%\paragraph{Kernel gradient flow}
%Since we have shown that the training process of wide neural networks can be approximated by
%a kernel regressor,
%we are particularly interested in the following
%\textit{kernel gradient flow} method~\citep{zhang2005_BoostingEarly,yao2007_EarlyStopping},
%which falls into a broader class of methods known as spectral algorithms in kernel regression
%~\citep{bauer2007_RegularizationAlgorithms}.
%
%Let us introduce the sampling operator $K_{\x} : \R \to \caH$ defined by $K_{\x} y = y k(\x,\cdot)$,
%whose adjoint $K_{\x}^* : \caH \to \R$ is given by $K_{\x}^* f = f(\x) = \ang{f,k(\x,\cdot)}_{\caH}$.
%We also define the empirical covariance operator $T_{\X} : \caH \to \caH$ by $T_{\X} = \frac{1}{n}\sum_{i=1}^n K_{\x_{i}} K_{\x_{i}}^*.$
%It can also be shown that $T_{\X}$ is self-adjoint and positive, and $\norm{T_{\X}} \leq \norm{T_{\X}}_1 \leq \kappa^2$.
%Using these notations, the empirical loss can be written as
%\begin{align}
%  \caL(f) = \frac{1}{2n}\sum_{i=1}^n (f(\x_i) - y_i)^2 = \frac{1}{2n}\sum_{i=1}^n (K_{\x_i} f - y_i)^2,
%\end{align}
%and its Fréchet derivative is shown in \citet{yao2007_EarlyStopping} to be
%\begin{align*}
%  \nabla \caL(f) = \frac{1}{n}\sum_{i=1}^n K_{\x_i} (K_{\x_i}^* f - y_i) = T_{\X} f - g_Z,
%\end{align*}
%where we define $g_Z \coloneqq \frac{1}{n} \sum_{i=1}^n K_{\x_i} y_i$.
%Therefore, we can consider the following gradient flow equation in $\caH$:
%\begin{align}
%  \label{eq:5_KGD_Flow}
%  \pdv{t}f_t = - \nabla \caL(f_t) = -(T_{\X} f_t - g_Z),
%\end{align}
%where we set the initial value $f_0 = 0$.
%The matrix form of \cref{eq:5_KGD_Flow} is
%\begin{align*}
%  \pdv{t} f_t(\x) = - k(\x,\X) (f_t(\X) - \mathbf{y}),
%\end{align*}
%which is in the same form as \cref{eq:2_NTK_GF}.
%Finally, solution $\hat{f}^{\mathrm{GF}}_t$ of equation \cref{eq:5_KGD_Flow} gives the regressor of kernel gradient flow method,
%which can be given explicitly by
%\begin{align*}
%   \hat{f}^{\mathrm{GF}}_t = \varphi^{\mathrm{GF}}_t(T_{\X}) g_Z,
%    \qq{or}
%    \hat{f}^{\mathrm{GF}}_t(\x) = k(\x,\X)\varphi^{\mathrm{GF}}_t\left(\frac{1}{n}k(\X,\X)\right) \bm{y},
%\end{align*}
%where $\varphi^{\mathrm{GF}}_t(z) \coloneqq (1-e^{-tz})/z$
%defines a class of continuous function on $z \in [0,+\infty)$ and
%$\varphi^{\mathrm{GF}}_t(T_{\X})$ (or $\varphi^{\mathrm{GF}}_t( k(\X,\X)/n)$) is understood by the spectral decomposition.

In this section, using the spectral properties of the NTK obtained in the previous section,
we derive the optimal rates of over-parameterized neural networks by combining the NTK theory and the kernel regression theory.
Let $d$ be fixed, $\caX \subseteq \R^d$ and $\mu$ be a sub-Gaussian\footnote{
That is, $\mu(\left\{ x \in \R^d : \norm{x} \geq t \right\}) \leq 2 \exp(-t^2 / C^2),~ \forall t \geq 0$ for some constant $C > 0$.
}
probability distribution supported on $\caX$ with upper bounded Riemann-integrable density.
Suppose we are given i.i.d.\ samples $(\x_1,y_1),(\x_2,y_2),\dots,(\x_n,y_n) \in \caX \times \R$ generated from the model
$y = f^*(\x) + \ep$,
where $\x \sim \mu$, $f^* : \caX \to \R$ is an unknown regression function and the independent noise $\ep$ is sub-Gaussian.

In terms of notations, we denote $\X = (\x_1,\dots,\x_n)$ and $\bm{y} = (y_1,\dots,y_n)^T$.
For a kernel function $k : \caX\times\caX \to \R$, we write $k(\x,\X) = (k(\x,\x_1),\dots,k(\x,\x_n))$
and $k(\X,\X) = \big(k(\x_i,\x_j)\big)_{n\times n}$.

\subsection{Setting of the neural network}
\label{subsec:NN_Setting}

We are interested in the following fully connected ReLU neural network $f(\x;\bm{\theta})$ with $L$-hidden layers
of widths $m_1,m_2,\dots, m_L$, where $L \geq 2$ is fixed.
The network includes bias terms on the first and the last layers.
To ensure that the final predictor corresponds to the kernel regressor, we consider a special mirrored architecture.
In detail, the network model is given by the following:
\begin{align*}
  % \label{eq:NN_Arch}
  \begin{split}
    % \bm{\alpha}^{(0,p)}(\x)&=\tilde{\bm{\alpha}}^{(0,p)}(\x)=\tilde{\x}\in\mb{R}^{d+1},~p\in[2],\\
    \bm{\alpha}^{(1,p)}(\x) & = \sqrt{\tfrac{2}{m_{1}}} \sigma\xkm{\bm{A}^{(p)}\x+\bm{b}^{(0,p)}}\in\mb{R}^{m_1},~ p\in\dk{1,2},\\
    \bm{\alpha}^{(l,p)}(\x) & = \sqrt{\tfrac{2}{m_{l}}}\sigma\xkm{\bm{W}^{(l-1,p)}\bm{\alpha}^{(l-1,p)}(\x)}\in\mb{R}^{m_{l}},~
    l \in \dk{2,3,\dots,L},~p\in\dk{1,2},\\
    g^{(p)}(\x;\bm{\theta}) & = \bm{W}^{(L,p)}\bm{\alpha}^{(L,p)}(\x)+b^{(L,p)}\in\mb{R},~p\in\dk{1,2},\\
    f(\x;\bm{\theta}) &= \frac{\sqrt{2}}{2}\zk{ g^{(1)}(\x;\bm{\theta}) - g^{(2)}(\x;\bm{\theta}) }\in\mb{R}.
  \end{split}
\end{align*}
% \begin{equation}
% \label{eq:NN_Arch}
% \begin{aligned}
% \bm{\alpha}^{(1,p)}(\x) &= \sqrt {\frac{2}{m_1}} \pt{\bm{A}^{(p)} \x + \bm{b}^{(0,p)}} \in \R^{m_1},\quad p =1,2 \\
% \bm{\alpha}^{(l+1,p)}(\x) &= \sqrt {\frac{2}{m_{l+1}}} \bm{W}^{(l,p)} \sigma\mpt{\bm{\alpha}^{(l,p)}(\x)}  \in \R^{m_{l+1}}, \quad~l =1,2,\dots,L-1,~p =1,2 \\
% g^{(p)}(\x;\bm{\theta}) &= \bm{W}^{(L,p)} \sigma\mpt{\bm{\alpha}^{(L,p)}(\x)} + b^{(L,p)}\in\mb{R},\quad p=1,2 \\
% f(\x;\bm{\theta}) &= \frac{\sqrt{2}}{2}\mbk{g^{(1)}(\x;\bm{\theta}) - g^{(2)}(\x;\bm{\theta})}\in\mb{R}
% \end{aligned}
% \end{equation}
Here, $\bm{\alpha}^{(l,p)}$ represents the hidden layers; $l\in\dk{1,2,\dots,L}$, $p\in\dk{1,2}$ stand for the index of layers and parity respectively;
$\sigma(x) \coloneqq \max(x,0)$ is the ReLU activation (applied elementwise);
parameters $\bm{A}^{(p)}\in\mb{R}^{m_1\times d}$, $\bm{W}^{(l,p)} \in \R^{m_{l+1} \times m_l}$, $\bm{b}^{(0,p)} \in \R^{m_{1}}$, $b^{(L,p)} \in \R$,
where we set $m_{L+1}=1$;
and we use $\bm{\theta}$ to represent the collection of all parameters flatten as a column vector.
Letting $m = \min(m_1,m_2,\dots,m_{L})$, we assume that $\max(m_1,m_2,\dots,m_{L}) \leq C_{\mr{width}} m$ for some constant $C_{\mr{width}}$.

%\begin{figure*}[t]%
%  \centering
%  \usetikzlibrary{positioning}

\newcommand{\myfontsize}{\fontsize{8pt}{12pt}\selectfont}

\begin{tikzpicture}[x=1.8cm, y=1.25cm, >=stealth,scale=0.85]
\centering
		% input layer nodes
		\foreach \i in {1, ..., 3} {
			\node[circle, draw=black, fill=white, minimum size=1cm] (x\i) at (0, -\i+2.75) {$x_{\i}$};}
		\node at (0, -1) {$\vdots$};
		\node[circle, draw=black, fill=white, minimum size=1cm] (x4) at (0, -1.75) {$x_d$};

		
		% hidden layer 1 nodes
		\foreach \i in {1, ..., 2} {
			\node[circle, draw=black, fill=white, minimum size=1cm] (a11\i) at (2, -\i+4) {};
			\node at (2, -\i+4) [font=\myfontsize] {$\boldsymbol{\alpha}^{(1,1)}_{\i}$};}
     	\node at (2, 1.25) {$\vdots$};
		\node[circle, draw=black, fill=white, minimum size=1cm] (a113) at (2, 0.5) {};
	    \node at (2, 0.5) [font=\myfontsize] {$\boldsymbol{\alpha}^{(1,1)}_{m_1}$};
	
	    \foreach \i in {1, ..., 2} {
		    \node[circle, draw=black, fill=white, minimum size=1cm] (a12\i) at (2, -\i+3.5-3){};
		    \node at (2, -\i+3.5-3) [font=\myfontsize] {$\boldsymbol{\alpha}^{(1,2)}_{\i}$};} 
         \node[circle, draw=black, fill=white, minimum size=1cm] (a123) at (2, -3){};
         \node at (2, -2.25) {$\vdots$};
         \node at (2, -3) [font=\myfontsize] {$\boldsymbol{\alpha}^{(1,2)}_{m_1}$};
		
		
		% other hidden layers
		\foreach \i in {1, ..., 2} {
			\node[circle, draw=black, fill=white, minimum size=1cm] (a31\i) at (4, -\i+4) {};
			%\node at (4.5, -\i+4) [font=\myfontsize] {$\boldsymbol{\alpha}^{(i,1)}_{\i}$};
		}
		\node at (4, 1.25) {$\vdots$};
		\node[circle, draw=black, fill=white, minimum size=1cm] (a313) at (4, 0.5) {};
		%\node at (4.5, 0.5) [font=\myfontsize] {$\boldsymbol{\alpha}^{(3,1)}_{m_i}$};
		
		\foreach \i in {1, ..., 2} {
			\node[circle, draw=black, fill=white, minimum size=1cm] (a32\i) at (4, -\i+3.5-3){};
			%\node at (4.5, -\i+3.5-3) [font=\myfontsize] {$\boldsymbol{\alpha}^{(i,2)}_{\i}$};
		} 
		\node[circle, draw=black, fill=white, minimum size=1cm] (a323) at (4, -3){};
		\node at (4, -2.25) {$\vdots$};
		%\node at (4.5, -3) [font=\myfontsize] {$\boldsymbol{\alpha}^{(i,2)}_{m_i}$};
	
		% hidden layer L nodes
	    \foreach \i in {1, ..., 2} {
	    	\node[circle, draw=black, fill=white, minimum size=1cm] (aL1\i) at (6, -\i+4) {};
	    	\node at (6, -\i+4) [font=\myfontsize] {$\boldsymbol{\alpha}^{(L,1)}_{\i}$};}
	    \node at (6, 1.25) {$\vdots$};
	    \node[circle, draw=black, fill=white, minimum size=1cm] (aL13) at (6, 0.5) {};
	    \node at (6, 0.5) [font=\myfontsize] {$\boldsymbol{\alpha}^{(L,1)}_{m_L}$};
	    
	    \foreach \i in {1, ..., 2} {
	    	\node[circle, draw=black, fill=white, minimum size=1cm] (aL2\i) at (6, -\i+3.5-3){};
	    	\node at (6, -\i+3.5-3) [font=\myfontsize] {$\boldsymbol{\alpha}^{(L,2)}_{\i}$};} 
	    \node[circle, draw=black, fill=white, minimum size=1cm] (aL23) at (6, -3){};
	    \node at (6, -2.25) {$\vdots$};
	    \node at (6, -3) [font=\myfontsize] {$\boldsymbol{\alpha}^{(L,2)}_{m_L}$};
	
	    
		
		% hidden layer 5 nodes
		\node[circle, draw=black, fill=white, minimum size=1cm] (f1) at (7.5, 1.75){$g^{(1)}$};
		\node[circle, draw=black, fill=white, minimum size=1cm] (f2) at (7.5, -1.75){$g^{(2)}$};
		
		% output layer node
		\node[circle, draw=black, fill=white, minimum size=1cm] (o) at (8.5, 0) {$f$};
		
		% connections
		\foreach \i in {1, ..., 4} {
			\foreach \j in {1, ..., 3} {
				\draw (x\i) -- (a11\j);
				\draw (x\i) -- (a12\j);
			}
		}
		
	
	   	% new connection between hidden layer 1, 2 and 3
	    \foreach \i in {1,2,3} {
	   	    \foreach \j in {1,2,3} {
	   		    \draw (a11\i) -- (a31\j);
	   		    \draw (a12\i) -- (a32\j);
	   		    \draw (a31\i) -- (aL1\j);
	   		    \draw (a32\i) -- (aL2\j);
	   	    }
	   }
  
	   
		
		% connections between hidden layer 3 and 4
		\foreach \j in {1, ..., 3}{
	           \draw (aL1\j) -- (f1);\draw (aL2\j) -- (f2);}
           
        \draw (f1) -- (o);
        \draw (f2) -- (o);

% labels for layers
\node[above=0.2cm of a111] {Hidden layer 1};
\node[above=0.2cm of a311] {$\cdots$};
\node[above=0.2cm of aL11] {Hidden layer $L$};
%\node[above=0.5cm of f1] {Hidden layer $L+1$};
\node[above=0.2cm of x1] {Input};
\node[above=0.2cm of o] {Output};
\end{tikzpicture}
%  \caption{An illustration of the mirrored architecture considered in this paper.}
%  \label{fig:NN}
%\end{figure*}

% \begin{figure*}[t]
%  \centering
%  % \usetikzlibrary{positioning}

\newcommand{\myfontsize}{11\fontsize{8pt}{12pt}\selectfont}
\newcommand{\minimumsize}{0.8cm}
\newcommand{\upsize}{0.65}

\begin{tikzpicture}[x=1.95cm, y=1.2cm, >=stealth]
\centering
		% % input layer nodes
		% \foreach \i in {1, ..., 3} {
		% 	\node[circle, draw=black, fill=white, minimum size=\minimumsize] (x\i) at (0, -\i+2.75) {};}
		% \node at (0, -1) {$\vdots$};
		% \node[circle, draw=black, fill=white, minimum size=\minimumsize] (x4) at (0, -1.75) {};
  
        % input layer nodes
		\foreach \i in {1, ..., 3} {
			\node[circle, draw=black, fill=white, minimum size=1cm] (x\i) at (0, -\i+2.75) {$x_{\i}$};}
		\node at (0, -1) {$\vdots$};
		\node[circle, draw=black, fill=white, minimum size=1cm] (x4) at (0, -1.75) {$x_d$};
		
		% hidden layer 1 nodes
		\foreach \i in {1, ..., 2} {
			\node[circle, draw=black, fill=white, minimum size=\minimumsize] (a11\i) at (1.5, -\i+3.5+\upsize) {};}
     	\node at (1.5, 0.75+\upsize) {$\vdots$};
		\node[circle, draw=black, fill=white, minimum size=\minimumsize] (a113) at (1.5, \upsize) {};
	
	    \foreach \i in {1, ..., 2} {
		    \node[circle, draw=black, fill=white, minimum size=\minimumsize] (a12\i) at (1.5, -\i+1-\upsize){};} 
         \node[circle, draw=black, fill=white, minimum size=\minimumsize] (a123) at (1.5, -2.5-\upsize){};
         \node at (1.5, -1.75-\upsize) {$\vdots$};
		
		
		% other hidden layers
		\foreach \i in {1, ..., 2} {
			\node[circle, draw=black, fill=white, minimum size=\minimumsize] (a31\i) at (3, -\i+3.5+\upsize) {};		}
		\node at (3, 0.75+\upsize) {$\vdots$};
		\node[circle, draw=black, fill=white, minimum size=\minimumsize] (a313) at (3, \upsize) {};
		
		\foreach \i in {1, ..., 2} {
			\node[circle, draw=black, fill=white, minimum size=\minimumsize] (a32\i) at (3, -\i+1-\upsize){};	} 
		\node[circle, draw=black, fill=white, minimum size=\minimumsize] (a323) at (3, -3-\upsize+0.5){};
		\node at (3, -1.75-\upsize) {$\vdots$};
	
		% hidden layer L nodes
	    \foreach \i in {1, ..., 2} {
	    	\node[circle, draw=black, fill=white, minimum size=\minimumsize] (aL1\i) at (4.5, -\i+3.5+\upsize) {};}
	    \node at (4.5, 0.75+\upsize) {$\vdots$};
	    \node[circle, draw=black, fill=white, minimum size=\minimumsize] (aL13) at (4.5, \upsize) {};
	    
	    \foreach \i in {1, ..., 2} {
	    	\node[circle, draw=black, fill=white, minimum size=\minimumsize] (aL2\i) at (4.5, -\i+1-\upsize){};} 
	    \node[circle, draw=black, fill=white, minimum size=\minimumsize] (aL23) at (4.5, -2.5-\upsize){};
	    \node at (4.5, -1.75-\upsize) {$\vdots$};

	
	    
		
		% hidden layer 5 nodes
		\node[circle, draw=black, fill=white, minimum size=\minimumsize] (f1) at (6, 1.25+\upsize){$f^{(1)}$};
		\node[circle, draw=black, fill=white, minimum size=\minimumsize] (f2) at (6, -1.25-\upsize){$f^{(2)}$};
		
		% output layer node
		\node[circle, draw=black, fill=white, minimum size=\minimumsize] (o) at (7, 0) {$y$};
		
		% connections
		\foreach \i in {1, ..., 4} {
			\foreach \j in {1, ..., 3} {
				\draw (x\i) -- (a11\j);
				\draw (x\i) -- (a12\j);
			}
		}
		
	
	   	% new connection between hidden layer 1, 2 and 3
	    \foreach \i in {1,2,3} {
	   	    \foreach \j in {1,2,3} {
	   		    \draw (a11\i) -- (a31\j);
	   		    \draw (a12\i) -- (a32\j);
	   		    \draw (a31\i) -- (aL1\j);
	   		    \draw (a32\i) -- (aL2\j);
	   	    }
	   }
  
	   
		
		% connections between hidden layer 3 and 4
		\foreach \j in {1, ..., 3}{
	           \draw (aL1\j) -- (f1);\draw (aL2\j) -- (f2);}
           
        \draw (f1) -- (o);
        \draw (f2) -- (o);

% labels for layers
\node[above=0.5cm of a111] {Hidden layer 1};
\node[above=0cm of a111] {$\bm{\alpha}^{(1,1)}$};
\node[above=0cm of a121] {$\bm{\alpha}^{(1,2)}$};
\node[above=0.5cm of a311] {$\cdots$};
\node[above=0.5cm of aL11] {Hidden layer $L$};
\node[above=0cm of aL11] {$\bm{\alpha}^{(L,1)}$};
\node[above=0cm of aL21] {$\bm{\alpha}^{(L,2)}$};
\node[above=0cm of x1] {Input};
\node[above=0cm of o] {Output};
\end{tikzpicture}
%  \includegraphics[width=\textwidth]{fig/tikz/NN_mirror.pdf}
%  \caption{An illustration of the mirrored architecture considered in this paper.}
%  \label{fig:NN}
% \end{figure*}


%\xmark{A figure is needed to illustrate the architecture}

% \subsection{Initialization}

\paragraph{Initialization}
Considering the mirrored architecture, we initialize the parameters in one parity to be i.i.d.\ normal and set the parameters in the other parity be the same as the corresponding ones.
More precisely,
\begin{align*}
  % \label{eq:NN_Init}
  \begin{aligned}
    &\bm{A}^{(1)}_{i,j},\bm{W}^{(l,1)}_{i,j},\bm{b}^{(0,1)}_i,b^{(L,1)} \iid N(0,1),\quad \text{for}~l=0,1,\dots,L, \\
    & \bm{W}^{(l,2)} = \bm{W}^{(l,1)},\quad\bm{A}^{(2)}=\bm{A}^{(1)},\quad\bm{b}^{(0,2)} = \bm{b}^{(0,1)}, \quad b^{(L,2)} = b^{(L,1)}.
  \end{aligned}
\end{align*}
Such kind of ``mirror initialization'' ensures that the model output is always zero at initialization,
which is also considered in \citet{lai2023_GeneralizationAbility}.

\paragraph{Training}
Neural networks are often trained by the gradient descent (or its variants) with respect to the empirical loss
$\mathcal{L}(\bm{\theta}) = \frac{1}{2n}\sum_{i=1}^n (f(x_i;\bm{\theta}) - y_i)^2$.
For simplicity, we consider the continuous version of gradient descent, namely the gradient flow for the training process.
Denote by $\bm{\theta}_t$ the parameter at the time $t \geq 0$, the gradient flow is given by
\begin{align}
  \label{eq:2_GD}
  \dot{\bm{\theta}}_t = - \nabla_{\bm{\theta}} \mathcal{L}(\bm{\theta}_t)= - \frac{1}{n} \nabla_{\bm{\theta}} f(\X;\bm{\theta}_t) (f(\X;\bm{\theta}_t) - \bm{y})
\end{align}
where $f(\X;\bm{\theta}_t) = (f(x_1;\bm{\theta}_t),\dots,f(x_n;\bm{\theta}_t))^T$ and $\nabla_{\bm{\theta}} f(\X;\bm{\theta}_t)$ is an $M \times n$
matrix where $M$ is the number of parameters.
Finally, let us denote by $\fNN(\x) \coloneqq f(\x;\bm{\theta}_t)$
the resulting neural network predictor.

\subsection{Uniform convergence to kernel regression}\label{subsec:UniformConv}

Although the gradient flow \cref{eq:2_GD} is a highly non-linear and hard to analyze,
the celebrated neural tangent kernel (NTK) theory~\citep{jacot2018_NeuralTangent} provides a way to approximate the gradient flow by a kernel regressor
when the width of the network tends to infinity,
which is also referred to as the \textit{lazy training regime}.
Introducing a random kernel function $K_t(\x,\x') = \ang{\nabla_{\bm{\theta}} f(\x;\bm{\theta}_t),\nabla_{\bm{\theta}} f(\x';\bm{\theta}_t)}$,
it is shown that $K_t(\x,\x')$ concentrates in probability to a deterministic kernel $\NTK$ called the neural tangent kernel (NTK).
Consequently, the predictor $\fNN(\x)$ is well approximated by the kernel regressor $\fNTK(\x)$ given by the following gradient flow:
\begin{align}
  \label{eq:2_NTK_GF}
  \dv{t} \fNTK(\x) = - \NTK(\x,\X) (\fNTK(\X) - \bm{y}),
\end{align}
where $\fNTK(\X) = (\fNTK(x_1),\dots,\fNTK(x_n))^T$.
Thanks to the mirrored architecture, we have $\fNN(\x) \equiv 0$ at initialization and thus we also have $\fNTK(\x) \equiv 0$.
The recursive formula of the NTK also enables us to give explicitly~\citep{jacot2018_NeuralTangent,bietti2020_DeepEquals}
the formula of $\NTK$ in \cref{eq:NTK_Formula}.

%It is clear that the gradient flow equation \cref{eq:2_GD} is a highly non-linear differential equation and hard to solve explicitly.
%After introduced a time-varying kernel function
%\begin{align}
%  \label{eq:Kt_Formula}
%  K_t(\x,\x') = \ang{\nabla_{\bm{\theta}} f(\x;\bm{\theta}_t),\nabla_{\bm{\theta}} f(\x';\bm{\theta}_t)}
%\end{align}
%which we called the neural network kernel(NNK) in this paper,
%\cite{jacot2018_NeuralTangent} observed that
%\begin{align}
%  \label{eq:2_NN_GF}
%  \dot{f}(\x;\bm{\theta}_t) = - \frac{1}{n} K_t(\x,\X)\left( f(\X;\bm{\theta}_t) - \bm{y} \right),
%\end{align}
%where $K_t(\x,\X) =  \nabla_{\bm{\theta}} f(\x;\bm{\theta}_t) \left[ \nabla_{\bm{\theta}} f(\X;\bm{\theta}_t) \right]^T$
%is a $1\times n$ vector.
%It might be the \eqref{eq:2_NN_GF} motivates them to introduce the neural tangent kernel.
%Moreover, \cite{jacot2018_NeuralTangent} showed that with the random Gaussian initialization,
%$K_t(\x,\x')$ concentrates to a time-invariant kernel called the neural tangent kernel (NTK)~\citep{jacot2018_NeuralTangent}:
%% which is given by
%\begin{align*}
%  % \label{eq:NTK_Def}
%  K_{t}(\x,\x') \overset{\mb{P}}{\rightarrow} \NTK(\x,\x')  \qq{as} {m \to \infty},
%  % \NTK(\x,\x') \overset{p}{=} \lim_{m \to \infty}K_{t}(\x,\x')
%  %\ang{\nabla_{\bm{\theta}} f(\x;\bm{\theta}), ~ \nabla_{\bm{\theta}} f(\x';\bm{\theta})}
%\end{align*}
%where $\overset{\mb{P}}{\rightarrow}$ stands for converging in probability.
%%Moreover, thanks to the overparametrization, $K_t(\x,\x')$ is also stable during the training.
%Thus, they considered the
%kernel regressor $\fNTK(\x)$ given by the following gradient flow%~\citep{lee2019_WideNeural}
%\begin{align}
%  \label{eq:2_NTK_GF}
%  \pdv{t} \fNTK(\x) = - \NTK(\x,\X) (\fNTK(\X) - \bm{y})
%\end{align}
%and illustrated that if both the equations \cref{eq:2_NN_GF} and \cref{eq:2_NTK_GF} are starting from zeros, then $\hat f^{\mathrm{NN}}_{t}(\x)$ is well approximated by $\hat f^{\mathrm{NTK}}_{t}(\x)$.
%Since most studies of \cref{eq:2_NN_GF} assumed that $\hat{f}^{\mathrm{NTK}}_0(\x) \equiv 0$,  we adopted
%the mirror initialization so that $\hat{f}^{\mathrm{NN}}_0(\x) \equiv 0$.

%Furthermore, \citet{jacot2018_NeuralTangent} also provided an explicit recursive formula of the NTK. %\citep{jacot2018_NeuralTangent,allen-zhu2019_ConvergenceTheory}.
%Let us define
%\begin{align*}
% % \label{eq:NTK_DefRecur}
%  \begin{aligned}
%    \Sigma_0(\x,\x') &= N_0(\x,\x') = \ang{\x,\x'} + 1
%  \end{aligned}
%\end{align*}
%and for $l=1,\dots,L$,
%\begin{align}
%  \label{eq:NTK_DefRecur}
%  &\begin{aligned}
%  \Sigma_{l}(\x,\x') &= 2 \E_{(u,v)}\left[ \sigma(u) \sigma(v) \right], \\
%     N_{l}(\x,\x') &= \Sigma_{l}(\x,\x') + 2 N_{l-1}(\x,\x') \E_{(u,v)}\left[ \dot{\sigma}(u) \dot{\sigma}(v) \right],
%  \end{aligned}
%\end{align}
%where  $(u,v)\sim N(\bm{0},\bm{B}_{l-1}(\x,\x'))$, the matrix $\bm{B}_l(\x,\x')$ is defined as:
%\[\bm{B}_l(\x,\x') =
%\begin{pmatrix}
%  \Sigma_l(\x,\x)  & \Sigma_l(\x,\x')  \\
%  \Sigma_l(\x,\x') & \Sigma_l(\x',\x')
%\end{pmatrix}\]
%and $\dot{\sigma}(x) = \bm{1}_{\{x > 0\}}$ is the weak derivative of $\sigma(x)$.
%The NTK associated with the network defined in this paper can be explicitly written as
%\begin{align}
%  \label{eq:NTK_Def0}
%  &\NTK(\x,\x') = N_L(\x,\x') + 1.
%\end{align}


Although previous works
~\citep{lee2019_WideNeural,arora2019_ExactComputation,allen-zhu2019_ConvergenceTheory}
showed that the neural network regressor $\fNN(\x)$ can be approximated by $\fNTK(\x)$,
most of these results are established pointwisely, namely, for fixed $\x$,
$\sup_{t\geq 0}\abs{\fNTK(\x)-\fNN(\x)}$ is small with high probability.
However,
to analyze the generalization performance of $\fNN(\x)$, the convergence is further needed to be uniform over $x \in \caX$.
Consider the simple case of two-layer neural network, \citet{lai2023_GeneralizationAbility} rigorously showed such uniform convergence.
With more complicated analysis, we prove the uniform convergence of $\fNN(\x)$ to $\fNTK(\x)$ for multilayer neural networks.
To state our result, let us denote by $\lambda_0 = \lambda_{\min}\left( \NTK(\X,\X) \right)$ the minimal eigenvalue of the kernel matrix,
which, by \cref{prop:NTK_PD}, can be assumed to be positive in the following.
%which $\lambda_0$
%It will be shown by \cref{thm:NTK_PD} that $\NTK$ is positive definite and thus $\lambda_0 > 0$ almost surely.
%Therefore, we will assume that $\lambda_0 > 0$ in the following.
% Let us denote
% \begin{align}
%   \label{eq:2_KernelMatrix}
%   \K_{\mathrm{NT}}(\x,\X) \coloneqq (k(\x,x_1),\dots,k(\x,x_n)),
% %  \quad K_{\mathrm{NT}} \coloneqq \frac{1}{n}\Big(k(x_i,x_j)\Big)_{n\times n}
% \end{align}
% for $k(\x,y) = \NTK(\x,y)$.
% Consider the kernel regressor $\fNTK(\x)$ define by following flow
% \begin{align}
%   \label{eq:2_NTK_GF}
%   \pdv{t} \fNTK(\x) = - \K_{\mathrm{NT}}(\x,\X) (\hat{f}^{\NTK}_t(\X) - \mathbf{y})
% \end{align}
% with initialization $\hat{f}^{\mathrm{NTK}}_0(\x) = 0$.

\begin{lemma}
  \label{lem:UnifConverge}
  Denote $M_{\bm{X}} = \sum_{i=1}^n \norm{x_i}_2$ and $B_r = \dk{x \in \R^d : \norm{x} \leq r}$ for $r \geq 1$.
  There exists a polynomial $\mathrm{poly}(\cdot)$ such that
  for any $\delta \in (0,1)$ and $k > 0$,
  when $m \geq \mathrm{poly}(n,M_{\bm{X}},\lambda_0^{-1},$ $\norm{\bm{y}}, \ln(1/\delta), k)$ and $m \geq r^k$,
  with probability at least $1-\delta$ with respect to random initialization, we have
  \begin{align*}
    \sup_{t\geq 0} \sup_{\x \in B_r} \abs{\fNTK(\x) - \fNN(\x)} \leq O(r^2 m^{-\frac{1}{12}} \sqrt {\ln m}).
  \end{align*}
\end{lemma}


\cref{lem:UnifConverge} shows that as $m$ tends to infinity, $\fNN(\x)$ can be approximated uniformly by $\fNTK(\x)$ on a bounded set, which is also allowed to grow with $m$.
Consequently, we can study the generalization performance of the neural network in the lazy training regime by that of the corresponding kernel regressor.

%\begin{corollary}
%  \label{cor:UnifConvergence}
%  There exists a polynomial $\operatorname{poly}(\cdot)$ such that
%  for any $\ep > 0$ and $\delta \in (0,1)$,
%  when $m \geq \mathrm{poly}(n,\lambda_0^{-1},\norm{\bm{y}},\ln(1/\delta), 1/\ep)$,
%  one has that
%  \begin{align*}
%    \sup_{t \geq 0}\abs{\norm{\fNN - f^*}_{L^2} - \norm{\fNTK - f^*}_{L^2}}
%    \leq \ep
%  \end{align*}
%  holds with probability at least $1-\delta$ with respect to random initialization.
%\end{corollary}

To establish \cref{lem:UnifConverge},
it is essential to demonstrate the uniform convergence of the kernel $K_t(\x,\x')$ towards $\NTK(\x,\x')$.
This is achieved by first establishing the Hölder continuity of $K_t(\x,\x')$ and $\NTK(\x,\x')$,
and then applying an $\epsilon$-net argument in conjunction with the pointwise convergence.
Since the detailed proof is laborious, it is deferred to \cref{sec:A_NN}.



\subsection{The optimal rates of the over-parameterized neural network}



With the uniform convergence of the neural network to the kernel regressor established and the eigenvalue decay rate of the NTK determined,
we can now derive the optimal rates of the over-parameterized neural network.
Let us denote by $\caH = \caH_{\mathrm{NTK}}$ the RKHS associated with the NTK \cref{eq:NTK_Formula} on $\caX$.
We introduce the integral operator $T$ in \cref{eq:T_Def} and recall its spectral decomposition in \cref{eq:MercerDecomp}.
The kernel regression literature often introduce the interpolation spaces of the RKHS to characterize the regularity of the regression function
~\citep{steinwart2012_MercerTheorem,fischer2020_SobolevNorm}.
For $s \geq 0$, we define the interpolation space $[\caH]^s$ by
\begin{align}
  \zk{\caH}^s = \left\{ \sum_{i =1}^\infty a_i \lambda_i^{s/2} e_i ~\Big|~ \sum_{i =1}^\infty a_i^2 < \infty \right\}
  \subseteq L^2,
\end{align}
which is equipped with the norm $\norm{\sum_{i =1}^\infty a_i \lambda_i^{s/2} e_i}_{[\caH]^s} \coloneqq \left(  \sum_{i =1}^\infty a_i^2  \right)^{1/2}$.
It can be seen that $[\caH]^s$ is a separable Hilbert space with $\xk{\lambda_i^{s/2} e_i}_{i \geq 1}$ as its orthonormal basis.
We also have $[\caH]^0 = L^2$ and $[\caH]^1 = \caH$.
Moreover, when $s \in (0,1)$,
the space $[\caH]^s$ also coincides with the space $(L^2,\caH)_{s,2}$ defined by real interpolation~\citep{steinwart2012_MercerTheorem}.
%which provides us with
We also denote by $B_R([\caH]^s) = \left\{ f \in [\caH]^s \mid \norm{f}_{[\caH]^s}^2 \leq R \right\}$.
Then, we derive the following optimal rates of the neural network from the optimality result in the kernel regression~\citep{lin2018_OptimalRates}.

\begin{proposition}
  \label{prop:NN_Gen}
  Suppose $f^* \in B_R([\caH]^s) \cap L^\infty$ for constants $s > \frac{1}{d+1}$ and $R > 0$.
  Let us choose $t_{\mathrm{op}} = t_{\mathrm{op}}(n) \asymp n^{(d+1)/ [s(d+1)+d]}$.
  Then, there exists a polynomial $\operatorname{poly}(\cdot)$ such that for any $\delta \in (0,1)$,
  when $n$ is sufficiently large and the width $m \geq \mathrm{poly}(n,\ln(1/\delta),\lambda_{0}^{-1})$,
  with probability at least $1-\delta$ with respect to random samples and random initialization,
  \begin{align}
    \label{eq:NN_Rate}
    \norm{\hat{f}^{\mathrm{NN}}_{t_{\mathrm{op}}} - f^*}_{L^2}^2 \leq C \left( \ln\frac{12}{\delta} \right)^2 n^{-\frac{s(d+1)}{s(d+1)+d}},
  \end{align}
  where the constant $C>0$ is independent of $\delta,n$.
  Moreover, the convergence rate in \cref{eq:NN_Rate} achieves the optimal rate in $B_R([\caH]^s)$.
\end{proposition}

The results in the kernel regression literature also allows us to provide the following sup-norm learning rate.

\begin{proposition}
  \label{prop:NN_Sup_Gen}
  Under the settings of \cref{prop:NN_Gen}, suppose further that $s \geq 1$ and $\caX$ is bounded.
  Then, when $n$ is sufficiently large,  with probability at least $1-\delta$,
  \begin{align*}
    \norm{\hat{f}^{\mathrm{NN}}_{t_{\mathrm{op}}} - f^*}_{\infty}^2 \leq C \left( \ln\frac{12}{\delta} \right)^2 n^{-\frac{(s-1)(d+1)}{s(d+1)+d}},
  \end{align*}
  where the constant $C>0$ is independent of $\delta,n$.
\end{proposition}


\begin{remark}
  \cref{prop:NN_Gen} shows the minimax optimality of wide neural networks,
  where optimal rate is also adaptive to the relative smoothness of the regression function to the NTK\@.
%  The rate increases as the smoothness index $s$ increases, approaching $n^{-1}$.
  Our result extends the result in \citet{lai2023_GeneralizationAbility} to the scenario of $d > 1$ and $L > 1$,
  and also distinguishes with \citet{hu2021_RegularizationMatters,suh2022_NonparametricRegression} in the following aspects:
  (1) The critical uniform convergence (\cref{lem:UnifConverge}) is not well-supported in these two works, as pointed out in \citet{lai2023_GeneralizationAbility};
  (2) They have to assume the data distribution is uniform on the sphere, while we allow $\caX$ to be a general domain;
  (3) They introduce an explicit $\ell_2$ regularization in the gradient descent and approximate the training dynamics by kernel ridge regression (KRR),
  while we consider directly the kernel gradient flow and early stopping serves as an implicit regularization, which is more natural.
  Moreover, our gradient method can adapt to higher order smoothness of the regression function and do not
  saturate as KRR~\citep{li2023_SaturationEffect} or consequently their $\ell_2$-regularized neural networks.
\end{remark}

Moreover, using the idea in \citet{caponnetto2010_CrossvalidationBased},
we can also show that the cross validation can be used to choose the optimal stopping time.
Let us further assume that $\supp \mu$ is bounded and $y \in [-M,M]$ almost surely for some $M$ and introduce the truncation $L_{M}(a)=\min\{\abs{a},M\}\operatorname{sgn}(a)$.
Suppose now we have $\tilde{n}$ extra independent samples $(\tilde{\x}_1,\tilde{y}_1),\dots,(\tilde{\x}_{\tilde{n}},\tilde{y}_{\tilde{n}})$,
where $\tilde{n} \geq c_{\mathrm{v}} n$ for some constant $c_{\mathrm{v}} > 0$.
Let $T_n$ be a set of stopping time candidates, we can choose the empirical stopping time by cross validation
\begin{align}
  \label{eq:5_StoppingTimeCV}
  \hat{t}_{\mathrm{cv}} = \argmin_{t \in T_n}
  \sum_{i=1}^{\tilde{n}} \left[ L_{M}\xk{\fNN(\tilde{\x}_i)} - \tilde{y}_i \right]^2.
\end{align}

\begin{proposition}
%[Empirical stopping time]
  \label{prop:NN_CV}
  Under the settings of \cref{prop:NN_Gen} and the further assumptions given above,
  let $T_n = \left\{ 1, Q, \dots,Q^{\lfloor \ln_Q n \rfloor} \right\}$ for arbitrary fixed $Q > 1$ and $\hat{t}_{\mathrm{cv}}$ be chosen from \cref{eq:5_StoppingTimeCV}.
  Define $\hat{f}^{\mathrm{NN}}_{\mr{cv}}(x) = L_{M}\xk{\hat{f}^{\mathrm{NN}}_{\hat{t}_{\mathrm{cv}}}(x)}$.
  Then, there exists a polynomial $\operatorname{poly}(\cdot)$
  such that when $n$ is sufficiently large and $m \geq \mathrm{poly}(n,\ln(1/\delta),\lambda_{0}^{-1})$,
  one has
  \begin{align*}
    \norm{\hat{f}^{\mathrm{NN}}_{\mr{cv}} - f^*}_{L^2}^2 \leq C \left( \ln\frac{12}{\delta} \right)^2 n^{-\frac{s(d+1)}{s(d+1)+d}}
  \end{align*}
  with probability at least $1-\delta$ with respect to random samples and initialization,
  where the constant $C>0$ is independent of $\delta,n$.
\end{proposition}

Early stopping, as an implicit regularization, is necessary for the generalization of neural networks.
The following proposition, which is a consequence of the result in \citet{li2023_KernelInterpolation},
shows overfitted multilayer neural networks generalize poorly.
\begin{proposition}
  \label{prop:InterpolationNoGen}
  Suppose further that the samples are distributed uniformly on $\bbS^d$ and the noise is non-zero.
  Then, for any $\ep > 0$ and $\delta \in (0,1)$, there is some $c > 0$ such that when $n$ and $m$ is sufficiently large,
  one has that
  \begin{align*}
    \E \zkm{\liminf_{t \to \infty} \norm{\hat{f}^{\mathrm{NN}}_t - f^*}_{L^2}^2 ~\Big|~ \X} \geq c n^{-\ep}
  \end{align*}
  holds with probability at least $1-\delta$.
\end{proposition}

\begin{remark}
  \cref{prop:InterpolationNoGen} seems to contradict with the ``benign overfitting'' phenomenon~(e.g., \citet{bartlett2020_BenignOverfittinga,frei2022_BenignOverfitting}).
  However, we point out that in these works the dimension $d$ of the input diverges with the sample size $n$,
  while in our case $d$ is fixed, so the setting is different.
  In fact, in the fixed-$d$ scenario, several works have argued that overfitting is harmful~\citep{rakhlin2018_ConsistencyInterpolation,beaglehole2022_KernelRidgeless,li2023_KernelInterpolation}
  and our result is consistent with theirs.
\end{remark}

\begin{remark}
  The requirement of uniformly distributed samples on the sphere is due to the technical condition of the embedding index in \citet{li2023_KernelInterpolation},
  which is critical for more refined analysis in the kernel regression~\citep{fischer2020_SobolevNorm}.
  With this condition, the requirement of $s$ in \cref{prop:NN_Gen} can further be relaxed to $s > 0$.
  We hypothesize that this embedding index condition is also satisfied for the NTK on a general domain,
  but we would like to leave it to future work since more techniques on function theory are needed.
\end{remark}

%\begin{remark}
%
%%  In order to ensure generalization of neural networks, early stopping is used as an implicit regularization.
%%  Specifically, in the case of uniformly distributed data on $\bbS^d$ and neural networks without bias terms,
%%  it is possible to demonstrate that overfitted multilayer neural networks cannot generalize by combining
%%  \cref{thm:UnifConverge} and the results in \textcolor{green}{Li2023Interpolation},
%%  thus extending the assertion made in \citet{lai2023_GeneralizationAbility}.
%%  We believe this result can be extended to general domains, but further techniques are required for a comprehensive investigation,
%%  which we leave for future work.
%%  For simplicity, let us consider uniformly distributed data on $\bbS^d$ and neural networks without the bias terms.
%%  Then, combining with \cref{thm:UnifConverge} and the results in \textcolor{green}{Li2023Interpolation},
%%  it can be shown that overfitted multilayer neural networks can not generalize,
%%  extending the assertion in \citet{lai2023_GeneralizationAbility}.
%%  We hypothesize that this result can be extended to general domains,
%%  but we would like to leave it for future work since further techniques are needed.
%
%\end{remark}

%\subsection{Choosing stopping time with cross validation}
%
%The previous sections have provided a priori selection rules for determining the optimal stopping time.
%In this section, we will demonstrate that the optimal stopping time can also be determined using a posteriori cross-validation.
%This result is also based on the theory of kernel regression~\citep{caponnetto2010_CrossvalidationBased}.
%
%Let us further assume that $y \in [-M,M]$ almost surely for some $M$.
%We introduce the truncation $L_{M}(a)=\min\{|a|,M\}\operatorname{sgn}(a)$ and the truncated regressor $L_M \hat{f}_t(\x)$,
%where $\hat{f}_t$ can be either $\hat{f}^{\mathrm{GD}}_t$ or $\hat{f}^{\mathrm{NN}}_t$.
%Suppose now we have $\tilde{n}$ extra independent samples $(\tilde{\x}_1,\tilde{y}_1),\dots,(\tilde{\x}_{\tilde{n}},\tilde{y}_{\tilde{n}})$,
%where $\tilde{n} \geq c_{\mathrm{v}} n$ for some constant $c_{\mathrm{v}} > 0$.
%Then, we can choose the empirical stopping time by cross validation
%\begin{align}
%  \label{eq:5_StoppingTimeCV}
%  \hat{t}_{\mathrm{cv}} = \argmin_{t \in T_n}
%  \sum_{i=1}^{\tilde{n}} \left( L_{M}\hat{f}_t(\tilde{\x}_i) - \tilde{y}_i \right)^2,
%\end{align}
%where $T_n$ is a set of stopping time candidates.
%The regressor with empirical stopping time is given by
%\begin{align}
%  \label{eq:5_CVRegressor}
%  \hat{f}_{\mathrm{cv}}(\x) \coloneqq L_M \hat{f}_{\hat{t}_{\mathrm{cv}}}(\x).
%\end{align}
%\citet[Section 3]{caponnetto2010_CrossvalidationBased} shown the following optimality result.
%
%\begin{proposition}
%  \label{prop:5_CVOptimal}
%  Suppose that $k$ has eigenvalue decay rate $\beta > 1$ with respect to $\mu$
%  and $f^* \in B_R([\caH]^s)$ for some $s \geq 1$.
%%  Let $s \geq 1$ and $\gamma \in [0,1]$.
%%  Suppose conditions (\hyperlink{cond:A}{$\bm{A}$}) and (\hyperlink{cond:B}{$\bm{B}$}) hold.
%%  Let $T_n = \left\{ 1,\dots,n \right\}$.
%  Let us fix arbitrary $Q>1$ and define
%  \begin{align}
%    \label{eq:5_CVSet}
%    T_n = \left\{ 1, Q, \dots,Q^{\lfloor \ln_Q n \rfloor} \right\}.
%  \end{align}
%  Then, for any $\delta \in (0,1)$, for sufficiently large $n$, one has
%  \begin{align*}
%    \norm{\hat{f}_{\mathrm{cv}}^{\mathrm{GD}} - f^*}_{L^2}^2
%    \leq C \left( \ln \frac{6}{\delta} \right)^2 n^{-\frac{s\beta}{s\beta+1}}
%  \end{align*}
%  holds with probability at least $1-\delta$, where the constant $C>0$ is independent of $\delta,n$.
%\end{proposition}
%
%Consequently, we can show the optimality of neural network with cross validation.
%Similarly, we choose the empirical stopping time via \cref{eq:5_StoppingTimeCV} for $\hat{f}^{\mathrm{NN}}_t$
%and denote by $\hat{f}^{\mathrm{NN}}_{\mathrm{cv}}$ the resulting regressor in \cref{eq:5_CVRegressor}.
%\begin{theorem}[Empirical stopping time]
%  \label{thm:NN_CV}
%  Suppose $\mu$ has Lebesgue density $p(\x)$ with $0 < g \leq p(\x) \leq G < \infty$ and
%  $f^* \in B_R([\caH_{\mathrm{NTK}}]^s)$ for some $s \geq 1$.
%%  Under the conditions of \cref{thm:NN_Gen},
%  For the neural network $\hat{f}^{\mathrm{NN}}_{\mathrm{cv}}$
%  with empirical stopping time chosen from $T_n$ given in \cref{eq:5_CVSet}, there exists a polynomial $\operatorname{poly}(\cdot)$, such that
%  when $n$ is sufficiently large and $m \geq \mathrm{poly}(n,\ln(1/\delta),\lambda_{0}^{-1})$,
%  one has
%  \begin{align*}
%    \norm{\hat{f}^{\mathrm{NN}}_{\mathrm{cv}} - f^*}_{L^2}^2 \leq C \left( \ln\frac{12}{\delta} \right)^2 n^{-\frac{s(d+1)}{s(d+1)+d}}
%  \end{align*}
%  with probability at least $1-\delta$ with respect to random samples and initialization,
%  where the constant $C>0$ is independent of $\delta,n$.
%\end{theorem}

%\begin{remark}
%  \cref{prop:NN_CV} shows that if cross validation is used to select the stopping time, wide neural networks can be minimax optimal.
%  This result justifies the effectiveness of cross validation in practice.
%%  Moreover, slight modification of the proof of \cref{prop:5_CVOptimal} can show a broad class of parameter set $T_n$
%%  is also admissible, including $T_n = \left\{ 1,\dots,n \right\}$.
%\end{remark}








