
In this paper,
we provide a strategy to determine the eigenvalue decay rate (EDR) of a large class of kernel functions defined on a general domain rather than $\mathbb S^{d}$.
This class of kernel functions include but are not limited to the neural tangent kernel associated with neural networks with different depths and various activation functions.
After proving that the dynamics of training the wide neural networks uniformly approximated that of the neural tangent kernel regression on general domains,
we  can further illustrate  the minimax optimality of the wide neural network provided that the underground truth function
$f\in [\mathcal H_{\mr{NTK}}]^{s}$, an interpolation space associated with the RKHS $\mathcal{H}_{\mr{NTK}}$ of NTK\@.
We also showed that the overfitted neural network can not generalize well.
We believe our approach for determining the EDR of kernels might be also of independent interests. 








%
%In this paper, we consider the generalization ability of deep wide feedforward ReLU neural networks defined on a bounded domain $\mathcal X \subset \mathbb R^{d}$.
%We first demonstrate that the generalization ability of the neural network can be fully characterized by that of the corresponding deep neural tangent kernel (NTK) regression.
%We then investigate on the spectral properties of the deep NTK and show that the deep NTK is positive definite on $\mathcal{X}$ and its eigenvalue decay rate (EDR) is $(d+1)/d$.
%Thanks to the well established theories in kernel regression, we then conclude that multilayer over-parameterized neural networks trained by gradient descent with proper early stopping achieve the minimax rate,
%provided that the regression function lies in the reproducing kernel Hilbert space (RKHS) associated with the corresponding NTK\@.
%Finally, we illustrate that the overfitted multilayer wide neural networks can not generalize well on $\mathbb S^{d}$.
%We believe our technical contributions in determining the EDR of NTK on $\mathbb R^{d}$ might be of independent interests.
%{
%%  \color{blue}
%}